\documentclass[a4paper,10pt]{article}
\usepackage[italian]{babel} 
\usepackage[T1]{fontenc} 
\usepackage[utf8]{inputenc} 
\usepackage{amsfonts}
\usepackage{amsthm}
\usepackage[margin=25mm]{geometry}
\usepackage{mathtools}
\usepackage{listings}
\usepackage{centernot}
\usepackage{multicol}
\usepackage{pgfplots}
\usepackage{soul}


%opening
\title{Giorno 6}
\author{Lorenzo Pace}

\theoremstyle{definition}
\newtheorem{deff}{Def.}[section]
\newtheorem{lemma}[deff]{Lemma}
\newtheorem{es}[deff]{Es.}
\newtheorem{ex}[deff]{Esercizio}
\newtheorem{teo}[deff]{Teorema}
\newtheorem{prop}[deff]{Proposizione}
\lstset
{ %Formatting for code in appendix
    language=Pascal,
    otherkeywords={print, ref, nil, return, elif}, 
    basicstyle=\footnotesize,
    numbers=left,
    stepnumber=1,
    showstringspaces=false,
    tabsize=1,
    breaklines=true,
    breakatwhitespace=false,
}
\renewcommand{\qedsymbol}{\rule{1ex}{1ex}}
\setlength{\parindent}{0em}
\usetikzlibrary{shapes}
\usetikzlibrary{shapes.misc}

\newcommand{\reals}{\mathbb{R}}
\newcommand{\rationals}{\mathbb{Q}}
\newcommand{\integers}{\mathbb{Z}}
\newcommand{\naturals}{\mathbb{N}}

\tikzset{cross/.style={cross out, draw=black, minimum size=2*(#1-\pgflinewidth), inner sep=0pt, outer sep=0pt},
%default radius will be 1pt. 
cross/.default={1pt}}


\begin{document}

\begin{center}
    \LARGE Giorno 6
\end{center}



\begin{multicols}{2}
 \section{Alberi Binari di Ricerca}
Un albero binario di ricerca (\emph{ABR}, o \emph{BST}, \emph{Binary Search Tree}) è un albero binario tale che, per ogni nodo interno $v$, tutti gli elementi del sottoalbero sinistro di $v$ hanno chiave $< v.key$, e tutti gli elementi del sottoalbero destro hanno chiave $\geq v.key$.
\bigskip



\begin{center}
 
\begin{tikzpicture}[level distance=2.4em, every node/.style = {shape=rectangle, align=center, circle, draw=black!60, thick, minimum size=2mm}, level 1/.style={sibling distance=5em},
  level 2/.style={sibling distance=2.5em},
  level 3/.style={sibling distance=30pt},
  level 4/.style={sibling distance=3em}]]
  \node (Root) {$30$}
    child{node{$10$}
        child{node{$5$}}
        child{node{$20$}}}
    child{node{$40$}
        child{node{$37$}}
        child{node{$50$}
            child[missing]{}
            child{node{$69$}}}};

\end{tikzpicture}

\end{center}

\end{multicols}
\subsection{Ricerca in un ABR}
La ricerca di un elemento $k$ di un ABR è, come suggerisce il nome, la ricerca binaria:
\begin{quote}
 \begin{lstlisting}[escapeinside={(?}{?)}]
ABR-Search(v, k):
    if(v (?==?) nil): return nil;
    if(v.key (?==?) k): return v;
    if(k (?$\geq$?) v.key): return ABR-Search(v.right, k);
    else: return ABR-Search(v.left, k);
 \end{lstlisting}

\end{quote}
Come la ricerca binaria su array, la ricerca binaria ha un tempo di esecuzione $O(h)$ = $O(\log n)$ al caso medio, dove $h$ è l'altezza dell'albero.

\subsection{Inserimento in un ABR}
\begin{quote}
 \begin{lstlisting}[escapeinside={(?}{?)}]
ABR-Insert(r, k):
    n = new nodo di chiave k;
    v = r;
    u = nil;
    while(v (?$\neq$?) nil):
        u = v;
        if(k (?$\geq$?) v.key): v = v.right;
        else: v = v.left;
    if(k (?$\geq$?) u.key): u.right = n;
    else: u.left = n;
        
 \end{lstlisting}
\end{quote}
Anche questa procedura ha costo $O(h)$, poiché al caso peggiore si inserisce in un elemento del livello più basso, il che comporta $h$ iterazioni del while.
\subsection{Max, Min, Successore, Predecessore}
Il massimo di un albero di radice $v$ è l'elemento più a destra (min duale), costo $O(h)$;
\begin{multicols}{2}

\begin{quote}
 \begin{lstlisting}[escapeinside={(?}{?)}]
ABR-Max(r):
    v = r;
    while(v.right (?$\neq$?) nil):
        v = v.right;
    return v;
 \end{lstlisting}
\end{quote}
\begin{quote}
 \begin{lstlisting}[escapeinside={(?}{?)}]
ABR-Min(r):
    v = r;
    while(v.left (?$\neq$?) nil):
        v = v.left;
    return v;
 \end{lstlisting}
\end{quote}
\end{multicols}
Il successore può essere il minimo del sottoalbero destro, o in assenza di questo l'ancestor più vicino che contiene $v$ nel suo sottoalbero sinistro. (pred duale)
\begin{multicols}{2}
 
 \begin{lstlisting}[escapeinside={(?}{?)}]
ABR-Succ(v, k):
    if(v.right (?$\neq$?) nil):
        return ABR-Min(v.right);
    else:
        p = v.p;
        u = v;
        while(p (?$\neq$?) nil && p.left (?$\neq$?) u):
            u = p;
            p = u.p;
        return p;
 \end{lstlisting}

 \begin{lstlisting}[escapeinside={(?}{?)}]
ABR-Pred(v, k):
    if(v.left (?$\neq$?) nil):
        return ABR-Max(v.left);
    else:
        p = v.p;
        u = v;
        while(p (?$\neq$?) nil && p.right (?$\neq$?) u):
            u = p;
            p = u.p;
        return p;
 \end{lstlisting}

\end{multicols}

\subsection{Eliminazione}
Ci sono tre casi:
\begin{enumerate}
 \item Il nodo $v$ da eliminare non ha figli: in tal caso semplicemente si rimuovono i riferimenti dal parent.
 \item $v$ ha un figlio: si sostituisce il figlio a $v$, correggendo il riferimento al parent del figlio.
 \item $v$ ha due figli: si copia il valore (+ dati satelliti) del successore di $v$ in $v$, si pone il successore di $v$ uguale al suo figlio destro (anche $nil$). Sicuramente il successore ha al più un figlio destro poiché se avesse un figlio sinistro il successore sarebbe lui.

\end{enumerate}

\begin{quote}
 \begin{lstlisting}[escapeinside={(?}{?)}]
ABR-Delete(v, k):
    if(v (?è?) una foglia):
        if(v.p.right (?==?) v): v.p.right = nil;
        else: v.p.left = nil;
    elif(v ha un figlio dx):
        if(v.p.right (?==?) v): v.p.right = v.right;
        else: v.p.left = v.right;
    elif(v ha un figlio sx):
        if(v.p.right (?==?) v): v.p.right = v.left;
        else: v.p.left = v.left;
    else:
        u = ABR-Succ(v);
        <copio key e dati satelliti di u (?in?) v>;
        u = u.left;
        
 \end{lstlisting}
\end{quote}


\begin{center}

 
\begin{tikzpicture}[level distance=2.4em, every node/.style = {shape=rectangle, align=center, circle, draw=black!60, thick, minimum size=2mm}, level 1/.style={sibling distance=5em},
  level 2/.style={sibling distance=2.5em},
  level 3/.style={sibling distance=30pt},
  level 4/.style={sibling distance=3em}]]
  \node (Root) {$30$}
    child{node{$10$}
        child{node{$5$}}
        child{node{$20$}}}
    child{node{$40$}
        child{node{$37$}
            child[missing]{}
            child{node{$38$}}}
        child{node{$50$}
            child[missing]{}
            child{node{$69$}}}};

\end{tikzpicture}
\begin{tikzpicture}[level distance=2.4em, every node/.style = {shape=rectangle, align=center, circle, draw=black!60, thick, minimum size=2mm}, level 1/.style={sibling distance=5em},
  level 2/.style={sibling distance=2.5em},
  level 3/.style={sibling distance=30pt},
  level 4/.style={sibling distance=3em}]]
  \node (Root) [black, fill=lightgray]{$37$}
    child{node{$10$}
        child{node{$5$}}
        child{node{$20$}}}
    child{node{$40$}
        child{node[black, fill=lightgray]{$37$}
            child[missing]{}
            child{node{$38$}}}
        child{node{$50$}
            child[missing]{}
            child{node{$69$}}}};

\end{tikzpicture}
\begin{tikzpicture}[level distance=2.4em, every node/.style = {shape=rectangle, align=center, circle, draw=black!60, thick, minimum size=2mm}, level 1/.style={sibling distance=5em},
  level 2/.style={sibling distance=2.5em},
  level 3/.style={sibling distance=30pt},
  level 4/.style={sibling distance=3em}]]
  \node (Root) {$37$}
    child{node{$10$}
        child{node{$5$}}
        child{node{$20$}}}
    child{node{$40$}
        child{node[black, fill=lightgray]{$38$}
            child[missing]{}
            child{node[cross, thin]{$38$}}}
        child{node{$50$}
            child[missing]{}
            child{node{$69$}}}};

\end{tikzpicture}
\bigskip



 Esempio, cancellazione di 30
 
\end{center}

La cancellazione ha il costo della ricerca del successore, ossia $O(h)$.\smallskip

Il problema degli ABR è che spesso tendono a diventare molto sbilanciati; servirebbero degli alberi con le stesse proprietà degli ABR, ma necessariamente bilanciati (alberi 2-3).

\section{Alberi 2-3}
\begin{multicols}{2}

Gli alberi 2-3 sono alberi tali che \textbf{ogni nodo interno abbia 2 o 3 figli, e che tutte le foglie siano sullo stesso livello}. I valori inseriti si trovano \textbf{tutti nelle foglie}, ordinati per chiave da sinistra verso destra, mentre ogni nodo interno $x$ ha due attributi: 

\begin{enumerate}
 \item $x.S$ è il \textbf{massimo del sottoalbero sinistro};
 \item $x.M$, se il nodo ha tre figli, indica il \textbf{massimo del sottoalbero centrale}.
\end{enumerate}
\smallskip

Spesso l'$i$-esimo figlio (ordinato) del nodo $x$ è indicato con $x_i$.\smallskip
\begin{center}
 \scriptsize
\begin{tikzpicture}[level distance=5em, every node/.style = {shape=rectangle, align=center, circle, draw=black!60, thick, minimum size=2mm}, level 1/.style={sibling distance=8em},
  level 2/.style={sibling distance=3em},
  level 3/.style={sibling distance=30pt},
  level 4/.style={sibling distance=3em}]]
  \node (Root) {$11, 15$}
    child{node[minimum size=8mm]{$7, 10$}
        child{node{7}}
        child{node{10}}
        child{node{11}}}
    child{node[minimum size=8mm]{$12$}
        child{node{12}}
        child{node{15}}}
    child{node[minimum size=8mm]{$18, 21$}
        child{node{18}}
        child{node{21}}
        child{node{24}}};

\end{tikzpicture}

\smallskip

\small
L'attributo a sinistra dei nodi con tre figli è $S$, l'attributo a destra $M$. $M$ è nullo nel nodo $(12)$.
\end{center} 
\end{multicols}


Tra le principali \textbf{proprietà} abbiamo:
\begin{enumerate}
 \item L'altezza $h$ di un albero 2-3, rispetto al numero di foglie $f$ è tale che:
 $\log_2 f\leq h \leq \log_3 f$
 \item Il numero di nodi $n$ è tale che:
$2^{h+1}-1 \leq n \leq \dfrac{3^{h+1} - 1}{2}$
\end{enumerate}
\subsection{Ricerca in un Albero 2-3}
In modo simile alla ricerca binaria (e ternaria), si parte dalla radice e si considera, guardando $S$ ed $M$, in quale sottoalbero si potrebbe trovare la chiave cercata.

\begin{quote}
 \begin{lstlisting}[escapeinside={(?}{?)}]
Search(r, k):
    if(r (?è?) una foglia): return r.key == k ? r : nil;
    if(k (?$\leq$?) r.S): return Search((?$r_1$?), k);
    elif(r ha due figli || k (?$\leq$?) r.M): return Search((?$r_2$?), k);
    else: : return Search((?$r_3$?), k);
 \end{lstlisting}
\end{quote}

Il costo della ricerca al caso pessimo è $O(h)$.

\subsection{Inserimento e cancellazione}
\subsubsection{Inserimento}
Per inserire un elemento $x$ in un albero 2-3, si cerca anzitutto la chiave dell'elemento, per individuare il nodo $p$ che ne sarebbe padre. Poi ci sono due possibilità: 
\begin{enumerate}
 \item $p$ ha due figli: Si inserisce ordinatamente (scambiando poi l'ordine dei figli se necessario) il terzo figlio $x$.
 \item $p$ ha tre figli: Si applica la procedura split:
 \begin{enumerate}
  \item Si inserisce ordinatamente $x$ come quarto figlio temporaneo di $p$.
  \item Si creano due nuovi nodi $n, m$ da due figli, tali che $n$ ha come figli $p_1$ e $p_2$, ed $m$ ha come figli $p_3$ e $p_4$ e si pongono come figli del padre di $p$:
  \begin{enumerate}
   \item Se il padre di $p$ aveva due figli, adesso ne ha 3 ok.
   \item Se il padre di $p$ aveva tre figli, adesso ne ha 4, si applica split al padre di $p$. Se questo era la radice, si crea una nuova radice.
  \end{enumerate}

 \end{enumerate}
\end{enumerate}
\begin{center}
\tiny

 \begin{tikzpicture}[level distance=5em, every node/.style = {align=center}, level 1/.style={sibling distance=5em},
  level 2/.style={sibling distance=2em},
  level 3/.style={sibling distance=30pt},
  level 4/.style={sibling distance=3em}]]
  \node (Root)[circle, draw=black] {$11, 15$}
    child{node[circle, draw=black, minimum size=8mm]{$7$}
        child{node[rectangle, draw=black]{7}}
        child{node[rectangle, draw=black]{11}}}
    child{node[circle, draw=black, minimum size=8mm]{$12$}
        child{node[rectangle, draw=black]{12}}
        child{node[rectangle, draw=black]{15}}}
    child{node[circle, draw=black,fill=lightgray, minimum size=8mm]{$18, 21$}
        child{node[rectangle, draw=black]{18}}
        child{node[rectangle, draw=black]{21}}
        child{node[rectangle, draw=black]{24}}};

\end{tikzpicture}
\hspace{1mm}
 \begin{tikzpicture}[level distance=5em, every node/.style = {align=center}, level 1/.style={sibling distance=6em},
  level 2/.style={sibling distance=2em},
  level 3/.style={sibling distance=30pt},
  level 4/.style={sibling distance=3em}]]
  \node (Root)[circle, draw=black] {$11, 15$}
    child{node[circle, draw=black, minimum size=8mm]{$7$}
        child{node[rectangle, draw=black]{7}}
        child{node[rectangle, draw=black]{11}}}
    child{node[circle, draw=black, minimum size=8mm]{$12$}
        child{node[rectangle, draw=black]{12}}
        child{node[rectangle, draw=black]{15}}}
    child{node[circle, draw=black, fill=lightgray, minimum size=8mm]{$18, 21$}
        child{node[rectangle, draw=black]{18}}
        child{node[rectangle, draw=black]{21}}
         child{node[rectangle, draw=black, fill=lightgray]{22}}
        child{node[rectangle, draw=black]{24}}};

\end{tikzpicture}
\hspace{1mm}
 \begin{tikzpicture}[level distance=5em, every node/.style = {align=center}, level 1/.style={sibling distance=4em},
  level 2/.style={sibling distance=2em},
  level 3/.style={sibling distance=30pt},
  level 4/.style={sibling distance=3em}]]
  \node (Root)[circle,fill=lightgray, draw=black] {$11, 15$}
    child{node[circle, draw=black, minimum size=8mm]{$7$}
        child{node[rectangle, draw=black]{7}}
        child{node[rectangle, draw=black]{11}}}
    child{node[circle, draw=black, minimum size=8mm]{$12$}
        child{node[rectangle, draw=black]{12}}
        child{node[rectangle, draw=black]{15}}}
    child{node[circle, draw=black,fill=lightgray, minimum size=8mm]{$18$}
        child{node[rectangle, draw=black]{18}}
        child{node[rectangle, draw=black]{21}}}
     child{node[circle, draw=black,fill=lightgray, minimum size=8mm]{$22$}
         child{node[rectangle, draw=black]{22}}
        child{node[rectangle, draw=black]{24}}};

\end{tikzpicture}
\hspace{1mm}
 \begin{tikzpicture}[level distance=5em, every node/.style = {align=center}, level 1/.style={sibling distance=8em},
  level 2/.style={sibling distance=4em},
  level 3/.style={sibling distance=14pt},
  level 4/.style={sibling distance=3em}]]
  \node (Root)[circle, draw=black, fill=lightgray,minimum size=8mm] {$15$}
    child{node[circle, draw=black,fill=lightgray, minimum size=8mm]{$11$}
        child{node[circle, draw=black, minimum size=8mm]{$7$}
        child{node[rectangle, draw=black]{7}}
        child{node[rectangle, draw=black]{11}}}
        child{node[circle, draw=black, minimum size=8mm]{$12$}
        child{node[rectangle, draw=black]{12}}
        child{node[rectangle, draw=black]{15}}}}
    child{node[circle, draw=black,fill=lightgray, minimum size=8mm]{$21$}
        child{node[circle, draw=black, minimum size=8mm]{$18$}
        child{node[rectangle, draw=black]{18}}
        child{node[rectangle, draw=black]{21}}}
        child{node[circle, draw=black, minimum size=8mm]{$22$}
         child{node[rectangle, draw=black]{22}}
        child{node[rectangle, draw=black]{24}}}};

\end{tikzpicture}\smallskip


\small Inserimento di 22
\end{center}

\subsubsection{Cancellazione}
Per la cancellazione di un elemento $x$ ci sono tre possibilità:
\begin{enumerate}
 \item $x$ è la radice $\implies$ l'albero diventa vuoto.
 \item Il padre di $x$ ha tre figli: ne rimangono due ok;
 \item Il padre di $x$ ha due figli: ne rimane 1; si applica la procedura Fuse, duale a Split:
 \begin{enumerate}
  \item Se il fratello ``più vicino'' del padre di $x$ ha 2 figli, lo ``fondiamo'' col padre di $x$, creando un unico nodo che ha come figli i tre figli dei due nodi fusi.
  \item Se il fratello ``più vicino'' del padre di $x$ ha 3 figli, creiamo due nuovi nodi e dividiamo ordinatamente i figli tra questi due.
  \begin{enumerate}
   \item Se il padre di $x.p$ aveva tre figli, adesso ne ha due, ok.
   \item Se il padre di $x.p$ aveva due figli, adesso ne ha uno, si applica Fuse sul padre di $x.p$.
  \end{enumerate}

 \end{enumerate}
 

\end{enumerate}

\end{document}
