 
\documentclass[a4paper,10pt]{article}
\usepackage[utf8]{inputenc}
\usepackage[margin=25mm]{geometry}
\usepackage[italian]{babel}
\usepackage{amsmath}
\usepackage{amsthm}
\usepackage{amsfonts}
\usepackage{centernot}
\usepackage{multicol}
\usepackage{tikz}
\usepackage{listings}
\usepackage{courier}
\usepackage{MnSymbol}
\usetikzlibrary{patterns.meta}
\setlength{\parindent}{0em}
\newcommand{\reals}{\mathbb{R}}
\newcommand{\integers}{\mathbb{Z}}
\newcommand{\naturals}{\mathbb{N}}
\newcommand{\cnot}{\centernot}


\definecolor{backcolour}{rgb}{0.95,0.95,0.92}
\lstdefinestyle{mystyle}{
    language=caml,
    backgroundcolor=\color{backcolour},
    numberstyle=\tiny,
    basicstyle=\ttfamily\small,
    keywordstyle=\bfseries,
    breakatwhitespace=false,
    breaklines=true,
    captionpos=b,
    keepspaces=true,
    numbers=left,
    numbersep=5pt,
    showspaces=false,
    showstringspaces=false,
    showtabs=false,
    tabsize=2
}

\lstset{style=mystyle}

\begin{document}


\begin{center}
    \LARGE Giorno 2

    \Large Implementazione di Interpreti
\end{center}\smallskip

\begin{abstract}
Si presentano un interprete di espressioni aritmetiche, con cenni sulla struttura ed implementazione di scanner e parser, un interprete del $\lambda$-calcolo ed uno di una versione didattica di Caml.
\end{abstract}

\section{Introduzione}

\subsection{Ciclo di interpretazione}
Le fasi dell'interpretazione sono:
\begin{itemize}
 \item Scanner: Analisi lessicale, restituisce token list
 \item Parser: Analisi grammaticale, genera AST
 \item Type checker: Controllo dei tipi
 \item Interprete: restituisce il risultato.
\end{itemize}

\textbf{Scanning e parsing} possono essere realizzati a livello di intero programma o di singole istruzioni/espressioni, e lo stesso vale anche per le altre fasi.
\subsubsection{Esempio: NodeJS vs OCaml}

\emph{Node.js} controlla prima la sintassi di tutto il programma, e poi interpreta:\smallskip

\quad \texttt{console.log(1); console.log(2;} $\to$ Non stampa 1 prima di vedere che manca la parentesi.\medskip

Mentre il toplevel di OCaml esegue il parsing sulle singole espressioni:\smallskip

\quad \texttt{let x = 10;; lett y = 20;;} $\to$ Esegue comunque la prima espressione.

\subsection{Interprete di espressioni aritmetiche}
\subsubsection{Scanner}
\begin{itemize}
 \item Definiamo i token come tipo union:

\quad \texttt{type token =}

\quad \texttt{Tkn\_NUM of int | Tkn\_OP of string | Tkn\_LPAR | Tkn\_RPAR | Tkn\_END;;}

\item Creiamo un'eccezione \texttt{ParseError}:

\quad \texttt{exception ParseError of string * string;;}

\item Che utilizzeremo nel seguente modo:

\quad \texttt{raise (ParseError(``Tokenizer'', ``unknown symbol''\^{}c))}

...all'interno della funzione ricorsiva \texttt{tokenize}, che scansionando i caratteri costruisce una lista di token ad essi corrispondenti, e.g.
\begin{lstlisting}
 match c with
    |" " -> tokens | "(" -> Tkn_LPAR :: tokens | ")" -> Tkn_RPAR
    | "+" | "-" | "*" | "/" -> (Tkn_OP c) :: tokens
    | "0" | "1" | ... | "9" -> Tkn_NUM (cifre consecutive accorpate)
\end{lstlisting}



\end{itemize}
Problema: Implementazione non tail-recursive, ma vabbè.
\newpage
\subsubsection{Parser}
Si realizza un \emph{parser a discesa ricorsiva}:
\begin{itemize}
 \item Si crea una funzione per ogni categoria sintattica 
 
 (nell'esempio: $Exp ::= Term[\pm Exp]$, \quad $Term ::= Factor [*\,\, Term \,\,\vert\,\, / Term]$, \quad $Factor ::= n \,\,|\,\, (Exp)$)
 \item Le funzioni create sono mutuamente ricorsive (\texttt{and})
 \item L'AST corrisponderà all'albero delle chiamate
\end{itemize}

\paragraph{Implementazione}
\begin{itemize}
 \item Si passa la lista dei token per riferimento
 \item Si definiscono due funzioni speciali: \textbf{lookahead}, che permette di vedere qual è il prossimo token da processare senza eliminarlo, e \textbf{consume}, che NON restituisce ma elimina il primo token della lista.
 
 \begin{lstlisting}
  let tokens = ref (tokenize s) in 
  
  let lookahead () = match !tokens with
    | [] -> raise (ParseError("Parser", "lookahead error"))
    | t :: _ -> t
    
  in consume () = match !tokens with
    | [] -> (ParseError("Parser", "consume error"))
    | t::tkns -> tokens := tkns \end{lstlisting}

 
 \item Ad esempio, per gestire le operazioni binarie delle espressioni:
 
 \begin{lstlisting}
  let rec exp () = (* si noti che la funzione ha tipo unit *)
    let t1 = term() in  
        match lookahead () with 
            | Tkn_OP "+" -> consume() ; Op (Add t1, exp())
            | Tkn_OP "-" -> consume() ; Op (Sub t1, exp()) \end{lstlisting}
La chiamata ricorsiva ad \texttt{exp} in \texttt{Op (Add t1, exp())} fa un passo della costruzione dell'AST.
\item L'AST completo viene ottenuto facendo: \texttt{let ast = exp ()} e in seguito controllando se una chiamata di \texttt{lookahead()} restituisce solo il token di fine. In caso contrario non tutti i token sono stati valutati, e c'è stato perciò un parse error.
\end{itemize}

\subsubsection{Nota sui parser}
L'implementazione dei parser non è sempre così semplice; si usano solitamente dei \emph{parser generator} che li creano per noi (non argomento del corso). Nei linguaggi che implementeremo partiremo perciò dall'AST, e non implementeremo un parser.
\newpage
\subsubsection{Interprete}
\label{bs_ss}
Vi sono due approcci alla definizione della semantica di un linguaggio:

\begin{itemize}
 \item Approccio \textbf{big-step}:
 La relazione di transizione descrive in un solo passo \emph{l'intera computazione}.
 
 (le singole operazioni sono descritte nell'albero di derivazione della transizione)
 
 \[n \to n \quad\quad \dfrac{E_1 \to n_1 \quad E_2 \to n_2 \quad n_1 \text{ op }n_2 = n}{E_1 \text{ op }E_2 \to n} \] 
 \item Approccio \textbf{small-step}: Ogni passo della relazione di transizione esegue \emph{una singola operazione}.
 
 \[ n \to n \quad \quad \dfrac{E_1 \to E_1 '}{E_1 \text{ op }E_2 \to E'_1 \text{ op } E_2} \]

 \[\dfrac{E_2 \to E_2 '}{n \text{ op }E_2 \to n \text{ op } E_2'} \quad \quad \dfrac{n_1 \text{ op } n_1 = n}{n_1\text{ op } n_2 \to n} \]
 
\end{itemize}
 \paragraph{Metodo sistematico per passare da semantica a codice}
 \begin{itemize}
  \item Si fa pattern matching sull'espressione, creando un caso per ogni tipo di nodo dell'AST (nel nostro caso i nodi possono essere \emph{operazioni} o \emph{valori}, le espressioni sono i sottoalberi dell'AST)
  \item  Si verificano le pre-condizioni delle varie regole (non assiomi), chiamando ricorsivamente l'interprete.
  \item Quando le pre-condizioni sono verificate, si calcola il risultato della transizione.
 \end{itemize}

 [Esempi di interprete big-step vs small-step: listing \texttt{[16]} e \texttt{[18]} di ``Introduzione allo sviluppo di interpreti'']

 \section{Interprete del $\lambda$-calcolo}
 \subsection{Sintassi}
  \[ e := x \,\,|\,\, \lambda x.e \,\,|\,\, e\,\,e \]
  
  \begin{lstlisting}
   type id = string;;
   type exp = Var of id | Lam of id * exp | App of exp * exp\end{lstlisting}

 \subsection{Semantica}
  Realizziamo una versione deterministica della semantica che abbiamo visto:
  
   \[ (\lambda x.e_1)e_2 \to  e_1 \{ x:=e_2 \}\]
 
  \[ \dfrac{e_1 \to e'}{e_1e_2 \to e'e_2} \quad\quad\quad \dfrac{e_2 \to e'}{e_1e_2 \to e_1'} \quad\quad\quad \dfrac{e \to e'}{\lambda x.e \to \lambda x .e'}\]
  
  Al fine di realizzare un interprete del $\lambda$-calcolo dobbiamo anzitutto implementare l'operazione più complessa della semantica del $\lambda$-calcolo: la \textbf{capture-avoiding substitution}.\smallskip
 
 Partiamo dalla definizione formale:
 
 \[\begin{aligned}
   x\{x := e\} \quad &\equiv\quad  e\\
   y\{x := e\} \quad &\equiv\quad  y, \text{ se } y \neq x\\
   (e_1 e_2)\{x := e\}\quad &\equiv\quad (e_1\{x:=e\})(e_2\{x:=e\})\\
   (\lambda y.e_1)\{ x:=e \}\quad &\equiv\quad  \begin{cases}
                                                    \lambda y.(e_1\{ x:=e \}) &\text{se $y \neq x$ e $y \notin FV(e)$}\\
                                                    \lambda z.((e_1\{ x:=z \})\{ x:=e \}) &\text{se $y \neq x$ e $y \in FV(e)$, z fresca}
                                               \end{cases}
  \end{aligned}
\]
 
 E notiamo che possiamo implementare senza problemi questa definizione tramite il pattern matching, ma è necessario definire un metodo per creare variabili ``fresche'' e per costruire l'insieme (per i nostri scopi basta una lista) delle variabili libere di un'espressione.
 \newpage
 \subsubsection{Implementazione di FV}
 La definizione dell'insieme delle variabili libere è la seguente:
 
\[\begin{aligned}
   &FV(x) = \{x\}\\
&FV(e_1 e_2) = FV(e_1) \cup FV(e_2)\\
&FV(\lambda x.e) = FV(e) \setminus \{x\}\\
  \end{aligned}
\]

In OCaml:

\begin{lstlisting}
let rec fvs e = 
    match e with
        | Var x -> [x]
        | Lam (x, e) -> List.filter (fun y -> x <> y) (fvs e)
        | App (e_1 e_2) -> (fvs e_1) @ (fvs e_2)
;;\end{lstlisting}
\subsubsection{Generatore di id freschi}
\begin{lstlisting}
let newvar = 
    let x = ref 0 in
        fun () ->
            let c = !x in 
            incr x;
            "v"^(string_of_int x)
\end{lstlisting}

Ogni volta che è chiamata genera una variabile \texttt{v$n$}, partendo da \texttt{v0}. Riesce a mantenere $x$ poiché è un riferimento ed è sempre nella chiusura della funzione.\smallskip

A questo punto abbiamo tutti i mezzi per implementare la capture avoiding substitution, realizzando tutti i casi tramite il pattern matching (si veda il listing \texttt{[6]} di ``Un interprete del $\lambda$-calcolo'')\medskip

La semantica è implementata usando il \emph{metodo sistematico} descritto in \ref{bs_ss} (si veda il listing \texttt{[8]}). È interessante notare che non è possibile implementare precisamente la semantica non deterministica che abbiamo definito; si deve dare priorità ad una delle regole di valutazione dell'applicazione, e in base all'ordine dei pattern si può scegliere se dare priorità alla riduzione o all'applicazione funzionale.\bigskip

L'implementazione completa dell'interprete mostrato a lezione è reperibile qua:

\texttt{https://www.cs.umd.edu/class/spring2019/cmsc330/lectures/interp.ml}

\newpage
\section{MiniCaml}
\subsection{Ambiente}
\paragraph{Perché il $\lambda$-calcolo non ha l'ambiente?}Nel $\lambda$-calcolo l'ambiente non era necessario, poiché i binding tra identificatori e valori erano implementati tramite la sostituzione\medskip

A differenza dall'interprete del $\lambda$-calcolo, per realizzare l'interprete del \emph{MiniCaml} serve un'implementazione dell'ambiente. Potremmo realizzarlo come:
\begin{itemize}
 \item Lista di coppie $(nome, valore)$, \small (non ciò che useremo)
 \item \normalsize\textbf{Funzione Polimorfa} $\Gamma : Ide -> Value \cup \{Unbound\}$; 
 
 $\Gamma(x)$ denota il valore $v$ associato ad $x$ nell'ambiente, oppure, se non esiste il binding, il valore speciale \emph{Unbound}. La funzione $\Gamma$ può essere estesa con un legame, notazione $\Gamma[x = v]$, e:
 
 \[ \Gamma [x = v](y) = \begin{cases}
                            v & \text{se }y = x\\
                            \Gamma(y) &\text{se } y \neq x
                        \end{cases}
 \]
\end{itemize}

\begin{lstlisting}
 (* 't = tipo dei valori esprimibili *)
 type 't env = ide -> 't;; 
 (* ambiente vuoto *)
 let emptyenv = fun x -> Unbound;;
 
 (* aggiornamento ambiente s con (x, v) *)
 let bind s x v = 
    fun i -> if (i = x) then v else (s i);;
\end{lstlisting}
{\it
\paragraph{Nota} Il tipo dell'ambiente è polimorfo perché sarà definito in mutua ricorsione con il tipo dei valori esprimibili (\texttt{Closure} e \texttt{RecClosure}, vedi \ref{closure})
}
\subsection{Il linguaggio}
\label{il_ling}
\begin{lstlisting}
type ide = string
type exp =
    | CstInt of int                 (* costante Int *)
    | CstTrue                       (* costante True *)
    | CstFalse                      (* costante False *)
    | Sum of exp * exp
    | Diff of exp * exp
    | Prod of exp * exp
    | Div of exp * exp
    | Eq of exp * exp
    | Iszero of exp
    | Or of exp * exp
    | And of exp * exp
    | Not of exp
    | Den of ide                    (* Entita' denotabile (variabile) *)
    | Ifthenelse of exp * exp * exp
    | Let of ide * exp * exp        (* Dichiarazione di ide: modifica ambiente *)
    | Fun of ide list * exp         (* Astrazione di funzione *)
    | Apply of exp * exp list       (* Applicazione di funzione *)
\end{lstlisting}

Si noti che \texttt{Fun} ha come ``ramo argomenti'' una lista; questa è la lista dei parametri. In realtà questa parte sarà semplificata, e useremo funzioni di una sola variabile.\smallskip
\newpage

\subsection{Espressioni}

Per il momento lavoriamo solo con le espressioni, (fino a riga 16 del listing in \ref{il_ling}). Dato che abbiamo più di un tipo di dato esprimibile, dobbiamo fare del \textbf{typechecking}.
\setlength\columnsep{20pt}
\begin{multicols}{2}
\textbf{Valori esprimibili / descrittori di tipo}
\begin{lstlisting}
 (* Possibili risultati delle valutazioni di espressioni *)
 type evT = Int of int
          | Bool of bool
          | Unbound
\end{lstlisting}

\textbf{Tipi esistenti}
\begin{lstlisting}
type tname = 
 | TInt
 | TBool
 | ...
 | ...
\end{lstlisting}

\end{multicols}

Tramite il typechecking vogliamo vedere se un descrittore di tipo ha effettivamente il tipo che vogliamo: controlliamo perciò una \textbf{coppia \texttt{(type, typeDescriptor)}}.\smallskip

\textbf{Codice del typechecking}
\begin{multicols}{2}

\begin{lstlisting}
let typecheck (type, typeDescriptor) =
    match type with
    | TInt ->
        (match typeDescriptor with
            | Int(u) -> true
            | _ -> false)
    | TBool ->
        (match typeDescriptor with
            | Bool(u) -> true
            | _ -> false)
    | _ -> failwith ("not a valid type");;

\end{lstlisting}
\begin{itemize}
 \item Se il tipo è int:
 \begin{itemize}
  \item Se il typeDescriptor matcha un intero, allora il tipo corrisponde
  \item Se no non corrisponde
 \end{itemize}
 \item Se il tipo è bool:
 \begin{itemize}
  \item Se il typeDescriptor matcha un booleano, allora il tipo corrisponde
  \item Se no non corrisponde
  \item[]
 \end{itemize}
\end{itemize}

\end{multicols}
\paragraph{Implementazione delle operazioni di base} Le operazioni sono implementate come funzioni:
\begin{lstlisting}
let is_zero x = match (typecheck(TInt,x), x) with
    | (true, Int(y)) -> Bool(y=0)
    | (_, _) -> failwith("run-time error");;
    
let int_eq(x,y) =
    match (typecheck(TInt,x), typecheck(TInt,y), x, y) with
    | (true, true, Int(v), Int(w)) -> Bool(v = w)
    | (_,_,_,_) -> failwith("run-time error ");;
    
let int_plus(x, y) =
    match(typecheck(TInt,x), typecheck(TInt,y), x, y) with
    | (true, true, Int(v), Int(w)) -> Int(v + w)
    | (_,_,_,_) -> failwith("run-time error ");;
\end{lstlisting}

Quindi la funzione \texttt{eval}, che prenderà come parametri un'espressione el'ambiente, sarà del tipo:
\begin{lstlisting}
 let rec eval (e:exp) (amb: evT env) =
    match e with
        | CstInt(n) -> Int(n)
        | CstTrue -> Bool(true)
        | CstFalse -> Bool(false)
        | Iszero(e1) -> is_zero(eval e1 amb)
        | Eq(e1, e2) -> int_eq((eval e1 amb), (eval e2 amb))
        | Sum(e1, e2) -> int_plus ((eval e1 amb), (eval e2 amb))
        | ...
\end{lstlisting}\smallskip
\textbf{Binding di identificatori}


Per coprire anche il caso di \texttt{e = Den i} (entità denotabile), aggiungiamo anche

\begin{lstlisting}
 Den (i) -> amb i
\end{lstlisting}
Che restituisce il valore associato all'identificatore \texttt{i} nell'ambiente \texttt{amb}.

\newpage
\paragraph{If then else}
L'if viene implementato, a differenza delle altre espressioni, con una strategia \emph{non-eager} (per evitare di valutare il ramo che non ci interessa, che e.g. potrebbe non terminare)
\begin{lstlisting}
 Ifthenelse(cond,e1,e2) ->
let g = eval cond amb in
match (typecheck("bool", g), g) with
| (true, Bool(true)) -> eval e1 amb
| (true, Bool(false)) -> eval e2 amb
| (_, _) -> failwith ("nonboolean guard")
\end{lstlisting}

\subsection{Variabili, funzioni}
\subsubsection{Let: blocco}
Il \texttt{let x = e1 in e2} valuta l'espressione \texttt{e1}, trasformandola nel valore \texttt v, e calcola l'espressione \texttt{e2} a partire dall'ambiente $\Gamma$ a cui aggiungo il binding \texttt{(x, v)}

\[ \dfrac{\Gamma \triangleright e_1 \to v_1 \quad \quad \Gamma [x = v_1] \triangleright e_2 \to v_2}{\Gamma \triangleright Let (x, e_1, e_2) \to v_2} \]

(dove $\Gamma \triangleright e$ significa ``$e$, valutata nell'ambiente $\Gamma$'')

\paragraph{Nota} L'operazione di estensione \textbf{corrisponde} alla push di un record di attivazione. Per come è implementato l'ambiente, l'estensione crea una nuova \emph{funzione ambiente}, costruita a partire dall'ambiente precedente, che sarà ``dimenticata'' all'uscita del blocco (pop);

\begin{lstlisting}
 let rec eval((e: exp), (amb: evT env)) =
    match e with
    ...
    ...
    | Let(i, e, ebody) ->
        eval ebody (bind amb i (eval e amb))
\end{lstlisting}

\subsubsection{Funzioni}
Consideriamo le funzioni con un solo parametro:
\begin{center}
\texttt{Fun of ide * exp} \quad\quad \texttt{Apply of exp * exp}
\end{center}

Dato che adesso dobbiamo esprimere anche tipi funzione, dobbiamo \textbf{estendere i valori esprimibili}:
\label{closure}
\begin{lstlisting}
type evT = Int of int | Bool of bool | Unbound | Closure of ide * exp * evT env
\end{lstlisting}

Una \emph{chiusura} avrà perciò la forma \texttt{(x, e, env)}, dove \texttt x è il parametro formale della funzione, \texttt e è il corpo della funzione e \texttt{env} è l'ambiente che era attivo alla definizione della funzione ($\Gamma_{decl}$).\medskip

\textbf{Dichiarazione di funzione:}
\[ \Gamma \triangleright Fun (x, e) \to Closure (\texttt{"x"}, e, \Gamma) \]
Vado a \textbf{ripescare la funzione} dall'ambiente (chi è ``Var''? forse Den boh possibile errore nelle slide):
\[ \Gamma \triangleright Var (\texttt{"f"} \to Closure(\texttt{"x"}, body, \Gamma_{fDecl}))\]
\textbf{Applicazione di funzione}, dove $v_a$ è il ``valore attuale'' ottenuto valutando il parametro attuale:
\[ \dfrac{\Gamma \triangleright arg \to v_a\quad \quad \Gamma_{fDecl}[x = v_a] \triangleright body \to v}{\Gamma \triangleright Apply(Den(\texttt{"f"}), arg) \to v} \]

Spiegazione di Milazzo\footnote{meglio di mille spiegazioni mie -- (dalle slide ``MiniCAML\_parte1'')}:

\begin{quote}\it
 Il corpo della funzione viene valutato nell’ambiente
ottenuto legando il parametro formale al valore del parametro
attuale nell’ambiente nel quale era stata valutata l’astrazione.
\end{quote}

\newpage
\textbf {Implementazione: funzioni}
\begin{lstlisting}
let rec eval (e: exp) (amb: evT env) =
    match e with
        | ...
        | Fun(i, a) -> Closure(i, a, amb)
        | Apply(Den(f), eArg) ->
            let fclosure = amb f in
            (match fclosure with
                | Closure(arg, fbody, fDecEnv) ->
                    let aVal = eval eArg amb in
                    let aenv = bind fDecEnv arg aVal in
                        eval fbody aenv
                | _ -> failwith("non functional value"))
        | Apply(_,_) -> failwith("Application: not first order function") ;;
\end{lstlisting}

\paragraph{Nota sullo scoping dinamico} La differenza tra \textbf{scoping statico} e \textbf{scoping dinamico} è che con lo scoping statico l'ambiente è costruito in base \emph{alla struttura del programma} (\underline{come nel nostro caso}), mentre con lo scoping dinamico l'ambiente è costruito in base \emph{al flusso del codice}; quest'ultimo è più semplice da implementare, basterebbe infatti usare l'ambiente corrente invece di $\Gamma_{fDecl}$. \smallskip

Devo perciò modificare \texttt{evT}, sostituendo \texttt{Closure} con \texttt{Funval of ide * exp}, non c'è più bisogno dell'ambiente.

\subsection{Funzioni ricorsive}
\subsubsection{Let rec}
Definiamo un costrutto \texttt{let rec}, che allo stesso tempo ha la funzione di creare un blocco e di creare una funzione con nome:
\begin{lstlisting}
 type exp =
    ...
    | Letrec of ide * ide * exp * exp
\end{lstlisting}

In particolare: \texttt{Letrec("f", "x", fbody, letbody)}, dove $f$ è il nome della funzione, $x$ è il parametro formale, $fbody$ è il corpo della funzione, e $letbody$ è il corpo del let.

\subsubsection{Valori esprimibili}
Dobbiamo estendere ulteriormente i valori esprimibili:
\begin{lstlisting}
 type evT = | Int of int | Bool of bool
            | Unbound | Closure of ide * exp * evT env
            | RecClosure of ide * ide * exp * evT env
\end{lstlisting}
 $\implies$ \texttt{RecClosure(funName, param, funBody, staticEnvironment)}\smallskip

Si noti che RecClosure in realtà esprime semplicemente la chiusura di una funzione con nome, che può benissimo anche non essere ricorsiva.

\subsubsection{Codice dell'interprete}
Il codice dell'interprete è simile a quello delle funzioni anonime non ricorsive, ma valuto \texttt{letbody} nell'ambiente $\Gamma[f = RecClosure(f, i, fBody, amb)]$, per rendere possibile la ricorsione:\smallskip

\textbf{Dichiarazione di funzione ricorsiva:}
\begin{lstlisting}
...
| Letrec(f, i, fBody, letBody) ->
    let benv =
        bind amb f (RecClosure(f, i, fBody, amb))
            in eval letBody benv
...
\end{lstlisting}

\newpage
\textbf{Applicazione di funzione ricorsiva}

\begin{lstlisting}
| Apply(Den f, eArg) ->
    let fclosure = amb f in
        match fclosure with
            | Closure(arg, fbody, fDecEnv) ->
                ...
                ...
            | RecClosure(f, arg, fbody, fDecEnv) ->
                let aVal = eval eArg amb in
                    let rEnv = bind fDecEnv f fclosure in
                        let aEnv = bind rEnv arg aVal in
                            eval(fbody, aEnv)
        | _ -> failwith("non functional value")
| Apply(_,_) -> failwith("not function")
\end{lstlisting}

Unica differenza dalle funzioni non ricorsive: bind di $f$ nell'ambiente \texttt{(9)} prima del bind dell'argomento.

\subsection{Funzioni di ordine superiore}
Per ora non possiamo passare funzioni come argomento, ossia trattarle come valori di prima classe.\smallskip

Per ora, infatti, (si veda riga \texttt{(5)} del listing in \ref{closure}) la apply deve avere la forma \texttt{Apply (Den (f), eArg)}, se no si ha un errore a runtime (\emph{``not first order function''}). \medskip

Per risolvere basta:
\begin{itemize}
 \item Accettare anche espressioni \texttt{eF} 
 \item Valutarle prima di fare il patternmatching con le chiusure.

\end{itemize}
\begin{lstlisting}
| Apply(eF, eArg) ->
    let fclosure = eval eF amb in
        (match fclosure with
            | Closure(arg, fbody, fDecEnv) ->
                let aVal = eval eArg amb in
                    let aenv = bind fDecEnv arg aVal in
                        eval fbody aenv
            | RecClosure(f, arg, fbody, fDecEnv) ->
                let aVal = eval eArg amb in
                    let rEnv = bind fDecEnv f fclosure in
                        let aenv = bind rEnv arg aVal in
                            eval fbody aenv
            | _ -> failwith("non functional value")) ;;
\end{lstlisting}
\newpage
\subsection*{Esempio Run-Time Stack}
Si fornisce adesso un'idea di come funzionano i record di attivazione per i linguaggi non interpretati:

\begin{itemize}
 \item Un record di attivazione consiste di uno Static Link (catena statica), che punta al record di attivazione del chiamante, e di un valore (valore aggiunto all'ambiente)
 \item Se il valore non è funzionale, si mantiene nello stack
 \item Se il valore è funzionale (\textbf{non ricorsivo}), si mantiene una coppia \texttt{<f, A>}, dove $f$ è un puntatore al codice della funzione, ed $A$ è una copia dello static link, ossia un puntatore all'ambiente in cui dobbiamo eseguire la funzione.
 \item Se il valore funzionale è \textbf{ricorsivo}, non possiamo eseguire la funzione nel record di attivazione puntato dallo static link: dobbiamo copiare un puntatore al record di attivazione corrente, in cui è definito il nome della funzione.
\end{itemize}

\textbf{Problema}: Che succede se una funzione cerca di restituire un valore funzionale?\smallskip

\emph{Il RdA che contiene l'ambiente a cui quella funzione fa riferimento potrebbe essere stato cancellato!} 

$\implies$ \textbf{Soluzione}: mettiamo un flag ai record di attivazione, che indica se quel RdA è in uso; in tal caso non lo cancelliamo.



\end{document}
