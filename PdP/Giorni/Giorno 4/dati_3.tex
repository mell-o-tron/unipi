 
 
\documentclass[a4paper,10pt]{article}
\usepackage[utf8]{inputenc}
\usepackage[margin=25mm]{geometry}
\usepackage[italian]{babel}
\usepackage{amsmath}
\usepackage{amsthm}
\usepackage{amsfonts}
\usepackage{centernot}
\usepackage{multicol}
\usepackage{tikz}
\usepackage{listings}
\usepackage{courier}
\usepackage{MnSymbol}
\usetikzlibrary{patterns.meta}
\setlength{\parindent}{0em}
\newcommand{\reals}{\mathbb{R}}
\newcommand{\integers}{\mathbb{Z}}
\newcommand{\naturals}{\mathbb{N}}
\newcommand{\cnot}{\centernot}


\definecolor{backcolour}{rgb}{0.95,0.95,0.92}
\lstdefinestyle{mystyle}{
    language=caml,
    backgroundcolor=\color{backcolour},
    numberstyle=\tiny,
    basicstyle=\ttfamily\small,
    keywordstyle=\bfseries,
    breakatwhitespace=false,
    breaklines=true,
    captionpos=b,
    keepspaces=true,
    numbers=left,
    numbersep=5pt,
    showspaces=false,
    showstringspaces=false,
    showtabs=false,
    tabsize=2
}

\lstset{style=mystyle}
\lstset{
 morekeywords={module, sig, struct}
}
\begin{document}


\begin{center}
    \LARGE Giorno 4\smallskip

    \Large Dati
\end{center}\smallskip
\begin{abstract}
 
\end{abstract}

\section{Polimorfismo}
\subsection{Idea}
Fin ora abbiamo considerato il polimorfismo \emph{parametrico}, quello in cui si hanno tipi con un parametro (e.g. la $<T>$ dei generics in Typescript.)\smallskip

Vogliamo edesso parlare di \textbf{polimorfismo per sottotipo}, ossia vogliamo definire, ad esempio, delle funzioni che accettano come parametro sia un certo tipo $T$ che i suoi ``sottotipi'', che non sono altro che versioni più specializzate di un certo tipo. 

\paragraph{Esempio classico} Se si hanno:

\begin{itemize}
 \item Il record ``persona'', con i campi ``nome'' e ``cognome''
 \item Il record ``studente'' uguale a ``persona'' ma con in più un campo ``matricola''
\end{itemize}
Allora ``studente'' può essere visto come \textbf{sottotipo} di ``persona'', scriviamo:
\[ \texttt{studente } <: \texttt{ persona} \]

\subsection{Sottotipi}
\subsubsection{Subsumption}
La regola di \emph{subsumption} ci permette di applicare funzioni che accettano come parametro sia dati di un tipo che dati dei suoi sottotipi.
\[ \dfrac{\Gamma \vdash e : S \quad S <: T}{\Gamma \vdash e :T} \]

Ossia ``se un'espressione è di tipo $S$ e $S<:T$ allora $e$ è anche di tipo $T$''
\subsubsection{Definizione della relazione di sottotipo}
Data la subsumption, il problema diventa la definizione della relazione di sottotipo. Alcuni linguaggi (class based) utilizzano l'estensione, altri (object based) il subtyping strutturale (i.e. $A$ sottotipo di $B$ se ha i campi di $B$ + altri campi). Un esempio di subtyping strutturale è la relazione di sottotipo per i record:
\paragraph{Subtyping dei record}
    \[ \{ l_i = v_i\quad i = 1, \hdots, k+n \} <: \{ l_i = v_i \quad i = 1, \hdots, k \} \]\smallskip
    \[ \dfrac{\{ k_i = u_i\quad i = 1, \hdots, n \} \text{ permutazione di }\{ l_j = v_j \quad j = 1, \hdots, n \}}{\{ k_i = u_i\quad i = 1, \hdots, n \} <: \{ l_j = v_j \quad j = 1, \hdots, n \}} \]\smallskip

Caso particolare - il record ha campi di tipo record:
    \[ \dfrac{\forall i \quad S_i <: T_i}{\{ l : S_i\quad i = 1, \hdots, n \} <: \{ l_j : T_j\quad j = 0, \hdots, n \}} \]
    
e.g. una lista di studenti è sottotipo di una lista di persone.

\subsubsection{Top}
È conveniente avere un tipo che sia supertipo di ogni tipo: lo chiamiamo \textit{\textbf{Top}} (Object in Java)\newpage

\subsection{Funzioni}
\subsubsection{Subsumption per le funzioni}
Vogliamo che una funzione $S_1 \to S_2$ possa prendere un argomento di tipo $T_1 <: S_1$ e restituire un risultato di tipo $T_2 :> S_2$

\[ \dfrac{T_1 <: S_1 \quad\quad S_2 <: T_2}{S_1 \to S_2 <: T_1 \to T_2} \]

Diciamo che la relazione di sottotipo è 
\begin{itemize}
 \item \textbf{Covariante} per il risultato delle funzioni, dove \emph{covariante} significa che la relazione d'ordine è conservata; in generale per un operatore $F$:
 
 \[ \dfrac{S :< T}{F\langle S\rangle <: F \langle T\rangle } \]
 
 Dove con $F\langle T \rangle$ indico un operatore che prende (o restituisce) tipo $T$ o un \textbf{costruttore di tipo} che prende come parametro\footnote{Vedi: \texttt{en.wikipedia.org/wiki/Type\_constructor} \quad e\quad \texttt{en.wikipedia.org/wiki/List\_(abstract\_data\_type)}} $T$.
 \item \textbf{Controvariante} per il parametro della funzione, dove \emph{controvariante} significa che la relazione di sottotipo è invertita.
\end{itemize}

 \subsection{Altri esempi}
 \subsubsection{Classi in Java}
 In java una sottoclasse \textbf{non} può cambiare i tipi dei parametri dei \textbf{metodi}, ma può rendere più specifico il tipo del risultato (\textbf{covariante} sul risultato (come le funzioni prima), e \textbf{invariante} sugli argomenti):
 \[ \dfrac{S_1 <: T_1}{m : S \to S_1 <: m : S \to T_1} \]
 \subsubsection{Liste}
 \[ \dfrac {S <: T}{List\,\,S <: List\,\,T} \]
List è un costruttore di tipo covariante.

\subsection{Riferimenti e puntatori}
\subsubsection{Riferimenti vs puntatori}
Riferimenti e puntatori sono simili, ma la differenza sta nel fatto che per i puntatori esiste un'\textbf{aritmetica dei puntatori}, i.e. si possono fare operazioni aritmetiche sulle locazioni di memoria, mentre i riferimenti permettono solo di creare delle variabili che accedono alla stessa locazione di memoria.

\subsubsection{Aliasing}

Problema: \[\lambda r : Ref\,\,Nat . \lambda s Ref\,\, Nat . (r := 2 , s:=3; !r)\]

Restituisce 2, a meno che $s$ non sia un alias di $r$.\smallskip

I compilatori effettuano analisi degli alias per cercare di stabilire quando variabili diverse non si riferiscono alla stessa memoria.
\subsubsection{Regole di tipo}
\[\dfrac{\Gamma \vdash e_1 : \tau_1}{\Gamma \vdash ref\,\,e_1 : ref\,\,\tau_1} \hspace{2cm} \dfrac{\Gamma \vdash e_1 : ref\,\,\tau_1}{\tau \vdash !e_1 : \tau_1} \hspace{2cm} \dfrac{\Gamma \vdash e_1 : ref\,\, \tau_1 \quad\quad \Gamma \vdash e_2 : \tau_1}{\Gamma \vdash e_1 := e_2 : unit}\]

\newpage
\section{Abstract Data Types}
Un \textbf{tipo di dato astratto} consiste di un \textbf{insieme di dati} e una \textbf{collezione di operazioni} per operare sui dati di quel tipo. 
\begin{itemize}
 \item \textbf{estendibili}: si possono costruire nuovi ADT 
 \item \textbf{Astratti}: Rappresentazione nascosta agli utenti
 \item \textbf{Incapsulati}: Si opera con i dati solo attraverso operazioni dell'ADT
\end{itemize}
\subsection{Specifica ed implementazione}
La \textbf{specifica} descrive la semantica delle operazioni di un tipo di dato astratto; uno stesso ADT può avere diverse implementazioni, ma questa differenza sarà invisibile al programmatore.
\subsubsection{Approcci alla definizione di ADT}
\begin{itemize}
 \item \textbf{Signature + Axioms}: si definiscono le operazioni e i loro \emph{tipi}, e delle proprietà astratte da rispettare nella forma di \emph{assiomi}
 
 \begin{lstlisting}
  Stack <E>
    Signature                           //Operazioni significative
        new     : -> STACK
        push    : E, STACK
        top     : STACK -> E
        pop     : STACK -> STACK
        isEmpty : STACK -> Bool
        undef_e : -> E                  //Completezza (elem/stack vuoto)
        undef_s : -> STACK
    
    Axioms                              // Proprieta' astratte da rispettare
        forall e: E, stk: STACK
        top(push(e, stk)) = e ;
        top(new) = undef_e ;
        pop(push(e, stk)) = stk ;
        pop(new) = undef_s ;
        isEmpty(new) ;
        ~isEmpty(push(e, stk))
 \end{lstlisting}

\end{itemize}
\subsection{Moduli in OCaml}
\subsubsection{Signature}
\begin{lstlisting}
 module type BOOL = sig
    type t
    val yes : t
    val no  : t
    val choose : t -> 'a -> 'a -> 'a
 end
\end{lstlisting}

\subsubsection{Esempio: Implementazioni equivalenti di BOOL}
\begin{multicols}{2}
 \begin{lstlisting}
 module M1 : BOOL = struct
    type t = unit option
    let yes = Some ()
    let no  = None
    let choose v ifyes ifno =
        match v with
        | Some () -> ifyes
        | None -> ifno
 end\end{lstlisting}

 \begin{lstlisting}
  module M1 : BOOL = struct
    type t = int
    let yes = 1
    let no  = 0
    let choose v ifyes ifno =
        match v with
        | 1 -> ifyes
        | 0 -> ifno
 end\end{lstlisting}

\end{multicols}

\end{document}
