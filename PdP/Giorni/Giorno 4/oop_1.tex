 
 
\documentclass[a4paper,10pt]{article}
\usepackage[utf8]{inputenc}
\usepackage[margin=25mm]{geometry}
\usepackage[italian]{babel}
\usepackage{amsmath}
\usepackage{amsthm}
\usepackage{amsfonts}
\usepackage{centernot}
\usepackage{multicol}
\usepackage{tikz}
\usepackage{listings}
\usepackage{courier}
\usepackage{MnSymbol}
\usetikzlibrary{patterns.meta}
\setlength{\parindent}{0em}
\newcommand{\reals}{\mathbb{R}}
\newcommand{\integers}{\mathbb{Z}}
\newcommand{\naturals}{\mathbb{N}} 
\newcommand{\cnot}{\centernot}


\definecolor{backcolour}{rgb}{0.95,0.95,0.92}
\lstdefinestyle{mystyle}{
    language=caml,
    backgroundcolor=\color{backcolour},
    numberstyle=\tiny,
    basicstyle=\ttfamily\small,
    keywordstyle=\bfseries,
    breakatwhitespace=false,
    breaklines=true,
    captionpos=b,
    keepspaces=true,
    numbers=left,
    numbersep=5pt,
    showspaces=false,
    showstringspaces=false,
    showtabs=false,
    tabsize=2
}

\lstset{style=mystyle}
\lstset{
 morekeywords={module, sig, struct, Some, None, method, val, object, class, new}
}
\begin{document}


\begin{center}
    \LARGE Giorno 4\smallskip

    \Large OOP
\end{center}\smallskip
\begin{abstract}
 Nel paradigma orientato agli oggetti, il sistema software è caratterizzato da un insieme di oggetti cooperanti, ognuno con i propri attributi (stato) ed i loro metodi (funzioni che agiscono (o meno, vd \texttt{static}) sullo stato). Vi sono due approcci diversi all'OOP, uno che si basa sulla manipolazione degli oggetti (object-based, usa subtyping strutturale e prototipi per l'ereditarietà), ed uno in cui gli oggetti sono istanze delle classi, (class-based, subtyping nominale, extends). Si studiano, inoltre, gli aspetti OOP di OCaml.
\end{abstract}

\section{Il paradigma orientato agli oggetti}

Nel paradigma orientato agli oggetti il sistema software è caratterizzato da un insieme di \textbf{oggetti cooperanti}; ogni oggetto è caratterizzato da:
\begin{itemize}
 \item \textbf{Stato}: Gli attributi/proprietà/variabili dell'oggetto
 \item \textbf{Funzionalità}: i \textbf{metodi}/funzioni dell'oggetto.
 \item \textbf{Nome}: individua l'oggetto
 \item \textbf{Ciclo di vita}: Un oggetto può essere creato, riferito, disattivato
 \item \textbf{Locazione} di memoria.
\end{itemize}

Lo \textbf{stato del programma} consiste, idealmente, dell'insieme degli stati degli oggetti (anche se nella pratica a questo si aggiungono le strutture dati per il run-time support)

\subsection{Classi ed Oggetti}
Vi sono due approcci principali all'OOP:
\begin{itemize}
 \item \textbf{Object-based} Come js (anche se dal 2016 supporta anche le classi)
 \item \textbf{Class-based} Come Java.
\end{itemize}

\subsubsection{Approccio Object Based}
Nell'approccio object based gli oggetti sono trattati nel linguaggio in modo simile ai record, con le seguenti differenze:
\begin{itemize}
 \item I campi possono essere associati a funzioni
 \item Esiste \textbf{this}/self, tramite il quale un metodo può accedere ai campi dell'oggetto stesso.
 \item Js permette di modificare la struttura dell'oggetto dinamicamente, e.g. aggiungendo campi.
\end{itemize}

\subsubsection{Approccio Class Based}

Una classe definisce il contenuto degli oggetti di un certo tipo, e poi gli oggetti sono creati come istanze di una certa classe (e.g. \textbf{new})

\subsubsection{Differenze tra i due approcci}

\begin{itemize}
 \item L'approccio object-based consente al programmatore di lavorare con gli oggetti in modo più flessibile, ma rende difficile predire con precisione quale sarà il tipo di un oggetto (\textbf{ostacola controlli statici})
 \item L'approccio class-based richiede al programmatore una maggiore disciplina e consente di fare \textbf{controlli di tipo statici} sugli oggetti: il tipo di un oggetto sarà legato alla classe da cui è stato istanziato (\textbf{nominal typing}, vd \ref{nominal}) 
\end{itemize}
\newpage
\subsection{Inheritance e subtyping}
La scelta tra object-based e class-based ha impatto sui meccanismo di ereditarietà e sottotipatura del linguaggio:
\begin{itemize}
 \item I linguaggi \textbf{object-based} utilizzano la \emph{prototype-based inheritance} e lo \emph{structural (sub)typing}.
 \item I linguaggi \textbf{class-based} utilizzano la \emph{class-based inheritance} ed il \emph{nominal subtyping}
\end{itemize}

E voi direte: che sono? E io dirò:

\subsubsection{Ereditarietà}
L'ereditarietà (inheritance) è una funzionalità che permette di definire una classe/un oggetto a partire da un'altra/o esistente.

\paragraph{Ereditarietà per i linguaggi object-based:} Per ogni oggetto si mantiene una lista di prototipi, ossia di oggetti da cui esso eredita funzionalità. Esiste un oggetto da cui tutti gli oggetti ereditano funzionalità, ed è solitamente detto \textbf{object}. In js \texttt{\_\_proto\_\_} è usato per risalire di un livello la catena dei prototipi.

\paragraph{Ereditarietà per i linguaggi class-based:} I linguaggi class-based consentono di definire una classe come estensione di un'altra. La nuova classe eredita tutti i membri (valori e metodi) della precedente, con la possibilità di aggiungerne o ridefinirne alcuni. 

\subsubsection{Subtyping}
\begin{itemize}
 \item \textbf{Strutturale}: usato da linguaggi object based: un oggetto $B$ è sottotipo di $A$ se ha tutti i membri \textbf{pubblici} di $A$ + eventualmente altri.
 \item \label{nominal}\textbf{Nominale}: usato da linguaggi class-based: il tipo di un oggetto corrisponde alla classe da cui è stato istanziato: \textbf{nome della classe = tipo}. Un tipo-classe $B$ è sottotipo di un tipo-classe $A$ se la classe $B$ è estensione di $A$.
\end{itemize}


Di nuovo, il meccanismo usato dal \textbf{class-based} è più rigoroso, mentre l'altro è più flessibile.

\begin{itemize}
 \item Js ed OCaml adottano lo \textbf{structural} (anche se altri aspetti dell'ereditarietà di OCaml, non per gli oggetti, sono basati su simil-extends\footnote{qua immagino si faccia riferimento alla creazione di record con ``with'', ma potrei sbagliarmi.})
\end{itemize}


\newpage

\subsection{OOP di OCaml}

Un \textbf{oggetto} in ocaml è un valore fatto di campi e metodi. Gli oggetti possono essere creati direttamente (non ci sono classi). Il \textbf{tipo} di un oggetto è dato dai \textbf{metodi} che esso contiene (i \textbf{campi non influiscono} sul tipo). Esempio:

\begin{lstlisting}
 (* oggetto che realizza uno stack *)
 let s = object
    (* campo mutabile che contiene la rappresentazione dello stack*)
    val mutable v = [0; 2] (* Assumiamo per ora inizializzato non vuoto *)
    (* metodo pop *)
    method pop =
        match v with
        | hd :: tl ->
        v <- tl;
        Some hd
        | [] -> None
    (* metodo push *)
    method push hd =
        v <- hd :: v
 end ;;
\end{lstlisting}

Questo oggetto ha tipo: 
\begin{lstlisting}
 val s : < pop : int option; push : int -> unit > = <obj>

\end{lstlisting}

\paragraph{Note sintattiche}
\begin{itemize}
 \item Nei metodi senza parametri non è necessario aggiungere ()
 \item Non c'è this, i campi sono visibili nei metodi.
 \item Per invocare i metodi si usa la \texttt{\#} notation invece della dot notation.
\end{itemize}

\subsubsection{Digressione: type weakening}

Che sarebbe successo se avessi inizializzato il campo \texttt v dell'oggetto \texttt s con la lista vuota? (listing precedente, riga 4). Quale tipo verrebbe inferito per \texttt s?\smallskip

\textbf{Non sappiamo quale sia il tipo della lista}, ma sappiamo che non potrà cambiare, una volta assegnato, quindi non è un tipo veramente polimorfo!\smallskip

Il tipo degli elementi della lista è temporaneamente settato a \texttt{'\_weak}, per poi \textbf{essere ricalcolato al primo utilizzo}.

\begin{lstlisting}
 val s : < pop : '_weak option; push : '_weak -> unit > = <obj>
\end{lstlisting}

\subsubsection{Costruzione di oggetti tramite funzioni}
Possiamo restituire un oggetto come risulatato di una funzione; tra i parametri potranno ad esempio esserci dei valori di inizializzazione:


\begin{lstlisting}
 (* funzione che costruisce oggetti inizializzati con init *)
 let stack init = object
    val mutable v = init

    method pop =
        match v with
         | hd :: tl ->
            v <- tl;
            Some hd
         | [] -> None
    method push hd =
        v <- hd :: v
 end ;;
\end{lstlisting}

\newpage
\subsubsection{Polimorfismo di oggetti}
Se definiamo una funzione il cui corpo accede ad un metodo di un \textbf{oggetto che non è ancora stato definito}, il tipo del metodo è inferito se possibile, se no sarà polimorfo.\smallskip

Una \textbf{notazione} interessante è la seguente:

\begin{lstlisting}
let area sq = sq#width * sq#width ;;
\end{lstlisting}
Il cui tipo è:
\begin{lstlisting}
 val area : < width : int; .. > -> int = <fun>
\end{lstlisting}

Le \textbf{parentesi angolate} indicano che l'oggetto su cui è chiamata la funzione deve avere \textbf{almeno} ($\implies$ \textbf{subtyping strutturale}) un campo \texttt{width} di tipo intero.\smallskip

Possiamo anche definire tipi utilizzando le parentesi angolate:

\begin{lstlisting}
 type shape = < area : float >
 type square = < area : float; width : int >
\end{lstlisting}

È evidente che \texttt{square} è sottotipo di \texttt{shape}, e quindi ovunque è possibile usare un tipo \texttt{shape} possiamo usare al suo posto un tipo \texttt{square} (\textbf{principio di sostituzione})

\paragraph{Operatore di coercion} La conversione da \texttt{square} a \texttt{shape} non è però implicita: Si deve usare esplicitamente l'\textbf{operatore di coercion} \texttt{:>}

\begin{lstlisting}
 let lis2 = ( square 5 :> shape ) :: lis1 ;;
\end{lstlisting}

Si noti quindi che in OCaml i due concetti di \textbf{polimorfismo di oggetti} (funzioni che prendono oggetti di ``almeno'' un certo tipo) e \textbf{principio di sostituzione} (appena visto) sono trattati in modo diverso.

\subsubsection{Perché OCaml non fa coercion implicita come Java}

OCaml non fa \textbf{mai conversioni di tipo implicite}, mentre Java fa controlli dinamici di tipo svolti a runtime dall'interprete della JVM (resi più facili dal nominal subtyping).

\subsubsection{Classi in OCaml}
Per realizzare i meccanismi di ereditarietà, dato che usa lo stile \emph{object-based}, OCaml dovrebbe usare i prototipi, il cui funzionamento è troppo complicato: per questo sono introdotti anche dei costrutti di \textbf{classe}: Una classe si definisce con \textbf{class} e si instanzia con \textbf{new}

\begin{lstlisting}
 class istack = object                      (* classe per stack di interi *)
    val mutable v = [0; 2]                  (* inizializzato non vuoto *)
    method pop =
        match v with
        | hd :: tl ->
            v <- tl; 
            Some hd
        | [] -> None
    method push hd =
        v <- hd :: v
 end ;;

 let s = new istack ;;
\end{lstlisting}

Le classi possono essere \textbf{parametriche}:
\begin{lstlisting}
 class ['a] stack init = object             (* classe polimorfa per stack *)
    val mutable v : 'a list = init          (* init e' parametro costruttore *)
    method pop = ....
    method push hd = ....
    end;;
\end{lstlisting}

Il tipo di un'istanza della classe è:
\begin{lstlisting}
 let s = new stack ["pippo"] ;;
 (* => val s : string stack = <obj> *)
\end{lstlisting}
\textbf{Nota:} \texttt{string stack} è un \textbf{alias} per \texttt{<pop : string option; push : string -> unit>}


[Ereditarietà con inherit]
\end{document}
