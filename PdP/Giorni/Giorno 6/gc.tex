   
 
\documentclass[a4paper,10pt]{article}
\usepackage[utf8]{inputenc}
\usepackage[margin=25mm]{geometry}
\usepackage[italian]{babel}
\usepackage{amsmath}
\usepackage{amsthm}
\usepackage{amsfonts}
\usepackage{centernot}
\usepackage{multicol}
\usepackage{tikz}
\usepackage{listings}
\usepackage{courier}
\usepackage{MnSymbol}
\usetikzlibrary{patterns.meta}
\setlength{\parindent}{0em}
\newcommand{\reals}{\mathbb{R}}
\newcommand{\integers}{\mathbb{Z}}
\newcommand{\naturals}{\mathbb{N}}
\newcommand{\cnot}{\centernot}
\usepackage{graphicx}
\graphicspath{ {./images/} }

\definecolor{backcolour}{rgb}{0.95,0.95,0.92}
\lstdefinestyle{mystyle}{
    language=java,
    backgroundcolor=\color{backcolour},
    numberstyle=\tiny,
    basicstyle=\ttfamily\small,
    keywordstyle=\bfseries,
    breakatwhitespace=false,
    breaklines=true,
    captionpos=b,
    keepspaces=true,
    numbers=left,
    numbersep=5pt,
    showspaces=false,
    showstringspaces=false,
    showtabs=false,
    tabsize=2
}

\lstset{style=mystyle}

\begin{document}


\begin{center}
    \LARGE Giorno 6\smallskip

    \Large Garbage Collection
\end{center}\smallskip


\section{Garbage Collection}
[tratto solo gli algoritmi, si vedano le prime slide per overview di allocazione in heap motivazioni, definizione di frammentazione interna ed esterna]

\subsection{Reference Counting}
Per ogni cella si mantiene un contatore dei riferimenti a quella cella. Quando questo arriva a zero, la cella diventa garbage e si può liberare.\smallskip

\textbf{Problema:} potrei rimanere con una \textbf{componente connessa} di celle che riferiscono le une alle altre che è irraggiungibile; 

\subsection{Mark-Sweep}
Faccio una visita del grafo delle reference e ``marco'' tutto ciò che è raggiungibile. Il resto è garbage;\smallskip

In questo modo risolvo il problema delle componenti connesse irraggiungibili.\smallskip

\textbf{Problemi:} Come si identificano i puntatori? Si deve avere informazioni sul \textbf{tipo}. 
\begin{itemize}
 \item Java: non è un problema, si hanno informazioni e sappiamo ``chi è una reference''
 \item OCaml: Si sacrifica il bit meno significativo, che è $0$ se e solo se il dato è un puntatore (lo sarebbe comunque, poiché i puntatori sono allineati alle parole, ma si pone uguale a 1 se non si ha un puntatore).
\end{itemize}

\textbf{Altri problemi:} Bisogna sospendere l'esecuzione; Non si interviene sulla \textbf{frammentazione}.

\subsection{Copying collection - Algoritmo di Cheney}
Questo metodo risolve il problema della frammentazione, aggiungendo un po' di overhead e sacrificando metà della memoria.\smallskip

\begin{itemize}
 \item Si suddivide lo heap in due parti uguali, dette ``from-space'' e ``to-space''
 \item Si può allocare solo nel \textbf{from-space}
 \item Periodicamente si copiano tutte le celle raggiungibili nel \textbf{to-space}, deframmentandole
 \item Si scambiano i ruoli di from-space e to-space ad ogni copia; in questo modo abbiamo effettivamente \textbf{perso ogni riferimento irraggiungibile}.
\end{itemize}

\subsection{GC Generazionale}
Si suddivide lo heap in $n$ aree, dove l'area $0$ è detta \texttt{young}, l'area $n-1$ è detta \texttt{old} e le aree in mezzo sono ``generazioni intermedie''.\smallskip

Inizialmente tutti gli oggetti sono creati nell'area \texttt{young}, e periodicamente gli oggetti young raggiungibili sono spostati in un'area ``più old''. Gli oggetti \texttt{young} sono ripuliti più spesso, mentre quelli old meno spesso (ma sono di più, quindi ha costo maggiore);\smallskip

Questo è l'approccio di Java e C\#, che hanno entrambi tre generazioni, ognuna che \textbf{utilizza un algoritmo diverso} per la garbage collection, e.g. Java usa \textbf{copy-collection} per le generazioni 0,1 e mark-sweep (+ meccanismi anti frammentazione) per la generazione 2.

\end{document}
