  
 
\documentclass[a4paper,10pt]{article}
\usepackage[utf8]{inputenc}
\usepackage[margin=25mm]{geometry}
\usepackage[italian]{babel}
\usepackage{amsmath}
\usepackage{amsthm}
\usepackage{amsfonts}
\usepackage{centernot}
\usepackage{multicol}
\usepackage{tikz}
\usepackage{listings}
\usepackage{courier}
\usepackage{MnSymbol}
\usetikzlibrary{patterns.meta}
\setlength{\parindent}{0em}
\newcommand{\reals}{\mathbb{R}}
\newcommand{\integers}{\mathbb{Z}}
\newcommand{\naturals}{\mathbb{N}}
\newcommand{\cnot}{\centernot}


\definecolor{backcolour}{rgb}{0.95,0.95,0.92}
\lstdefinestyle{mystyle}{
    language=java,
    backgroundcolor=\color{backcolour},
    numberstyle=\tiny,
    basicstyle=\ttfamily\small,
    keywordstyle=\bfseries,
    breakatwhitespace=false,
    breaklines=true,
    captionpos=b,
    keepspaces=true,
    numbers=left,
    numbersep=5pt,
    showspaces=false,
    showstringspaces=false,
    showtabs=false,
    tabsize=2
}

\lstset{style=mystyle}

\begin{document}


\begin{center}
    \LARGE Giorno 6\smallskip

    \Large OOP: Principio di Sostituzione di Liskov
\end{center}\smallskip

\begin{abstract}

\end{abstract}

\section{Progettazione e Sviluppo di programmi Java}
\subsection{Fasi di Progettazione e Sviluppo}
\begin{itemize}
 \item Definizione della gerarchia di classi e interfacce
 \item Identificazione dei membri pubblici di ogni classe
 \item Definizione delle specifiche di ogni metodo pubblico (condizioni su parametri e risultato, comportamento atteso del metodo):
 \begin{itemize}
  \item Aggiunta di commenti in cui scriviamo requirements ed effects
  \item Esprimendo condizioni su parametri e variabili (e.g. \textbf{assert})
 \end{itemize}

 \item Testing delle singole classi
 \item Definizione dei membri privati seguendo il principio di incapsulamento
 \item Implementazione del codice, da verificare con i test già sviluppati
\end{itemize}
\subsubsection{Esempio: IntSet}
\begin{lstlisting}
 // OVERVIEW: un IntSet e' un insieme di interi
    public class IntSet {
        public IntSet(int capacity) {
        // REQUIRES:    capacity non negativo
        // EFFECTS:     crea un insieme vuoto che puo' ospitare capacity elementi
    }
    public boolean add(int elem) throws FullSetException {
        // REQUIRES:    numero di elementi contenuti nell'insieme minore di capienza
        // EFFECTS:     se elem non e' presente nell'insieme lo aggiunge e restituisce true,
        //              restituisce false altrimenti
    }
    public boolean contains(int elem) {
        // REQUIRES:
        // EFFECTS:     restituisce true se elem p presente nell'insieme, false altrimenti
    }
 }
\end{lstlisting}

\paragraph{Nota:} \texttt{add} può sollevare un'eccezione quando l'insieme è ``pieno''. Le eccezioni sono gestite in Java come in Javascript, e possono estendere Exception o RuntimeException. Per usare un'eccezione in una classe bisogna dichiararlo nell'intestazione.


\paragraph{JavaDoc}
Esiste una sintassi per le specifiche tramite commenti che permette di generare documentazione:

\begin{lstlisting}
 /**
  * Aggiunge un elemento all'insieme
  * @param elem valore intero
  * @return true se l'inserimento viene aggiunto, false se gia' presente
  * @throws FullSetException se l'insieme e' pieno
  */
public boolean add(int elem) throws FullSetException {
...
}
\end{lstlisting}


\newpage

[Esempio di test e implementazione e altre cose, vd slide]

[vd. Infer, Pathfinder: model checker]

\section{Principio di Sostituzione di Liskov}
\paragraph{Principio di Sostituzione:}
\begin{quote}
 Un oggetto di un \textbf{sottotipo} può sostituire un oggetto del \textbf{supertipo} senza influire sul \textbf{comportamento} dei programmi che usano il supertipo
\end{quote}

\subsection{Differenza dalla subsumption}
\begin{itemize}
 \item La subsumption permette di considerare un tipo $A$ tale che $A <: B$ come di tipo $B$.
 \item Il principio di sostituzione parla di \textbf{comportamento}, e.g. se si hanno due classi ``\texttt{rettangolo}'' e ``\texttt{quadrato}'', che hanno entrambe metodi per calcolare l'area, e $\texttt{quadrato} <: \texttt{rettangolo}$, allora: si può utilizzare un oggetto ``\texttt{quadrato}'' al posto di un \texttt{rettangolo}, ed il comportamento sarà indistinguibile da quello con un \texttt{rettangolo} con tutti i lati uguali.
\end{itemize}

Il principio di Liskov può o meno valere tra due classi, e verificare se vale è un \textbf{problema indecidibile}.
\subsection{Regole indotte dal LSP}
Il principio si traduce nella pratica in regole da seguire:
\subsubsection{Regola della segnatura}
\begin{itemize}
 \item Gli oggetti del sottotipo devono avere tutti i metodi del supertipo
 \item Le segnature (signature) dei metodi del sottotipo devono essere compatibili con quelle corrispondenti del supertipo.
\end{itemize}
La presenza di tutti i metodi è \textbf{garantita dal compilatore Java} tramite i meccanismi di ereditarietà;\smallskip

In caso di overriding, il metodo della sottoclasse deve:
\begin{itemize}
 \item Avere la stessa firma del metodo della superclasse
 \item Sollevare meno eccezioni
 \item Avere un tipo di ritorno più specifico di quello della superclasse

\end{itemize}

\subsubsection{Regola dei metodi}
\begin{itemize}
 \item Le chiamate dei metodi del sotto-tipo devono \textbf{comportarsi} come le chiamate dei corrispondenti metodi del supertipo
\end{itemize}

Devono valere le seguenti regole:
\[ \text{precondizioni}_{super} \implies \text{precondizioni}_{sub} \]
\[ (\text{precondizioni}_{super} \wedge \text{postcondizioni}_{sub}) \implies \text{postcondizioni}_{super} \]

Dove le \textbf{postcondizioni} altro non sono che gli \textbf{effetti} del metodo.
\medskip

Ciò significa che \textbf{le precondizioni possono essere indebolite nel sottotipo, e le postcondizioni possono essere rafforzate.}
\paragraph{Esempio:}
Si hanno due classi: La \textbf{superclasse} \texttt{IntSet} e la sua \textbf{sottoclasse} \texttt{FlexIntSet} (rispettivamente insieme di interi di cardinalità fissata e insieme flessibile di interi):\smallskip

\textbf{Superclasse IntSet}
\begin{lstlisting}
 public boolean add(int elem) throws FullSetException {
    // REQUIRES: numero di elementi contenuti nell'insieme minore di capienza
    // EFFECTS: se elem non e' presente nell'insieme lo aggiunge e restituisce true,
    //          restituisce false altrimenti
\end{lstlisting}

Ossia la precondizione è $size < capacity$, mentre la postcondizione è $retval \in set$

\smallskip
\textbf{Sottoclasse FixedIntSet}
\begin{lstlisting}
 public boolean add(int elem) {
    // REQUIRES:
    // EFFECTS: se elem non e' presente nell'insieme lo aggiunge e restituisce true,
    //          restituisce false altrimenti
\end{lstlisting}

Le precondizioni sono vuote, quindi $true$, mentre le postcondizioni sono sempre $retval \in set$.\medskip

Di conseguenza:

\[ (size < capacity) \implies true \quad\quad (size < capacity) \wedge (retval \in set) \implies (retval \in set) \]

Sono entrambe vere.

\paragraph{Nota: perché seconda parte della regola?}
Ci chiediamo a cosa serve $\text{precondizioni}_{super}$ nella regola:
\[ (\text{precondizioni}_{super} \wedge \text{postcondizioni}_{sub}) \implies \text{postcondizioni}_{super} \]

Nell'esempio sopra, \textbf{dobbiamo poter gestire il caso} $capacity > size$; in tal caso il metodo del supertipo lancerà qualche eccezione forse, mentre il sottotipo non ha problemi a continuare ad aggiungere: non ha precondizioni!

\subsubsection{Regola delle proprietà}
\begin{itemize}
 \item Il sotto-tipo deve preservare tutte le proprietà che possono essere provate sugli oggetti del supertipo.
\end{itemize}

Ciò significa che i \textbf{ragionamenti} sulle proprietà degli oggetti del supertipo sono ancora validi su quelle del sottotipo; 

\paragraph{Esempi di proprietà}
Sempre rispetto all'esempio del punto precedente\smallskip

\textbf{Proprietà invarianti:}
\begin{itemize}
 \item  IntSet ha sempre elementi diversi
 \item Ad un oggetto FlexIntSet è sempre possibile aggiungere nuovi elementi
\end{itemize}
\smallskip

\textbf{Proprietà di evoluzione:}
\begin{itemize}
 \item Il numero di elementi in IntSet non può diminuire nel tempo
 \item Se $add(n) \to true$, allora da quel punto in poi $contains(n) \to true$.
\end{itemize}

\paragraph{Invarianti e incapsulamento}
\begin{itemize}
 \item La rappresentazione deve essere privata! Altrimenti dall'esterno si potrebbe modificare qualcosa modificando l'invariante!
\end{itemize}

\paragraph{ERRORE COMUNE} Anche se la rappresentazione di un attributo è privata, questo potrebbe essere un oggetto (e.g. un array), e se per caso questo oggetto fosse ritornato da una funzione, sarebbe \textbf{passato per riferimento}, errore grave che viola l'incapsulamento.

\begin{lstlisting}
public class IntSet {
    private int[] a;
    public int[] getElements() { return a; } // ERRORE GRAVE!!!
    
...
}
\end{lstlisting}



\end{document}
 
