 
\documentclass[a4paper,10pt]{article}
\usepackage[utf8]{inputenc}
\usepackage[margin=25mm]{geometry}
\usepackage[italian]{babel}
\usepackage{amsmath}
\usepackage{amsthm}
\usepackage{amsfonts}
\usepackage{centernot}
\usepackage{multicol}
\usepackage{tikz}
\usepackage{listings}
\usepackage{courier}
\usepackage{MnSymbol}
\usetikzlibrary{patterns.meta}
\setlength{\parindent}{0em}
\newcommand{\reals}{\mathbb{R}}
\newcommand{\integers}{\mathbb{Z}}
\newcommand{\naturals}{\mathbb{N}}
\newcommand{\cnot}{\centernot}


\definecolor{backcolour}{rgb}{0.95,0.95,0.92}
\lstdefinestyle{mystyle}{
    language=caml,
    backgroundcolor=\color{backcolour},
    numberstyle=\tiny,
    basicstyle=\ttfamily\small,
    keywordstyle=\bfseries,
    breakatwhitespace=false,
    breaklines=true,
    captionpos=b,
    keepspaces=true,
    numbers=left,
    numbersep=5pt,
    showspaces=false,
    showstringspaces=false,
    showtabs=false,
    tabsize=2
}

\lstset{style=mystyle}

\begin{document}


\begin{center}
    \Large Giorno 3 - Dati\smallskip

\end{center}\smallskip
\begin{abstract}
 I dati sono classificabili in base a ciò che ci possiamo fare (associarli ad un nome, ottenerli come soluzione di una valutazione di espressione, memorizzarli), ed i tipi sono classificabili in base a chi li usa (sistema, programma). Un descrittore di dato contiene informazioni sul suo tipo, ed ha vari usi nei controlli a runtime. L'ultima parte del \emph{giorno} consiste di una carrellata di tipi di dato, con un focus particolare sui record, ed accenni sugli array.
\end{abstract}

\section{Dati, tipi, e sistemi di tipo}
\subsection{Classificazione}
I dati possono essere:
\begin{itemize}
 \item \textbf{Denotabili} se possono essere associati ad un nome
 \item \textbf{Esprimibili} se possono essere il risultato della valutazione di un'espressione complessa (diversa dal semplice nome)
 \item \textbf{Memorizzabili} se possono essere memorizzati e modificati in una variabile.
\end{itemize}

\paragraph{Esempio} Le funzioni in ML sono denotabili:
\begin{lstlisting}[language = ml]
let plus (x, y) = x+y
\end{lstlisting}
Esprimibili:
\begin{lstlisting}[language = ml]
let plus = function(x: int) -> function(y:int) -> x + y
\end{lstlisting}
Ma \textbf{non sono memorizzabili} in una variabile.

\subsubsection{Tipi di dato di sistema e di programma}
I tipi di dato \emph{di sistema} sono quelli che la macchina virtuale usa per funzionare (e.g. il run-time stack), mentre quelli \emph{di programma} sono i tipi primitivi del linguaggio (bool, int, list...) e quelli definibili dall'utente.
\subsubsection{Tipo di dato: definizione}
Un tipo di dato è una collezione di valori rappresentato da 
\begin{itemize}
 \item Opportune strutture dati
 \item Un insieme di operazioni
\end{itemize}

Dei tipi di dato ci interessano la semantica (come si comportano) e l'implementazione.
\subsubsection{Descrittori di dato}
Alla rappresentazione concreta di un dato è associata un'altra struttura, detta \emph{descrittore di dato}, che contiene una descrizione del tipo del dato. I descrittori di dato sono utili, ad esempio:
\begin{itemize}
 \item Per fare typechecking dinamico (ossia controllare se l'uso dei tipi in un'operazione è corretto)
 \item Per selezionare l'operatore giusto nel caso di operazioni overloaded (e.g. $+$ sia per somma int che float, che per concatenare...)
\end{itemize}
\subsubsection{Esempi}
\begin{itemize}
 \item OCaml: controllo statico dei tipi $\implies$ i descrittori non servono.
 \item JavaScript: Tutto dinamico, servono i descrittori.
 \item Java: I descrittori contengono solo l'informazione ``dinamica'', perché il type checking è fatto in parte dal compilatore ed in parte dal supporto a run-time. (e.g. array: il controllo dell'accesso out-of-bound è realizzato a run-time)
\end{itemize}
Esempio strano: TypeScript fa i controlli statici, ma poi transpila su js, che fa lo stesso i controlli dinamici!\newpage

\subsection{Panoramica sui tipi di dato}
\subsubsection{Tipi di dato base (scalar)}
\begin{itemize}
 \item Booleani:
 \begin{itemize}
  \item \textbf{Valori}: true, false
  \item \textbf{Operazioni}: or, and, not, condizionali
  \item \textbf{Rappresentazione}: un byte
  \item \textbf{Note}: C non ha un tipo bool.
  \end{itemize}
 \item Char:
 \begin{itemize}
  \item \textbf{Valori}: a, A, b, B ...
  \item \textbf{Operazioni}: uguaglianza; code/decode; dipendono dal linguaggio
  \item \textbf{Rappresentazione}: un byte (ASCII) o due byte (UNICODE)
  \end{itemize}

  \item Interi:
 \begin{itemize}
  \item \textbf{Valori}: 0, 1, -1, 2, -2 ... maxint
  \item \textbf{Operazioni}: +, -, *, /...
  \item \textbf{Rappresentazione}: alcuni byte (2, 4), complemento a 2;
  \item \textbf{Note}: interi e interi lunghi (anche 8 byte)
  \item Limitati problemi nella portabilità quando la lunghezza non è specificata nella
definizione del linguaggio
  \end{itemize}
  \item Reali
 \begin{itemize}
  \item \textbf{Valori}: Valori razionali in un certo intervallo
  \item \textbf{Operazioni}: +, -, *, /...
  \item \textbf{Rappresentazione}: alcuni byte (4), virgola mobile
  \item \textbf{Note}: reali e reali lunghi (8 byte)
  \item Problemi nella portabilità quando la lunghezza non è specificata nella
definizione del linguaggio
  \end{itemize}
\end{itemize}
\subsubsection{Il tipo Void}
Il tipo void (unit), ha come unico valore \texttt{()}, nessuna operazione e serve per definire il tipo di operazioni che modificano lo stato senza restituire alcun valore.\smallskip
\begin{center}
 
\renewcommand{\arraystretch}{1.5}
\begin{tabular}{|c|c|}
\hline
 \textbf{Sintassi espressioni} & \textbf{Valori}\\
 \texttt{e ::= ... | void} & \texttt{v::= ... | void}\\
 \hline
 \textbf{Tipi} & \textbf{Regola di tipo}\\
 \texttt{$\tau$ ::= ... | void} & \texttt{$\Gamma \vdash$ void : Void}\\
 \hline
 \end{tabular}

\end{center}

\subsubsection{Tipi composti}
\begin{enumerate}
 \item \textbf{Record}: collezione di campi, ciascuno di un diverso tipo; un campo è selezionato col suo nome
 \item \textbf{Record varianti}: Record dove solo alcuni campi (mutuamente esclusivi) sono attivi a un dato istante
 \item \textbf{Array}: Funzione da un tipo indice (scalare) ad un altro tipo. Array di caratteri sono detti stringhe.
 \item \textbf{Insieme}: sottoinsieme di un tipo base.
 \item \textbf{Puntatore}: Riferimento ad un oggetto di un altro tipo.
\end{enumerate}

\subsubsection{Record (struct)}
Sono stati introdotti per manipolare in modo unitario dati di altro tipo eterogeneo. Li troviamo in: C, C++, CommonLisp, Ada, Pascal, Algol68, ...\smallskip

I record possono essere annidati, e sono \textbf{memorizzabili}, \textbf{esprimibili} e \textbf{denotabili}.\smallskip
\begin{itemize}
 \item Pascal non ha modo di esprimere “un valore record costante”
 \item C lo può fare, ma solo nell’inizializzazione (initializer)
\item Uguaglianza generalmente non definita (contra: Ada)

\end{itemize}


Esempi:
\begin{multicols}{2}
 
\begin{lstlisting}[language = c]
// C
struct studente {
    char nome[20];
    int matricola; };
    
studente s;
s.matricola = 343536 ;
\end{lstlisting}

\begin{lstlisting}[language = c]
// js (oggetto)
const CartoonCharacter =
    {Name:"Goofy",
    Created:1932,
    FirstAppeared:"Mickey's Revue"
    };
\end{lstlisting}

\end{multicols}
\subsubsection{Record: implementazione}
Memorizzazione sequenziale dei campi. Problema esempio:
\begin{multicols}{2}
 
\begin{lstlisting}[language=c]
 struct MixedData
{
    char Data1;     // char: 1 byte
    short Data2;    // short: 2 byte
    int Data3;      // int: 4 byte
    char Data4;     // char: 1 byte
};
\end{lstlisting}

Se ho parole di 32 bit, e memorizzo ogni campo in base alla quantità di memoria che occupa, ho:
\begin{center}
 
\begin{tikzpicture}
 \node[rectangle, draw]() at (0,0){Data1};
 \node[rectangle, draw]() at (1.5,0){Data2};
 \node[rectangle, draw]() at (3,0){Data2};
 \node[rectangle, draw, fill = lightgray]() at (4.5,0){Data3};
 \node[rectangle, draw, fill = lightgray]() at (0,-.75){Data3};
 \node[rectangle, draw, fill = lightgray]() at (1.5,-.75){Data3};
 \node[rectangle, draw, fill = lightgray]() at (3,-.75){Data3};
 \node[rectangle, draw]() at (4.5,-.75){Data4};

\end{tikzpicture}

\end{center}


\end{multicols}

\smallskip

Data3 si trova a cavallo tra due parole!! Possibili approcci:
\begin{enumerate}
 \item Allineamento dei campi alla parola (32/64 bit): occupo un'intera parola per ogni campo
 \begin{itemize}
  \item Spreco di memoria (un record contiene tipi di dato diversi, che non occupano tutti lo stesso spazio in memoria)
  \item Accesso semplice
 \end{itemize}
  \item Padding o packed record (\emph{cerco} di mettere più valori nella stessa parola)
 \begin{itemize}
  \item Disallineamento
  \item Accesso più costoso
 \end{itemize}
 \begin{lstlisting}[language=c]
struct MixedData {
    char Data1; /* 1 byte */
    char Padding1[1]; /* 1 byte for the following 'short'
                        to be aligned on a 2 byte boundary*/
    short Data2; /* 2 bytes */
    int Data3; /* 4 bytes - largest structure member */
    char Data4; /* 1 byte */
    char Padding2[3]; /* 3 bytes to make total size of
                        the structure 12 bytes */
};
\end{lstlisting}
\end{enumerate}


Piccola ottimizzazione (field-reordering):
\begin{lstlisting}[language = c]
struct MixedData {
    char Data1;
    char Data4;
    short Data2;
    int Data3;
};
\end{lstlisting}

Tra le miriadi di opzioni del compilatore si può scegliere che modello utilizzare.
\newpage
\subsubsection{Record: modello}
\renewcommand{\arraystretch}{1.5}
\begin{tabular}{|l|l|l|}
\hline
 \textbf{Sintassi espressioni} & \textbf{Valori} & \textbf{Tipi} \\
 \texttt{Exp ::=  ... } & \texttt{v::= ... } & \texttt{$\tau$ ::= ...}\\
 \texttt{| [l$_1$:e$_1$, ... l$_n$:e$_n$]} & \texttt{| [l$_1$:v$_1$, ... l$_n$:v$_n$]} & \texttt{| [l$_1$:$\tau_1$, ... ln:$\tau_n$]}\\
 \texttt{| e.l} & &\\
 \hline
 \end{tabular}
 \bigskip
 
\textbf{Semantica statica}: 
 \[\frac{\forall i : \Gamma \vdash e_i : \tau_i}{[l_1 : e_1,\hdots, l_k:e_k] : [l_1 :\tau_1,\hdots, l_k : \tau_k]}  \hspace{1cm}
 \frac{\Gamma \vdash e : [l_1 :\tau_1,\hdots, l_k : \tau_k], 1 \leq j \leq k}{\Gamma \vdash e.l_j : \tau_j}
 \]

\textbf{Semantica dinamica}: 
  \[\frac{1 \leq j \leq k}{[l_1 : v_1,\hdots, l_k:v_k].l_j \to v_j}  \hspace{1cm}
 \frac{e \to e'}{e.l \to e'.l}
 \]
 \[\frac{e_j \to e_j'}{[l_1:e_1,\hdots,  l_j:e_j, \hdots, l_k:e_k] \to [l_1:e_1,\hdots,  l_j:e_j', \hdots, l_k:e_k]}\]
 
 \subsubsection{Record: simulazione OCaml}
 \begin{lstlisting}
type label = Lab of string
type expr = ...
    | Record of (label * expr) list
    | Select of expr * label
    
Record [(Lab "size", Int 7); (Lab "weight", Int 255)]
 \end{lstlisting}

 Funzioni di valutazione:
 
 \begin{lstlisting}
    let rec lookupRecord body (Lab l) =
    match body with
        | [] -> raise FieldNotFound
        | (Lab l', v)::t ->
            if l = l' then v else lookupRecord t (Lab l)
 \end{lstlisting}

 Interprete:
 \begin{lstlisting}
  let rec eval e = match e with
    ...
    | Record(body) -> Record(evalRecord body)
    | Select(e, l) -> match eval e with
        |Record(body) -> lookupRecord body l
        | _ -> raise TypeMismatch

    and evalRecord body = match body with
        |[] -> []
        |(Lab l, e)::t -> (Lab l, eval e)::evalRecord t
 \end{lstlisting}

 \subsubsection{Array}
Un \textbf{array} è una \emph{collezione di dati omogenei}; si può vedere come una funzione dal tipo dell'indice (usualmente discreto, e.g. int) al tipo degli elementi. L'implementazione standard consiste nella memorizzazione degli elementi in \textbf{locazioni contigue}. \smallskip

Ovviamente la maggior parte di ciò è falso per gli array di js, che sono implementati in modo non-standard, possono essere non omogenei, sono dinamici e restituiscono ``undefined'' se si va out-of-bound.\smallskip

Il controllo sui bound è solitamente lasciato all'utente (contra: Java, Python), anche se Tony Hoare dice (e c'ha ragione) che è meglio fare un vero controllo a run-time, se no ci sono un sacco di vulnerabilità (code reuse, injection).
\end{document}
