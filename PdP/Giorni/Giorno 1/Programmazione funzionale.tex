\documentclass[a4paper,10pt]{article}
\usepackage[utf8]{inputenc}
\usepackage[margin=25mm]{geometry}
\usepackage[italian]{babel}
\usepackage{amsmath}
\usepackage{amsthm}
\usepackage{amsfonts}
\usepackage{centernot}
\usepackage{multicol}
\usepackage{tikz}
\usepackage{listings}
\usepackage{courier}
\usetikzlibrary{patterns.meta}
\setlength{\parindent}{0em}
\newcommand{\reals}{\mathbb{R}}
\newcommand{\integers}{\mathbb{Z}}
\newcommand{\naturals}{\mathbb{N}}
\newcommand{\cnot}{\centernot}


\definecolor{backcolour}{rgb}{0.95,0.95,0.92}
\lstdefinestyle{mystyle}{
    language=caml,
    backgroundcolor=\color{backcolour},
    numberstyle=\tiny,
    basicstyle=\ttfamily\footnotesize,
    keywordstyle=\bfseries,
    breakatwhitespace=false,         
    breaklines=true,                 
    captionpos=b,                    
    keepspaces=true,                 
    numbers=left,                    
    numbersep=5pt,                  
    showspaces=false,                
    showstringspaces=false,
    showtabs=false,                  
    tabsize=2
}

\lstset{style=mystyle}

\begin{document}

\begin{center}
    \LARGE Giorno 1
    
    \Large Programmazione Funzionale
\end{center}\smallskip

\begin{abstract}

 La programmazione funzionale è basata su \emph{computazioni senza stato}, che procedono invece mediante \emph{riscritture} di espressioni. Senza stato non c'è iterazione, quindi l'unico costrutto per il controllo di sequenza è la ricorsione. Sono presentati gli aspetti generali della programmazione funzionale, il $\lambda$-calcolo e degli esempi di estensioni tipate di questo ($\lambda$-calcolo tipato semplice, PCF)
\end{abstract}

\section{Aspetti generali}
\subsection{Dichiarazione e applicazione}
\begin{enumerate}
 \item Sia $f(x) = x^2$ la funzione che associa ad $x$ il suo quadrato; 
 \item Allora $f(2) = 4$.

\end{enumerate}

In (1) l'espressione sintattica $f(x)$ è usata per introdurre il nome della funzione \emph{anonima} $x \mapsto x^2$, mentre in (2) indica il risultato dell'\emph{applicazione} di $f$ ad uno specifico valore.\smallskip

Per distinguere tra nome e corpo della funzione, scriviamo (in OCaml):

\begin{itemize}
 \item \textbf{Dichiarazione di funzione:}
\begin{lstlisting}
 let f = (fun x -> x * x) in
\end{lstlisting}

Dove \texttt{fun x -> x * x} si dice \emph{astrazione}.
\lstset{firstnumber=last}
\item \textbf{Applicazione di funzione}
\begin{lstlisting}
 f(2);;   (* Stampa 4 *)
\end{lstlisting}
\end{itemize}

Si noti che \underline{le parentesi non sono necessarie}, avrei potuto infatti scrivere semplicemente \texttt{f 2}.

\subsection{Funzioni di ordine superiore, currying}

Le funzioni possono prendere come argomenti, o restituire, altre funzioni. Ad esempio si possono dichiarare funzioni ``di più parametri'' (currying) scrivendo:

\begin{lstlisting}
 let f = (fun x -> fun y -> x*y);;
\end{lstlisting}

Ossia la funzione $f(x, y) = xy$, scritta come: \[f : x \mapsto (y \mapsto (x \cdot y)) \] \texttt{f} si può applicare scrivendo, ad esempio:

\begin{itemize}
 \item ``\texttt{f 5 (6)}'' (si ricordi che \texttt{f 5} è la funzione $y \mapsto 5y$)
 \begin{multicols}{3}
 \item ``\texttt{(f 5) 6}''
 \item ``\texttt{(f 5) (6)}''
 \item ``\texttt{f 5 6}''
 \end{multicols}
 \end{itemize}
\subsection{Computazione come riduzione}
\label{com_com_rid}
Un linguaggio funzionale è costituito esclusivamente da espressioni.\smallskip

La valutazione di un'espressione consiste in una riscrittura (\emph{= riduzione}) di questa, che avviene sostituendo testualmente una sottoespressione del tipo ``funzione applicata ad un argomento'' con il corpo della funzione in cui si sostituiscono le occorrenze del parametro formale con il parametro attuale, e.g.

\begin{multicols}{2}

\begin{center}
Sia $f$ = \texttt{(fun x -> x * x)}
\end{center}


\begin{center}
 \texttt{f 5} \quad $\to$ \quad \texttt{(x * x) (5)} \quad $\to$ \quad \texttt{5 * 5}
\end{center}
\end{multicols}

\subsubsection{Redex}

Un Redex è una \emph{espressione riducibile}, ossia un'applicazione della forma \texttt{(fun x -> corpo) arg}
\begin{itemize}
 \item Il \emph{ridotto} di un redex è l'espressione che si ottiene sostituendo i parametri \texttt{x} del corpo con \texttt{arg}.
 \item $\beta$-regola: Se in un'espressione \texttt{exp1} compare come sottoespressione un redex, \texttt{exp1} si riduce nell'espressione \texttt{exp2} in cui il redex è sostituito con il suo ridotto.
\end{itemize}
\newpage
\subsection{Valutazione}
\begin{itemize}
 \item \textbf{Valori}: Si dice \emph{valore} un'espressione che non deve essere ulteriormente riscritta. Un valore può essere primitivo o funzionale.
 \item \textbf{Capture-avoiding substitution}: Lo stesso nome non è mai dato a variabili distinte (variabili ``fresche'', vd. esempio nella parte sul $\lambda$-calcolo)
 \item \textbf{Strategie di valutazione}: 
 \begin{itemize}
  \item[$\circ$] \textbf{Valutazione eager} -- \emph{call by value}, un redex è valutato solo se la sua parte argomento è un valore: si valuta quindi l'espressione del parametro attuale\emph{prima di essere associata al parametro formale}.
  \item[$\circ$] \textbf{Valutazione normale} -- \emph{call by name}, in ``ordine normale''; Un redex è valutato prima della sua parte argomento: l'espressione del parametro attuale \emph{non viene valutata
prima di essere associata al parametro formale}.
 \end{itemize}

\end{itemize}

\section{$\lambda$-calcolo}

\subsection{Sintassi}
Un programma è un'espressione:
\[ \begin{aligned}
        e::= & \,x \\&| \lambda x.e \\&| e e
        \end{aligned}\quad\quad
    \begin{aligned}
     &\text{Variabile}\\ &\text{Astrazione (fun dec)}\\ &\text{Applicazione (fun call)} 
    \end{aligned}
 \]

 Regole d'inferenza: 
 \begin{multicols}{3}
 \centering
 \it Variabile:\bigskip
 
 $ \dfrac{x \in Var}{x \in Exp} $
 
 Astrazione:\bigskip
 
 $ \dfrac{x \in Var \quad e \in exp}{\lambda x.e \in Exp} $
 
 Applicazione:\bigskip
 
 $\dfrac{e_1 \in Exp \quad e_2 \in Exp}{e_1e_2 \in Exp}$
 \end{multicols}

Convenzioni sintattiche:
\begin{multicols}{2}
 
\begin{itemize}
 \item Lo scope del $\lambda$ si estende il più a destra possibile: 
 \[\lambda x .\lambda y.xy = \lambda x.(\lambda y.(xy))\]
 \item L'applicazione associa a sinistra: \[e_1e_2e_3 = (e_1e_2)e_3\]
\end{itemize}
\end{multicols}

Aggiungiamo le \textbf{operazioni aritmetiche} e le \textbf{costanti numeriche} al lambda calcolo per semplificare.
 
\subsection{Variabili libere e legate}
Il $\lambda$ in $\lambda x.e$ agisce da operatore di \emph{binding}, cioè lega la variabile $x$ nell'espressione $e$. La variabile $x$ può essere sostituita da un'altra variabile \emph{fresca} (dove fresca = nuova, diversa in nome da tutte le altre che stiamo trattando) ed il programma rimane lo stesso. \bigskip

\textbf{Definizione:} Una variabile introdotta (dichiarata) da un $\lambda$ si dice \textbf{legata} da quel $\lambda$.\bigskip

\textbf{Definizione:} Una variabile che non è associata a nessun $\lambda$ si dice \textbf{libera}.\bigskip

\textbf{Definizione:} Si dicono \textbf{$\alpha$-equivalenti} due espressioni uguali a meno della sostituzione di una variabile legata con una variabile fresca., e.g.  $\lambda a.(a+1) \equiv_\alpha \lambda b . (b+1)$.\bigskip

\subsubsection{Definizione formale}

L'insieme $FV$ delle variabili libere è definito da:

\[\begin{aligned}
   &FV(x) = \{x\}\\
&FV(e_1 e_2) = FV(e_1) \cup FV(e_2)\\
&FV(\lambda x.e) = FV(e) \setminus \{x\}\\
  \end{aligned}
\]

\subsection{Alberi di sintassi astratta}
 Un albero di sintassi astratta (AST) rappresenta un programma.
 \begin{itemize}
  \item Le foglie sono variabili
  \item Un'astrazione $\lambda x.e$ è un nodo etichettato $\lambda x$ che ha come sottoalbero l'AST di $e$.
  \item Un'applicazione $e_1 e_2$ consiste di un nodo etichettato $@$ con come sottoalberi sinistro e destro gli AST di $e_1$ ed $e_2$
 \end{itemize}

 \subsection{Esecuzione}
 L'esecuzione avviene come descritto in \ref{com_com_rid}; quando si valuta una $\lambda$-espressione $ (\lambda x.e_1)e_2 $ si sostituisce ogni occorrenza di $x$ in $e_1$ con $e_2$ (valutata o meno, dipende se call-by-name o call-by-value).\bigskip
 
 \textbf{Notazione}: La notazione $e_1 \{ x:=e_2 \}$ descrive la $\lambda$-espressione $e_1$ in cui ogni occorrenza della variabile $x$ è sostituita da $e_2$.\bigskip
 
 \textbf{$\beta$-riduzione}: È la regola che cattura precisamente la nozione di applicazione funzionale ($\sim \beta$-regola):
 \[ (\lambda x.e_1)e_2 \to  e_1 \{ x:=e_2 \}\]
 \subsubsection{Capture Avoiding Substitution}
 
 
 \textbf{Problema}: Se $e_2$ contiene una variabile libera $y$ che è legata in $e_1$, la variabile $y$ potrebbe essere catturata, e l'espressione potrebbe cambiare di significato. E.g.
 

 \[(\lambda x . (\lambda y.(x+y))) \underbrace{y}_\text{libera} \to (\lambda y.(x+y))\{ x := y \} \to\underbrace{(\lambda y.(y+y))}_\text{$y$ catturata} \]
 
\textbf{Soluzione}: $e_1$ deve essere $\alpha$-convertita:
 
\[(\lambda y.(x+y))\{ x := y \} \underset{\alpha\text{-conv, $z$ fresh}}\to (\lambda z.(x+z))\{ x := y \} \to \underbrace{(\lambda z.(y+z))}_\text{$y$ libera} \]

\paragraph{Definizione formale della Capture Avoiding Substitution}
\[\begin{aligned}
   x\{x := e\} \quad &\equiv\quad  e\\
   y\{x := e\} \quad &\equiv\quad  y, \text{ se } y \neq x\\
   (e_1 e_2)\{x := e\}\quad &\equiv\quad (e_1\{x:=e\})(e_2\{x:=e\})\\
   (\lambda y.e_1)\{ x:=e \}\quad &\equiv\quad  \begin{cases}
                                                    \lambda y.(e_1\{ x:=e \}) &\text{se $y \neq x$ e $y \notin FV(e)$}\\
                                                    \lambda z.((e_1\{ x:=z \})\{ x:=e \}) &\text{se $y \neq x$ e $y \in FV(e)$, z fresca}
                                               \end{cases}
  \end{aligned}
\]

 \newpage
\subsection{Interprete del $\lambda$-calcolo}
\subsubsection{$\beta$ riduzione, formalmente}

La $\beta$-riduzione ($\to$) è definita da:
 \[ (\lambda x.e_1)e_2 \to  e_1 \{ x:=e_2 \}\]
 
  \[ \dfrac{e_1 \to e'}{e_1e_2 \to e'e_2} \quad\quad\quad \dfrac{e_2 \to e'}{e_1e_2 \to e_1'} \quad\quad\quad \dfrac{e \to e'}{\lambda x.e \to \lambda x .e'}\]
\textbf{Definizioni e notazioni}

\begin{itemize}
 \item Indichiamo con $\implies$ la chiusura riflessiva e transitiva di $\to$
 \item Una $\lambda$-espressione $e_1$ è \emph{$\beta$-riducibile} alla $\lambda$-espressione $e_2$ se $e_1 \implies e_2$
 \item Due $\lambda$-espressioni $e_1$, $e_2$ si dicono $\beta$\textbf{-equivalenti} se:
 \begin{itemize}
  \item Sono $\alpha$-equivalenti;
  \item Una delle due è $\beta$-riducibile all'altra ($e_1 \implies e_2 \vee e_2 \implies e_1$);
  \item Sono entrambe $\beta$-riducibili alla stessa $\lambda$-espressione ($e_1 \implies e \wedge e_2 \implies e$).
 \end{itemize}
In altre parole, due $\lambda$-espressioni sono $\beta$-equivalenti quando sono indistinguibili dal punto di vista computazionale: calcolano gli stessi risultati.
\item Una $\lambda$-espressione si dice in \textbf{forma normale beta} se non è ulteriormente riducibile.
\end{itemize}

\textbf{Teorema di Church Rosser:} L'ordine in cui vengono scelte le $\beta$-riduzioni non influisce sul risultato finale.

\subsubsection{Non terminazione}
La $\lambda$-espressione $\Omega = (\lambda x . xx)(\lambda x . xx)$ non può essere ridotta in forma normale, poiché la riduzione produce nuovamente $\Omega$.

\[  (\lambda x . xx)(\lambda x . xx) \to (xx)\{ x := (\lambda x . xx) \} \to (\lambda x . xx)(\lambda x . xx) \]

\subsubsection{Ricorsione: il combinatore Y}
La ricorsione è implementata tramite la funzione:
\[ Y = \lambda f.(\lambda x.f(xx))(\lambda x.f(xx)) \]
che gode della proprietà:
\[ YF \equiv_\beta F(YF) \]

Infatti:

\[ \begin{aligned}
    \lambda f.(\lambda x.f(xx))(\lambda x.f(xx)) F \implies &(\lambda x.F(xx))(\lambda x.F(xx))\\
    \implies &F((\lambda x.F(xx))(\lambda x.F(xx)))\\
    \equiv_\beta\, &F(YF)
   \end{aligned}
 \]
 \newpage
 \subsubsection{Costrutti nel $\lambda$-calcolo}
 \begin{multicols}{2}
  
 \begin{itemize}
  \item Booleani:
  \begin{itemize}
   \item True: $\lambda t.\lambda f.t$
   
   (tra $t$ e $f$ seleziona il primo)
   \item False: $\lambda t.\lambda f.f$
   
   (tra $t$ e $f$ seleziona il secondo)
  \end{itemize}
   
 \item Condizionale:
 \[ IF = \lambda \texttt{c}.\lambda \texttt{then}.\lambda \texttt{else}. \texttt{c then else} \]
 
 \begin{itemize}
 \item Se \texttt{c} è true: 
 
 $(\lambda t.\lambda f.t) \texttt{ then else} \implies \texttt{then}$
 
 \item Se \texttt c è false: 
 
 $(\lambda t.\lambda f.f) \texttt{ then else} \implies \texttt{else}$
\end{itemize}

 
 \end{itemize}
 \end{multicols}

 \textit{Esempio di Condizionale:}
 
 \[\begin{aligned}
&(\lambda \texttt{c} .\lambda \texttt{then} .\lambda \texttt{else}. \texttt{c then else}) \,\texttt{true}\, A B\\
\implies &(\lambda \texttt{then}.\lambda \texttt{else}.(\lambda t.\lambda f.t) \texttt{then else}) A B\\
\textit{(sosituisco A e B)} \implies &(\lambda t .\lambda f.t)A B \implies (\lambda f.A)B \implies A
\end{aligned}\]

 
 \begin{itemize}
  \item Numerali di Church:
  \begin{itemize}
   \item Si definisce $C_0 = \lambda z. \lambda s . z$\quad\quad (dove $z \sim$ zero ed $s \sim$ $succ$)
   \item $C_1 = \lambda z.\lambda s.sz$ 
   \item $C_n = \lambda z.\lambda s . s(s(\hdots (sz)))$
   \item Funzione \textbf{plus} $= \lambda m.\lambda n.\lambda z.\lambda s.m(n z s)s$\smallskip
   
   \textit{Idea:} Facendo $plus (a) (b)$ voglio sostituire il corpo di $a$ alla $z$ nel corpo di $b$:
   
   \[ plus (\lambda z .\lambda s. ssz) (\lambda z .\lambda s. sz) \implies (\lambda z .\lambda s. sssz)\]
  
  La funzione $plus$ fa proprio questo: \hfill $A = \lambda z.\lambda s. ssz \quad B = \lambda z.\lambda s. sz$ \hfill \,
  
  \[\begin{aligned}
     &(\lambda m.\lambda n.\lambda z.\lambda s.m(n z s)s) A B\\
     \implies &\lambda z.\lambda s.B(\underline{A z s})s\\
     \implies &\lambda z.\lambda s.B(ssz)s\\
     =\,\, &\lambda z.\lambda s.\underline{(\lambda z .\lambda s. sz)(ssz)}s\\
     \textit(idea) \implies &\lambda z.\lambda s. \underline{(\lambda s.sssz)s}\\
     \implies &\lambda z. \lambda s. sssz
    \end{aligned}
\]
  \item Funzione \textbf{times} $= \lambda m. \lambda n. m \,\,C_0 (Plus\,\, n)$
  
  \textit{Idea}: Sommare $n$ a zero $m$ volte.
  \end{itemize}
  
 \end{itemize}

 \newpage

 \subsubsection{Scoping statico}
 Il lambda calcolo adotta un meccanismo di scoping statico per definire la visibilità delle variabili. \smallskip
 
 Ad esempio: in  $ (\lambda x.x (\lambda x.x))z$ la $x$ più a destra della variabile si riferisce alla $x$ introdotta nel secondo $\lambda$.
 
 Tramite l'$\alpha$-conversione si ottiene una versione equivalente:
 
 \[ \lambda x.x (\lambda x.x))z \equiv_\alpha \lambda x.x (\lambda y.y))z \]
 
 \subsubsection{Dichiarazioni locali}
 Le dichiarazioni di variabili locali sono codificate tramite $\lambda$-astrazione e applicazione:
 \[\texttt{let $x$ = $e_1$ in $e_2$} \,\,\approx \,\, (\lambda x.e_2) e_1\]
 
 \subsubsection{Strutture dati}
 Si possono rappresentare anche strutture dati, e.g. una coppia $(a, b)$ si può rappresentare come:
 \[ \lambda x. IF\,\, x\,\, a\,\, b\,\, \]
 E posso accedere $a$ e $b$ utilizzando le due funzioni:
 
 \[ Fst = \lambda f.f True \hspace{2cm} Snd = \lambda f.f False \]
 
 
 \subsubsection{Call-by-Value vs Call-by-Name}
\begin{itemize}
 \item Riduzione call-by-value:
 \[(\lambda x.e_1)v \to e_1\{x := v\}\]
 \[ \dfrac{e_1 \to e'}{e_1e_2 \to e'e_2} \hspace{2cm} \dfrac{e_2 \to e'}{ve_2 \to ve'} \]
 
 \textit{Idea:} partendo da sinistra, seleziono la prima applicazione $e_1e_2$ e riduco l'espressione $e_1$ finché non è ridotta ad un valore (funzionale), poi valuto la parte argomento $e_2$ e infine applico la funzione.
 
 \item Riduzione call-by-name
 
 \[(\lambda x.e_1 )e_2 \to e_1\{x:=e_2\} \hspace {1.5cm} \dfrac{e_1 \to e'}{e_1e_2 \to e'e_2}\] 
 
  \textit{Idea:} Partendo da sinistra, seleziono l'applicazione $e_1e_2$, riduco $e_1$ ad un valore funzionale ed applico la funzione prima di calcolare il valore dell'argomento $\implies$ \textit{applico la funzione appena possibile}.
 \item In alcuni casi la call-by-value e la call-by-name possono comportarsi in modo diverso, ad esempio la valutazione dell'espressione:
 \[\texttt{IF True True}\,\,(\Omega \Omega)\]
 
 Termina e restituisce \texttt{True} se call-by-name (non cerca di valutare $\Omega \Omega$), mentre non termina mai se call-by-value.
\end{itemize}
\newpage

\section{Controllo dei tipi}
Nel $\lambda$-calcolo è possibile scrivere programmi che non sono corretti rispetto all'uso inteso dei valori:
\[ False \,\,0 = (\lambda t.\lambda f.f)(\lambda z.\lambda s.\lambda z) \to \lambda f.f\]

Questo programma produce un valore, ma non ha senso. Dobbiamo assegnare dei \emph{tipi} ai dati, ossia degli attributi che descrivono come il linguaggio permette di usare quel particolare dato.

\subsection{Sistema dei tipi}
Un sistema di tipi è un metodo sintattico, effettivo per dimostrare l'assenza di comportamenti anomali del programma strutturando le operazioni del programma in base ai tipi di valori che calcolano.
\begin{itemize}
 \item \textbf{Effettivo}: si può definire un algoritmo che calcola un'approssimazione statica dei comportamenti a runtime del programma
 \item \textbf{Strutturale}: I tipi delle componenti di un programma sono calcolati in modo composizionale, i.e. il tipo di un'espressione dipende solo dai tipi delle sue sottoespressioni
\end{itemize}

\subsubsection{Esempi}
[Divisione per zero, js $\to$ ts]
\subsubsection{Controllo dei tipi}
Il \emph{type checker} verifica che le intenzioni del programmatore (espresse dalle annotazioni di tipo) siano rispettate dal programma.\smallskip

Il controllo dei tipi può essere statico o dinamico. Se è statico trova gli errori prima dell'esecuzione del programma, e non degrada le prestazioni.

\subsection{Case study: sistema di tipo per espressioni}

Sintassi:
 \begin{center}
  \tt E ::= V | if E then E else E
  
  V ::= true | false | NV | isZero E
  
  NV ::= 0 | succ NV | pred NV
 \end{center}


\subsubsection{Esempi di regole di esecuzione:}
\begin{itemize}


\tt 
 \item IF-TRUE
 
 \begin{center}
  if \textbf{true} then $E_1$ else $E_2$ $\to$ $E_1$
 \end{center}

 \item IF-FALSE 
 
 \begin{center}if \textbf{false} then $E_1$ else $E_2$ $\to$ $E_2$ 
 \end{center}

 \item IF-COND \[\dfrac{E \to E'}{\texttt{if $E$ then $E1$ else $E_2$ $\to $ if $E'$ then $E1$ else $E_2$}}\]
 
\item \textrm{Regole della forma:}
\[\dfrac{E \to E'}{pred\,\,E \to pred\,\,E'} \hspace{1cm}\dfrac{E \to E'}{succ,\,E \to succ\,\,E'} \hspace{1cm }\textrm{... (isZero)}\]
\item $pred(succ\,\, NV) \to NV$ \textrm{e viceversa}
\item $pred\,\, 0 \to 0$, $isZero\,\, 0 \to true$, $isZero( succ\,\,NV) \to false$
\end{itemize}
\newpage
\subsubsection{Tipi per espressioni}

Gli unici due tipi di questo linguaggio sono \texttt{Bool} e \texttt{Nat}

Le regole di controllo dei tipi sono molto intuitive:
\begin{itemize}
 \item $true : Bool$\quad\quad $false : Bool$ \quad\quad$0 : Nat$
 \item Regole della forma:
 \[ \dfrac{E:Nat}{succ\,\,E : Nat} \hspace{2cm} \dfrac{E:Nat}{pred,\,E : Nat} \hspace{2cm} \dfrac{E:Nat}{isZero,\,E : Bool}\]
 \item $\dfrac{E : Bool\quad E_1 : T\quad E_2:T}{if\,\, E\,\, then\,\, E_1\,\, else\,\, E_2 : T}$, che non permette ad esempio di associare un tipo all'espressione
 \begin{center}
  \tt if true then 0 else false
 \end{center}
 
 ...anche se questa assume sempre valore numerico. In questo modo si garantisce però la composizionalità (tipo delle espressioni dipende solo da tipo delle sottoespressioni)\smallskip
 
 Ogni coppia espressione-tipo $(E, T)$ della relazione di tipo è caratterizzata da un albero di derivazione costruito da istanze delle regole di inferenza.
 \end{itemize}
 \subsubsection{Type Safety: Progresso e Conservazione}
 
 La correttezza (type safety) di un sistema di tipo è espressa formalmente dalle proprietà di \emph{progresso e conservazione}.
 
 \begin{itemize}
  \item \textbf{Progresso}: se $E:T$ allora $E$ è un valore oppure esiste $E'$ tale che $E \to E'$ 
  
  (ossia: una espressione ben tipata non si blocca a run-time)
  \item \textbf{Conservazione}: Se $E:T$ e $E \to E'$ allora $E' : T$ 
  
  (ossia: i tipi sono preservati dalle regole di esecuzione)
 \end{itemize}

    \subsubsection{Lemmi di inversione}
    I lemmi di inversione sono ``le regole di tipo lette al contrario'', e.g.:
    \begin{itemize}
     \item Dalla regola $true:Bool$ si ottiene il lemma di inversione: ``Se $true: R$ allora $R = Bool$''
    \item Dalla regola sui tipi delle espressioni condizionali si ottiene:
    \[\text{Se }(if\,\,E\,\,then\,\,E_1\,\,else\,\,E_2):R \texttt{ allora }E:Bool, \,\, E_1:R, \,\, E_2:R\]
    \end{itemize}
    
    Queste regole possono essere codificate in un algoritmo che restituisce il tipo di un'espressione (e sono utilizzate implicitamente al passo induttivo delle dimostrazioni di progresso e conservazione).
    
    \subsubsection{Forme canoniche}
    Le forme canoniche sono i valori che possono assumere le espressioni di un certo tipo:
    \begin{itemize}
     \item Se $v$ è di tipo \texttt{Bool}, allora $v = true$ oppure $v = false$
     \item Se $v$ è di tipo \texttt{Nat}, allora $v$ è un valore numerico.
    \end{itemize}
\newpage
    \subsubsection{Progresso}
    Se $E:T$ allora $E$ è un valore oppure esiste $E'$ tale che $E \to E'$
    
    \begin{proof}
    Per induzione sulla struttura della derivazione di $E:T$
    
    \begin{itemize}
     \item \textbf{Casi base:} 
     \[true : Bool \hspace{2cm} false: Bool \hspace{2cm} 0:Nat\]
     Immediato, poiché sono valori.
     \item \textbf{Casi induttivi:} (vediamo, come esempio, quello dell'If-Then-Else, gli altri sono analoghi.)
     
     Si consideri la regola:
     \[\dfrac{E : Bool\quad E_1 : T\quad E_2:T}{if\,\, E\,\, then\,\, E_1\,\, else\,\, E_2 : T}\]
     Per ipotesi induttiva vale il progresso per $E$, $E_1$, $E_2$, cioè o sono valori o si può fare un passo.
     
     \begin{itemize}
      \item Se $E$ è un valore allora per le forme canoniche deve valere $True$ o $False$. In questo caso si possono applicare le regole IF-TRUE e IF-FALSE per mostrare che l'espressione fa un passo e diventa $E_1$ o $E_2$, per cui vale la regola, per ipotesi induttiva.
      \item Se $E$ non è un valore si applica la regola IF-COND:
      \[\dfrac{E \to E'}{\texttt{if $E$ then $E1$ else $E_2$ $\to $ if $E'$ then $E1$ else $E_2$}}\]
      
      Quindi si fa un passo.
     \end{itemize}

    \end{itemize}
 
    \end{proof}
    \subsubsection{Conservazione}
    Se $E:T$ e  $E \to E'$ allora $E':T$
    \begin{proof}
     Analizziamo di nuovo solo il caso dell'If-Then-Else, per induzione.
     
      \[\dfrac{E : Bool\quad E_1 : T\quad E_2:T}{if\,\, E\,\, then\,\, E_1\,\, else\,\, E_2 : T}\]
     
     (Caso base: I guess per vacuità vale sui valori(?))\medskip
     
     Supponiamo che la regola valga per $E$, $E_1$, $E_2$.
     
     Possiamo fare un passo utilizzando la regola IF-COND citata prima, e sapendo, per hp induttiva, che la conservazione vale per $E$ il tipo rimarrà $Bool$, mentre i tipi di $E_1$ ed $E_2$ rimarranno gli stessi perché sono invariati.
    \end{proof}
\newpage
     \subsection{$\lambda$-calcolo tipato semplice}
     \subsubsection{Estensione con i booleani}
     Estendiamo il lambda calcolo nel seguente modo:
     \begin{center}
        \tt e ::= x | $\lambda$x:$\tau$.e | e e | true | false | if e then e else e
     \end{center}
     E notiamo che nel parametro formale della lambda astrazione appare un'annotazione di tipo $\tau$. La sintassi dei tipi è:
     
     \begin{center}
      \tt $\tau$ ::= Bool | $\tau\to \tau$
     \end{center}

     \subsubsection{Sintassi simil-OCaml}
     Utilizziamo una nuova sintassi, analoga a quella del lamda-calcolo tipato appena introdotto:
     \begin{center}
        \tt e ::= x | fun x:$\tau$ = e | Apply(e, e) | true | false | if e then e else e
     \end{center}
     
     \subsubsection{Ambiente}
     Le regole di tipatura si complicano rispetto al sistema di tipo per espressioni visto prima, poiché abbiamo delle variabili (la \texttt{x} in \texttt{fun x:$\tau$}); ci serve perciò un \emph{ambiente}.\smallskip
     
     L'ambiente è una funzione $\Gamma$ che associa nomi a tipi. Per indicare $\Gamma(x_i) = \tau_i$ noi utilizzeremo la notazione:
     \[ \Gamma = x_1 : \tau_1,\quad x_2 : \tau_2, \quad\hdots \quad x_k:\tau_k \]
     
     Mentre utilizzeremo la notazione $\Gamma, x:\tau$ per indicare l'\textbf{estensione} di $\Gamma$ con l'associazione $x:\tau$:
     \[(\Gamma, x:\tau)(y) = \begin{cases}
                              \tau &y = x\\
                              \Gamma(y) & y \neq x
                             \end{cases}
     \]
     
     Ovviamente questo vale solo se $y$ è nell'ambiente: in caso contrario $\Gamma(y) = undefined$.\bigskip
     
     Nota: nel gergo dei compilatori l'ambiente è chiamato ``tabella dei simboli''.
     
     \subsubsection{Giudizio di tipo}
     Sia $\Gamma$ un ambiente di tipo, si usa la notazione $\Gamma \vdash e:\tau$ per indicare che $e$ ha tipo $\tau$ nell'ambiente $\Gamma$.
     
     \subsubsection{Regole di tipo}
     
     \[ \Gamma \vdash true : Bool \hspace{2cm}  \Gamma \vdash false : Bool \hspace{2cm} \dfrac{\Gamma(x) = \tau}{\Gamma \vdash x: \tau} \]
     
     \textbf{Funzioni:}\smallskip
     \begin{itemize}
      \item \emph{Definizione:}
     
     \[ \dfrac{(\Gamma, x : \tau_1) \vdash e:\tau _2}{\Gamma \vdash (fun\,\, x:\tau_1 = e): \tau_1 \to \tau_2} \]
     
     In questo modo implemento lo scoping; la dichiarazione del parametro $x$ \textbf{sovrascrive} e annulla le precedenti dichiarazioni per $x$ che sono in $\Gamma$.
     
     \item \emph{Invocazione:}
     \[ \dfrac{\Gamma \vdash e_1 : \tau_1 \to \tau_2 \quad\quad \Gamma \vdash e_2 : \tau_1}{\Gamma \vdash Apply(e_1, e_2) : \tau_2} \]
     \end{itemize}
\newpage

\subsubsection{Type safety}
\begin{itemize}
 \item \textbf{Progresso}: Se $\emptyset \vdash e:\tau$ allora $e$ è un valore oppure esiste $e'$ tale che $e \to e'$
 \item \textbf{Conservazione}: Se $\Gamma \vdash e:\tau$ e $e \to e'$ allora $\Gamma \vdash e' : \tau$
\end{itemize}
Le dimostrazioni sono simili a quelle fatte per le espressioni:
\begin{itemize}
 \item Si deve fare una dimostrazione per ogni regola di derivazione
 \item Le regole che non hanno delle precondizioni sono usate come casi base, mentre quelle che ne hanno come casi induttivi

 \end{itemize} 
 
 \textbf{Progresso:}
 Unico caso meno ovvio: $Apply(e_1, e_2)$,\quad  $\emptyset\vdash e_1 : \tau_1 \to \tau_2$, \quad$\emptyset \vdash e_2 : \tau_1$
 \begin{proof}
  
 Per ipotesi induttiva sappiamo che il progresso vale per $e_1$ ed $e_2$; 
 
 \begin{itemize}
  \item Se le espressioni \textbf{possono fare un passo} allora si possono applicare le regole di riduzione dell'applicazione, quindi vale il progresso.
  \item Se \textbf{sono entrambe valori} allora per il lemma delle forme canoniche del lambda calcolo tipato (analogo a quelli per le espressioni e valido anche per valori funzionali) abbiamo che $e_1$ è della forma $(fun \,\, x:\tau_1 = e'): \tau_1 \to \tau_2$, quindi si può applicare la $\beta$-riduzione.
 \end{itemize}
 \end{proof}

  
  \textbf{Conservazione:} Con gli strumenti che abbiamo non possiamo dimostrare la conservazione della $Apply$, ci serve un nuovo lemma.
  
  \paragraph{Lemma di sostituzione}
   I tipi sono preservati dall'operazione di sostituzione:

   \[ \Gamma,x : \tau_1 \vdash e:\tau \quad \wedge \quad\Gamma \vdash e_1 : \tau_1 \implies \Gamma \vdash e\{ x=e_1 \} :\tau \]
   
   Questo lemma si dimostra per induzione sulla derivazione di $\Gamma, x :\tau_1 \vdash e : \tau$, dimostrazione nelle note.\medskip
   
   Possiamo adesso dimostrare la conservazione:
   
   \begin{proof} 
   La regola di tipo per la apply è:
   \[ \dfrac{\Gamma \vdash e_1 : \tau_1 \to \tau_2 \quad\quad \Gamma \vdash e_2 : \tau_1}{\Gamma \vdash Apply(e_1, e_2) : \tau_2} \]
   
   \begin{itemize}
    \item \textbf{Caso:} $e_1 = (fun \,\, x:\tau_1 = e_3)$, allora fa un passo tramite la $\beta$-riduzione:
    
    \[ Apply((fun\,\, x : \tau_1 = e_3), e_2) \to e_3\{x := e_2\} \]
    
    E per mostrare che vale la conservazione devo dimostrare che $e_3\{ x := e_2 \}$ ha lo stesso tipo di $Apply(e_1, e_2)$, ossia $\tau_2$. La regola di tipo che applico a questo punto è:
    \[ \dfrac{(\Gamma , x : \tau_1) \vdash e_3 : \tau_2}{\Gamma \vdash (fun\,\, x: \tau_1 = e_3) :\tau_1 \to \tau_2} \]
   
   che permette di dimostrare che $e_3 : \tau_2$, il che è diverso da $e_3\{ x := e_2 \} : \tau_2$, ma per il lemma di sostituzione sappiamo che, dato che il tipo di $x$ ed il tipo di $e_2$ sono uguali, vale $e_3\{x := e_2\} : \tau_2$ 
   
   \end{itemize}
   \end{proof}
\newpage
    \subsubsection{Estensione con booleani e numeri}
    \begin{flushleft}
     \tt  
     e ::= x | fun x:$\tau$ = e | Apply(e, e) 
     
     \quad \quad\quad\quad\,\,\,\,\,| true | false | n | e$\oplus$e | if e then e else e
     \end{flushleft}


    \textbf{Regole di tipo} per i naturali:
    
    \[ \Gamma \vdash n : Nat\hspace{2cm} \dfrac{\Gamma \vdash e_1 : \tau_1 \quad\quad \Gamma \vdash e_2 : \tau_2 \quad\quad \oplus : \tau_1 \times \tau_2 \to \tau}{\Gamma \vdash e_1 \oplus e_2 : \tau} \]
   
   \subsubsection{Dichiarazioni locali}
   Aggiungiamo alla sintassi il costrutto \texttt{let $x$ = $e_1$ in $e_2 : \tau$}\smallskip
   
   \textbf{Regole di valutazione:}
   \[ \dfrac{e_1 \to e'}{let\,\, x = e_1 \,\,in\,\, e_2 :\tau_2 \to  let\,\, x = e' \,\,in\,\, e_2 :\tau_2} \]
   
   \[let\,\,x = v \,\, in \,\, e_2 : \tau_2 \to e_2\{ x = v \}: \tau_2\]
   
   Queste regole mi obbligano a valutare completamente $e_1$ prima di sostituirla in $e_2$.
   
   \textbf{Regola di tipo:}
   \[ \dfrac{\Gamma \vdash e_1 : \tau_1 \quad\quad \Gamma, x:\tau_1 \vdash e_2:\tau_2}{let\,\, x = e_1\,\,in\,\, e_2 : \tau_2} \]
   
   \subsubsection{Ricorsione}
   Si usa $fix$, che permette di ottenere il punto fisso di una funzione, come segue:
   
   \[ \dfrac{}{fix(fun\,\, x : \tau = e) \to e[x = fix(fun\,\, x : \tau = e)]} \hspace{2cm} \dfrac{e \to e'}{fix\,\,e\to fix\,\,e'}\]
   
   E adesso è possibile anche tipare funzioni che non terminano (cosa che non si poteva fare, e.g. si provi a tipare un'applicazione del combinatore omega):
   
   \[ \dfrac{\Gamma \vdash e : \tau\to \tau}{\Gamma \vdash fix\,\, e:\tau}\]
   
   Il lambda calcolo tipato semplice + fix è noto in letteratura come PCF (Plotkin 1977).
   
   \subsubsection{Una sintassi più semplice per la ricorsione}
   
   Linguaggi come OCaml usano la sintassi \texttt{let rec}
   
   \begin{lstlisting}
let rec fact x = 
        if x <= 1 then 1 else x * fact (x-1);;   \end{lstlisting}

\end{document}
