\documentclass[a4paper,10pt]{article}
\usepackage[utf8]{inputenc}
\usepackage[margin=25mm]{geometry}
\usepackage[italian]{babel}
\usepackage{amsmath}
\usepackage{amsthm}
\usepackage{amsfonts}
\usepackage{centernot}
\usepackage{multicol}
\usepackage{tikz}
\usepackage{listings}
\usepackage{courier}
\usepackage{MnSymbol}
\usetikzlibrary{patterns.meta}
\setlength{\parindent}{0em}
\newcommand{\reals}{\mathbb{R}}
\newcommand{\integers}{\mathbb{Z}}
\newcommand{\naturals}{\mathbb{N}}
\newcommand{\cnot}{\centernot}


\definecolor{backcolour}{rgb}{0.95,0.95,0.92}
\lstdefinestyle{mystyle}{
    language=caml,
    backgroundcolor=\color{backcolour},
    numberstyle=\tiny,
    basicstyle=\ttfamily\footnotesize,
    keywordstyle=\bfseries,
    breakatwhitespace=false,         
    breaklines=true,                 
    captionpos=b,                    
    keepspaces=true,                 
    numbers=left,                    
    numbersep=5pt,                  
    showspaces=false,                
    showstringspaces=false,
    showtabs=false,                  
    tabsize=2
}

\lstset{style=mystyle}

\begin{document}
 

\begin{center}
    \LARGE Giorno 1
    
    \Large Macchine Astratte, Linguaggi, Interpretazione, Compilazione
\end{center}\smallskip

\begin{abstract}
    Una macchina astratta, associata ad un linguaggio, rappresenta il comportamento di una macchina fisica; Può essere implementata su una macchina fisica per realizzare un interprete del linguaggio associato, eventualmente realizzando macchine intermedie, usate anche dai compilatori per il run-time support. Le fasi della compilazione si dividono in front end e back end: la front end consiste di scanner (che genera i token) e parser (che genera gli AST), mentre la back end genera il codice.
\end{abstract}

\section{Macchine Astratte}
Una macchina astratta è un sistema virtuale che rappresenta il comportamento di una macchina fisica, individuando:
\begin{itemize}
 \item L'insieme delle \textbf{risorse} necessarie per l'esecuzione dei programmi
 \item Un insieme di \textbf{istruzioni} specificatamente progettato per operare con queste risorse.
\end{itemize}

\paragraph{Alcuni esempi di macchinie astratte:}
\begin{itemize}
 \item \textbf{Algol Object Code}: Macchina astratta per Algol60.
 \begin{itemize}
  \item \emph{Risorse}: Stack, Heap, Program Store
  \item \emph{Istruzioni}: Accesso a variabili e array, scope per variabili e procedure, call by name, call by value
 \end{itemize}
 
 \item \textbf{Smalltalk-80}: Linguaggio a oggetti compilato su macchina astratta \emph{stack based}
 \begin{itemize}
  \item \emph{Risorse}: Stack
  \item \emph{Istruzioni}: Bytecode per manipolazione dello stack, bytecode per invocazione di metodi, accesso a variabili di istanza di oggetti...
 \end{itemize}
 \item \textbf{Python}: macchina astratta stack-based con numerose istruzioni di base
 
\end{itemize}

\paragraph{Macchine astratte per linguaggi funzionali:} Sono caratterizzazte da:
\begin{itemize}
 \item Stack, per gestire chiamate di funzioni
 \item Environment, per gestire associazione tra variabili e valori
 \item Closure, per rappresentare il valore di una funzione
 \item Heap Garbage collector, per la gestione della memoria dinamica.
\end{itemize}

\subsection{Stack machine e bytecode}

\subsubsection{Stack machine}
Una stack machine è una macchina che è capace, ad esempio, di valutare un'espressione in notazione polacca inversa, utilizzando come uniche operazioni di accesso a memoria \texttt{push} e \texttt{pop}.

\subsubsection{Bytecode}
Un bytecode è un linguaggio intermedio tra linguaggio macchina e linguaggio di programmazione, usato per descrivere le operazioni che costituiscono un programma.


\newpage\subsection{Struttura di una macchina astratta}
Una macchina astratta è costituita da:
\begin{itemize}
 \item Un \textbf{interprete}, ossia un programma che prende in ingresso il programma da eseguire (il suo \emph{albero di sintassi astratta}) e lo esegue ispezionando la sua struttura.
 \item Una memoria (dati e programmi)
 \item Controllo
 \item Operazioni primitive
\end{itemize}

\subsubsection{Linguaggio macchina di una macchina astratta}
Data una macchina astratta $M$, il linguaggio $L_M$ è il linguaggio che ha come stringhe legali tutti i programmi interpretabili dall'interprete di $M$.

\subsubsection{Implementazione di macchine astratte}
L'implementazione di una macchina astratta $M$ consiste nella realizzazione di $M$ all'interno di un'altra macchina detta \textbf{macchina ospite} $M_O$, il cui linguaggio è utilizzato per descrivere le componenti di $M$.

\subsubsection{Interpretazione vs Compilazione}
Un linguaggio \emph{interpretato} viene eseguito sulla relativa macchina astratta $M$, che lavora sulla macchina ospite.\smallskip

Un linguaggio \emph{compilato} viene tradotto direttamente nel linguaggio macchina della macchina ospite, sulla quale verranno eseguiti direttamente, e quindi \textbf{non è necessario realizzare la macchina astratta}.

\subsubsection{Macchine intermedie}
Talvolta si realizza una \emph{macchina intermedia}, che aggiunge un \emph{supporto a tempo di esecuzione} ad $M_O$, ossia una collezione di strutture dati e sottoprogrammi che permette l'esecuzione del codice prodotto dal compilatore (e.g. gestione stack, gestione memoria (malloc), codice di debugging)

\subsubsection{Ricapitolando}
Ci sono tre famiglie di implementazioni:
\begin{itemize}
 \item \textbf{Interprete puro}:
 \begin{itemize}
  \item La macchina astratta corrisponde alla macchina intermedia
  \item L'interprete del linguaggio di $M$ è realizzato su $M_0$
  \item Implementazioni vecchie di alcuni linguaggi logici e funzionali, e.g. LISP, PROLOG
 \end{itemize}
 \item \textbf{Compilatore}
\begin{itemize}
 \item Macchina intermedia per estendere la macchina ospite con run-time support
\end{itemize}
\item implementazione mista
\begin{itemize}
 \item I programmi sono tradotti dal linguaggio di $M$ a quello della macchina intermedia (bytecode)
 \item I programmi in bytecode sono interpretati dalla macchina ospite (Java)
\end{itemize}
\end{itemize}
\newpage
\section{Struttura di un compilatore}
\subsection{Fasi della compilazione}
La compilazione si può dividere in due ``macrofasi'': la fase \textbf{front end}, che consiste nell'analisi del programma sorgente e della determinazione della sua struttura sintattica e semantica, e la fase \textbf{back end}, che consiste nella generazione del linguaggio macchina.\smallskip

Le vere e proprie fasi della compilazione sono:
\begin{itemize}
 \item \textbf{Scanner} (analisi lessicale) Genera dei \emph{token} a partire dal programma sorgente. I token consistono in: 
 \begin{itemize}
  \item Parole chiavi (if, while...)
  \item operatori, punteggiatura, parentesi...
  \item identificatori e costanti
 \end{itemize}
Esempio:
\begin{center}
 \texttt{if(x >= y) y = 42;} 
 
 $\downmapsto$ 
 
 $[IF][LPAR][ID(x)][GEQ][ID(y)][RPAR][ID(y)][BECOMES][INT(42)][SCOLON]$
\end{center}

 \item \textbf{Parser} (analisi sintattica) Legge i token, genera albero di sintassi astratta [ok]
 
 \paragraph{Ambiguità}
Se non si definisce la precedenza degli operatori, si possono avere più alberi di sintassi astratta diversi per lo stesso programma.
\item \textbf{Generazione del codice} (back end).
\end{itemize}

\subsection{Parser a discesa ricorsiva}
Spiegazione dettagliata in Giorno 2.

\subsection{AST in Java}
[Struttura albero con programmazione ad oggetti, sulle slide esempio di rappresentazione di albero in javascript, ok]
\subsection{Analisi statica}
[Taint analysis, vedi slide]
\end{document}
