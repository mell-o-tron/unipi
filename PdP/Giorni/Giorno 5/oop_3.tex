 
 
\documentclass[a4paper,10pt]{article}
\usepackage[utf8]{inputenc}
\usepackage[margin=25mm]{geometry}
\usepackage[italian]{babel}
\usepackage{amsmath}
\usepackage{amsthm}
\usepackage{amsfonts}
\usepackage{centernot}
\usepackage{multicol}
\usepackage{tikz}
\usepackage{listings}
\usepackage{courier}
\usepackage{MnSymbol}
\usetikzlibrary{patterns.meta}
\setlength{\parindent}{0em}
\newcommand{\reals}{\mathbb{R}}
\newcommand{\integers}{\mathbb{Z}}
\newcommand{\naturals}{\mathbb{N}}
\newcommand{\cnot}{\centernot}


\definecolor{backcolour}{rgb}{0.95,0.95,0.92}
\lstdefinestyle{mystyle}{
    language=java,
    backgroundcolor=\color{backcolour},
    numberstyle=\tiny,
    basicstyle=\ttfamily\small,
    keywordstyle=\bfseries,
    breakatwhitespace=false,
    breaklines=true,
    captionpos=b,
    keepspaces=true,
    numbers=left,
    numbersep=5pt,
    showspaces=false,
    showstringspaces=false,
    showtabs=false,
    tabsize=2
}

\lstset{style=mystyle}

\begin{document}


\begin{center}
    \LARGE Giorno 5\smallskip

    \Large OOP: Ereditarietà
\end{center}\smallskip

\begin{abstract}
 Nella prima parte si analizza l'ereditarietà in Java: Si introduce la \textbf{classe Object} ed alcuni suoi metodi, si parla del modificatore \texttt{protected} e delle classi astratte. Si introduce infine la gerarchia di classi, e nella seconda parte si parla di come si vanno a cercare in modo efficiente i metodi nelle superclassi.
\end{abstract}

\section{Ereditarietà}
L'eredità in java funziona per estensione esplicita, tramite \texttt{extends}:
\begin{lstlisting}
 class B extends A {...}
\end{lstlisting}
Ed una classe che non estende altre classi estende di default la classe \textbf{Object}.

\paragraph{Nota:} Una classe che ne estende un'altra \textbf{non ne eredita i membri privati}.

\subsection{La classe Object}
La classe Object mette a disposizione alcuni metodi assunti essere presenti in tutte le classi:
\begin{itemize}
 \item \texttt{toString()}: restituisce una rappresentazione testuale dell’oggetto
 \item \texttt{equals(obj)}: confronta l’oggetto corrente con l’oggetto obj
 \item \texttt{clone()}: crea una copia dell’oggetto
 \item ... \texttt{https://docs.oracle.com/javase/8/docs/api/java/lang/Object.html}
\end{itemize}

\subsubsection{Confronto con equals()}
\begin{lstlisting}
 String s1 = "ciao";
 String s2 = s1;
 String s3 = "ciao a tutti"
 String s4 = s3.substring(0,4);         // restituisce sottostringa "ciao"
 System.out.println(s1==s2);            // true (stesso riferimento)
 System.out.println(s2==s4);            // false (riferimenti diversi)
 System.out.println(s2.equals(s4));     // true (stesso contenuto)
\end{lstlisting}

\subsubsection{Le stringhe (letterali) sono speciali}
I letterali, (i.e. stringhe del tipo \texttt{"banana"}) sono \textbf{oggetti immutabili}: i metodi di String non modificano gli oggetti, ma ne restituiscono sempre dei nuovi come risultato.

\subsubsection{Riconoscimento di letterali uguali}
Il compilatore, per ottimizzare, riconosce letterali uguali ed ottimizza il codice \textbf{costruendo un solo oggetto}.

\subsection{Visibilità (protected)}
Oltre ai modificatori di accesso/visibilità che abbiamo visto, esiste: \texttt{protected}: rende visibili i membri \textbf{solo alle alle sotto-classi}, senza renderli del tutto pubblici.

\newpage
\subsection{Classi astratte}
Una \textbf{classe astratta} è una \textbf{classe parzialmente definita}:

\begin{lstlisting}
 public abstract class Solido {
    // variabile d'istanza
    private double pesoSpecifico ;
    
    // costruttore
    public Solido ( double ps ){
        pesoSpecifico = ps;
    }
    
    // metodo implementato
    public double peso (){
        return volume () * pesoSpecifico ;
    }
    
    // metodi astratti
    public abstract double volume ();
    public abstract double superficie ();
 }
\end{lstlisting}

Le classi astratte sono utilizzate estendendole, ed implementando i metodi astratti, in questo caso \texttt{volume} e \texttt{superficie}.

\subsection{Gerarchia di classi}
Il costrutto extends consente di creare una \textbf{gerarchia di classi}, rappresentabile come un albero la cui radice è Object. Dato che in Java alle classi sono associati \emph{tipi di oggetto}, la gerarchia di classi è una rappresentazione della relazione di sottotipo, con $Top = Object$.\smallskip

La regola di subsumption del sistema di tipi ci consente di ottenere un meccanismo di \textbf{polimorfismo per sottotipo}.
\begin{multicols}{2}
 
\begin{itemize}
 \item Proprietà transitiva:
 \[ \dfrac{T_1 <: T_2\quad T_2 <: T_3}{T_1 <: T_3} \]

 \item Subsumption
 \[ \dfrac{\Gamma \vdash_e exp : S \quad S :<T }{\Gamma \vdash_e exp : T}\]
 \item New
 \[ \Gamma \vdash_e new \,\,T() : T \]
 \item Assegnamento
 \[ \dfrac{\Gamma \vdash_e exp : t}{\Gamma \vdash_c T\,\, x = exp} \]
\end{itemize}

\end{multicols}

\subsubsection{Lo Structural subtyping è più debole del Nominal subtyping}

Lo Structural è più debole del Nominal poiché nel nominal vale:

\[ S <: T \implies \text{ogni membro pubblico di $T$ è membro pubblico di $S$} \]

La cui dimostrazione segue dalla definizione di extends/implements e dalla transitività di $<:$


\newpage
\section{Overloading, Overriding, Dynamic Dispatch}
\subsection{Overloading di metodi}
Ad ogni metodo è associata una \textbf{signature}, che contiene il suo nome ed il tipo dei suoi parametri.

\paragraph{Esempi:} 
\begin{center}
 
\begin{tabular}{|c|c|}
\hline
\bf Metodo & \bf Signature\\
\hline
\tt int getVal() & \tt getval\\
\hline
\tt int minimo(int x, int y) &\tt  minimo(int, int)\\
\hline
\tt int minimo(int a, int b) &\tt  minimo(int, int)\\
\hline
\tt double minimo(double a, double b) &\tt  minimo(double, double)\\
\hline
\end{tabular}

\end{center}

\paragraph{Nota} Si noti la signature \textbf{non tiene conto del nome dei parametri}, poiché la signature riassume solo le informazioni che possono essere dedotte dalla chiamata del metodo (quindi non i nomi dei parametri)\medskip


Più metodi di una stessa classe possono \textbf{avere lo stesso nome}, a patto che sia diversa la loro signature (ossia, nella pratica, il nome dei loro parametri). La tecnica di chiamare più metodi con lo stesso nome prende nome di \textbf{overloading}.

\paragraph{Formalmente:}
\[ \dfrac{\forall i \in [1, n]. \Gamma \vdash_e e_i : T_i \quad obj : S \quad T\,\, m(T_1 x_1 , \hdots, T_n x_n)C \in S}{\Gamma \vdash_e obj.m(e_1, \hdots, e_n) : T} \]

Ossia: Se tutti i parametri attuali sono del tipo giusto, la signature è $S$ e la dichiarazione del metodo ha signature $S$, allora si può assegnare tipo $T$ alla chiamata del metodo.
\smallskip

La procedura di scegliere il metodo con la signature corretta si dice \textbf{static dispatch}; \smallskip

(in generale, con il termine \textbf{dispatch} si indica il metodo con cui si collegano le chiamate di funzione/metodo alle loro definizioni)
\subsubsection{Invocare un metodo}

Quando si invoca un metodo, si controlla (staticamente) se questo è presente nella classe che descrive il \emph{tipo apparente} (tipo statico) dell'oggetto. Se il metodo si trova nella classe che descrive il \emph{tipo effettivo}, non sarà trovato automaticamente, ma si deve eseguire un downcast esplicito (se possibile dopo aver controllato il tipo effettivo con \texttt{instanceof}).\smallskip

Se il metodo non è trovato, c'è un errore a tipo di compilazione. 


\subsection{Dynamic Dispatch ed Overriding}
Superati i controlli statici, a tempo di esecuzione la chiamata del metodo può trovarsi nei seguenti casi:
\begin{itemize}
 \item Il metodo è presente nella classe dell'oggetto (ok!)
 \item Il metodo va cercato nelle superclassi
\end{itemize}

Nel secondo caso la ricerca del metodo parte dalla classe corrente (\textbf{tipo effettivo}) e risale la gerarchia. Questo prende il nome di \textbf{dynamic dispatch}.

\subsubsection{Realizzazione efficiente}
Ovviamente visitare l'albero ad ogni chiamata di metodo sarebbe molto inefficiente, perciò la JVM adotta una soluzione basata su:
\begin{itemize}
 \item \textbf{Tabelle dei metodi} (\textbf{dispatch vector}), una tabella con puntatori al codice dei metodi;
 \item \textbf{Sharing strutturale}, i.e. la tabella (dispatch vector) di una sottoclasse \textbf{riprende la struttura} della tabella della superclasse \textbf{aggiungendo righe} per i nuovi metodi (aka \textbf{l'ordine dei metodi in comune è lo stesso}) -- (n.b. le due tabelle sono distinte, è shared solo l'ordine!)
\end{itemize}

Queste due tecniche, permettono di accedere all'unica tabella che descrive le sottoclassi per indice, ed una chiamata di un metodo sarà tradotta nel bytecode con una chiamata \texttt{invokevirtual \#i}, dove $i$ è l'indice nella tabella.

\newpage

\subsubsection{Overriding}
Che succede se ``risalendo la gerarchia'' si trovano due metodi con la stessa firma? \textbf{Si sceglie il primo incontrato}, aka quello definito più di recente. In termini di \textbf{dispatch vector}, le tabelle hanno la stessa struttura, ma i puntatori possono essere diversi a causa dell'overriding.
\subsubsection{Ultime cose}
\begin{itemize}
 \item \textbf{Come si accede alle variabili d'istanza?} (intuitivamente, poi in realtà è più complicato) il compilatore aggiunge \texttt{this} come parametro implicito al metodo:
 
\begin{lstlisting} 
public String getMatricola() { return matricola; } 
\end{lstlisting}
Diventa

\begin{lstlisting}
 public String getMatricola(Studente this) { return this.matricola; }
\end{lstlisting}

\emph{(``a la Python'')}
 \item \textbf{Chiamata ai metodi statici:}
 
 I metodi statici hanno un \textbf{dispatch vector separato}, acceduto tramite \texttt{invokestatic} nel bytecode.
 
 $\implies$ sostanzialmente \textbf{non sono metodi}
\end{itemize}

\end{document}
 
