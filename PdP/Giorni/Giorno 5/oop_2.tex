 
 
\documentclass[a4paper,10pt]{article}
\usepackage[utf8]{inputenc}
\usepackage[margin=25mm]{geometry}
\usepackage[italian]{babel}
\usepackage{amsmath}
\usepackage{amsthm}
\usepackage{amsfonts}
\usepackage{centernot}
\usepackage{multicol}
\usepackage{tikz}
\usepackage{listings}
\usepackage{courier}
\usepackage{MnSymbol}
\usetikzlibrary{patterns.meta}
\setlength{\parindent}{0em}
\newcommand{\reals}{\mathbb{R}}
\newcommand{\integers}{\mathbb{Z}}
\newcommand{\naturals}{\mathbb{N}}
\newcommand{\cnot}{\centernot}


\definecolor{backcolour}{rgb}{0.95,0.95,0.92}
\lstdefinestyle{mystyle}{
    language=java,
    backgroundcolor=\color{backcolour},
    numberstyle=\tiny,
    basicstyle=\ttfamily\small,
    keywordstyle=\bfseries,
    breakatwhitespace=false,
    breaklines=true,
    captionpos=b,
    keepspaces=true,
    numbers=left,
    numbersep=5pt,
    showspaces=false,
    showstringspaces=false,
    showtabs=false,
    tabsize=2
}

\lstset{style=mystyle}

\begin{document}


\begin{center}
    \LARGE Giorno 5\smallskip

    \Large OOP: Java
\end{center}\smallskip

\section{Java}
\subsection{Struttura di un programma Java}
Un programma Java è \emph{un insieme di classi}. 
\begin{itemize}
 \item In linea di principio: ogni classe in un file diverso
 \item Una delle classi dovrà avere un metodo \emph{main} da cui parte l'esecuzione del programma.
\end{itemize}
\subsubsection{Hello World}
\begin{lstlisting}
 public class HelloWorld {
    public static void main (String [] args) {
        System.out.println("Hello World !");
    }
 }
\end{lstlisting}
\subsubsection{Nucleo imperativo}
Tutto uguale a C, tranne il fatto che non ci sono puntatori, gli array hanno una sintassi diversa:
\begin{lstlisting}
 int[] numeri;      // attenzione alle parentesi:
                    // "int numeri[]" e' sbagliato
 // inizializzazione
 numeri = new int[10];  // dimensione 10 e valori di default (0)
 numeri = {5,3,2,6};    // dimensione 4 e valori 5,3,2 e 6
 
 // anche insieme
 int[] numeri = new int[10
\end{lstlisting}

e il tipo \texttt{String} è in realtà una speciale classe di libreria; la concatenazione si fa con $+$, come se fosse un tipo primitivo.

\subsection{Classi}
\begin{lstlisting}
 class Name {
    public ...
    private ...
    public static ...
 }
\end{lstlisting}

\begin{itemize}
 \item \texttt{public}: l'attributo/metodo si può accedere anche dall'esterno della classe
 \item \texttt{private}: l'attributo/metodo non si può accedere dall'esterno
 \item \texttt{static}: un metodo statico può essere chiamato senza istanziare la classe.
 \item ... \textbf{final, abstract, transient, synchronized, volatile}
\end{itemize}
\subsubsection{Incapsulamento}
I modificatori \texttt{public} e \texttt{private} consentono di realizzare sistemi di \emph{incapsulamento} (i.e. la rappresentazione dell'oggetto rimane nascosta/privata, l'accesso dall'esterno è consentito solo attraverso un certo numero di metodi pubblici)\smallskip

Si possono definire anche \emph{metodi privati}, utilizzabili all'interno della classe (metodi ausiliari).\newpage

\subsection{Interfacce}
Un'interfaccia contiene le intestazioni dei membri pubblici di una classe e non contiene costruttori:
\begin{lstlisting}
 public interface BankAccount {                         // interface, non class
    public double getSaldo();
    public void versa(double somma);
    public boolean preleva(double somma);
 }
\end{lstlisting}

Per implementare un'interfaccia in una classe si scrive:
\begin{lstlisting}
 public class ContoCorrente implements BankAccount {    // implements!
    private double saldo;
    public ContoCorrente(double saldoIniziale) { ... }
    public void versa(double somma) { ... }
    public double getSaldo() { return saldo; }
    public boolean preleva(double somma) {
        if (saldo>=somma) { ... }
        else return false;
    }
 }
\end{lstlisting}

\textbf{Nessun costruttore!} Se ne usa uno di default e senza parametri che inizializza le variabili come da dichiarazione, oppure (se la dichiarazione non
prevede un assegnamento) a valori standard.

\subsubsection{A che serve implements?}
Usando \texttt{BankAccount} come tipo, abbiamo appena definito la relazione di sottotipo:
\[\texttt{ContoCorrente <: BankAccount}\]
\subsubsection{implementare più interfacce}
Una classe può implementare più interfacce:
\begin{lstlisting}
 public class ContoFlessibile implements BankAccount,DepositAccount {...}
\end{lstlisting}
Si avrà: 
\[\texttt{ContoFlessibile <: BankAccount\quad\quad ContoFlessibile <: DepositAccount}\]

\subsection{Tipo apparente e tipo effettivo}

\begin{itemize}
 \item Il Tipo Apparente (o \textbf{statico}) è il tipo usato dal compilatore per fare i controlli;
 \item Il Tipo Effettivo (o \textbf{dinamico}) è il tipo che l'oggetto avrà a runtime.
\end{itemize}

E.g. Dichiaro due variabili di tipo \texttt{BankAccount}, istanze di due diversi sottotipi di BankAccount:

\begin{lstlisting}
 BankAccount conto1 = new ContoCorrente(1000);
 BankAccount conto2 = new ContoLimitato(200,10);
\end{lstlisting}

Il \textbf{tipo apparente} di entrambe è \texttt{BankAccount}, mentre i \textbf{tipi effettivi} sono \texttt{ContoCorrente} e \texttt{ContoLimitato}.

\subsubsection{Cast/Coercion}
\begin{itemize}
 \item Il cast da sottotipo a supertipo (\textbf{upcast}) è implicito
 \item Il cast da supertipo a sottotipo (\textbf{downcast}) deve essere esplicito (stessa sintassi di C).
\end{itemize}

In OCaml il downcast non è possibile.



\newpage
\subsection{Membri statici e d'istanza}
\begin{itemize}
 \item I membri (variabili e metodi) \textbf{d'istanza} sono quelli che codificano lo stato di \textbf{un singolo oggetto} (un'\textbf{istanza} della classe)
 \item I membri \textbf{statici} codificano operazioni \textbf{di classe} (non operano sullo stato dei singoli oggetti).
\end{itemize}

\subsubsection{Esempio di utilizzo}
Se vogliamo assegnare un numero univoco ad ogni oggetto di una classe, si utilizzano una \textbf{variabile d'istanza} per mantenere il numero del singolo oggetto ed una \textbf{variabile statica} per mantenere il conto dei numeri assegnati (e.g. mantenere il numero di oggetti istanziati, se assegnamento lineare)

\subsection{Modello della memoria}
Identifichiamo tre aree nella JVM:
\begin{itemize}
 \item \textbf{Ambiente delle classi - Workspace}, che contiene il codice dei metodi e le variabili statiche
 \item \textbf{Stack}, contiene i RdA dei metodi con le variabili locali
 \item \textbf{Heap}, contiene gli oggetti (raggiungibili tramite \textbf{riferimenti}), con le loro variabili d'istanza
\end{itemize}

\subsubsection{Descrizione del funzionamento}

\begin{itemize}
 \item Inizialmente le classi sono caricate nell'ambiente delle classi 
 \item Poi il RdA del metodo main è caricato nello stack
 \item Gli oggetti creati con \textbf{new} dal metodo \texttt{main} (e poi quelli creati dai metodi degli oggetti chiamati da \texttt{main}, ecc.) sono caricati nello heap, e mantenuti per riferimento dai RdA
 \item I RdA dei metodi chiamati dagli oggetti vengono caricati nello stack.
\end{itemize}




\end{document}
