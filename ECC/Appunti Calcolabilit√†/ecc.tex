\documentclass[a4paper,10pt,oneside]{article}
\usepackage[utf8]{inputenc}
\usepackage[margin=1in]{geometry}
\usepackage[italian]{babel}
\usepackage{mathtools}
\usepackage{systeme}
\usepackage{multicol}
\usepackage{amsfonts}
\usepackage{amssymb}
\usepackage{amsthm}
\usepackage{fancyhdr}
\usepackage{lastpage}
\usepackage{graphicx}
\usepackage{listings}
\usepackage{color}
\usepackage{proof}
\usepackage{amsmath}
\usepackage{courier}
\usepackage{wrapfig}
\usepackage[dvipsnames]{xcolor}
\usepackage{tikz}

\usepackage{mdframed}

\usetikzlibrary{arrows.meta}
\graphicspath{{./images/}}

\definecolor{backcolour}{rgb}{0.95,0.95,0.92}
\lstdefinestyle{mystyle}{
    language = sql,
    backgroundcolor=\color{backcolour},
    numberstyle=\tiny,
    basicstyle=\ttfamily\footnotesize,
    breakatwhitespace=false,
    breaklines=true,
    captionpos=b,
    keepspaces=true,
    numbers=left,
    numbersep=5pt,
    showspaces=false,
    showstringspaces=false,
    showtabs=false,
    tabsize=2
}

\lstset{style=mystyle}
\setlength{\parindent}{0em}
\setlength{\parskip}{0em}

\newtheoremstyle{break}
{\topsep}{\topsep}{\hangindent=1em}{}{\bfseries}{\\}{2pt}{}

\theoremstyle{break}
\newtheorem{deff}{Definizione}[section]

\newtheorem{prop}{Proposizione}[subsection]
\newtheorem{teo}{Teorema}[subsection]
\newtheorem*{es}{Esempio}
\newtheorem{cor}{Corollario}[teo]

\newcommand{\dom}{\text{dom}}
\newcommand{\naturals}{\mathbb {N}}
\newcommand{\RE}{\mathcal {RE}}
\newcommand{\R}{\mathcal {R}}
\newcommand{\id}{\text{id}}
\newcommand{\rec}{\text{rec}}

\begin{document}
\section{Elementi di calcolabilità}
\subsection{Alcuni risultati classici}
\begin{mdframed}
\begin{teo} [Numero funzioni calcolabili, esistenza funzioni non calcolabili]
\begin{enumerate}
 \item []
 \item Le funzioni calcolabili sono $\#(\naturals)$; Inoltre le funzioni calcolabili \textbf{totali} sono $\#(\naturals)$.
 \item Esistono funzioni non calcolabili
\end{enumerate}
\end{teo}

\dotfill

\paragraph{Intuizione:} \textbf{Le funzioni calcolabili sono tante quante le macchine di Turing}, che si possono enumerare. Inoltre, \textbf{ci sono più funzioni che funzioni calcolabili}

\dotfill

\begin{proof}
\begin{enumerate}
 \item []
 \item Esistono \emph{almeno} $\#(\naturals)$ funzioni calcolabili (totali) poiché possiamo costruire $\#(\naturals)$ MdT $M_i$ che:
 \begin{itemize}
  \item Svuotano il nastro
  \item Scrivono la stringa $|^i$
  \item Si arrestano
 \end{itemize}
 Le funzioni calcolabili \emph{non sono più di} $\#(\naturals)$, poiché le MdT si possono enumerare.
 \item Con una costruzione analoga a quella di Cantor (argomento diagonale) si vede che $\{f : \naturals \to \naturals\}$ ha cardinalità $2^{\aleph_0}$, quindi ci sono più funzioni di quante funzioni calcolabili.
\end{enumerate}
\end{proof}
\end{mdframed}

\begin{mdframed}
\begin{teo}[Padding Lemma]
 Ogni funzione calcolabile $\varphi_x$ ha $\#\mathbb (N)$ indici, e l'insieme degli indici:
 \[ A_x \text{ t.c. } \forall y \in A_x \,\,.\,\, \varphi_y = \varphi_x \]

 si può costruire mediante una \textbf{funzione ricorsiva primitiva}.
\end{teo}
\dotfill
\paragraph{Intuizione:} Posso inserire uno \textbf{skip} alla fine di un programma senza alterarne il comportamento; Per le MdT inserisco un nuovo stato ed una quintupla che non fa niente.

\dotfill
\begin{proof}
 Per ogni MdT $M_x$, se $Q = \{ q_0, \hdots, q_k\}$, si ottiene una nuova MdT $M_{x_1}$, con $x_1 \in A_x$, aggiungendo uno stato $q_{k+1}$ ed una quintupla ``che non fa niente'': $(q_{k+1}, \#, q_{k+1}, \#, -)$.\smallskip


 Questo processo può essere ripetuto una quantità numerabile di volte.
\end{proof}

\end{mdframed}

\newpage

Come si vede intuitivamente dallo schema di un interprete, \textbf{bastano un while e dei for} per rappresentare una funzione calcolabile. Detto in maniera più formale:

\begin{center}
 \it Si possono rappresentare le funzioni calcolabili come \textbf{composizione di una funzione ricorsiva primitiva ed una funzione ricorsiva generale}.
\end{center}

\begin{mdframed}
\begin{teo}[Forma Normale - Kleene I]

Esistono un predicato $T(i, x, y)$ ed una funzione $U(y)$ \textbf{calcolabili totali} e \textbf{ricorsive primitive} tali che:

\[ \forall i, x . \varphi_i (x) = U(\mu y.T(i, x, y)) \]

\end{teo}
\paragraph{Nota:} Dato che $T$ è totale \textbf{converge sempre}, quindi $\mu y.T(i, x, y)$ è una \textbf{funzione ricorsiva generale}.

\dotfill

\paragraph{Intuizione:} $T$ è vero iff $y$ è codifica di una computazione terminante della $i$-esima MdT; $U$ dato l'indice $y$ di una computazione terminante restituisce il risultato della computazione. Componendo le due funz. si ottiene il risultato della computazione terminante di indice minimo, che è risultato di $\varphi_i(x)$

\dotfill

\begin{proof}
Si dimostra mostrando due funzioni che soddisfano il requisito:

\begin{itemize}
 \item $T(i, x, y)$ è \textbf{il predicato di Kleene}:
 \[ T(i, x, y) = tt \iff \text{ $y$ è la codifica di una computaz. \textbf{terminante} di $M_i$ con dato iniziale $x$} \]

 Il calcolo di $T$ avviene tramite i seguenti passi:

 \begin{enumerate}
  \item Dato $i$, recupera la macchina $M_i$
  \item Decodifica $y$
  \item Dato $x$ si controlla se il risultato è una computazione terminante della forma $M_i(x) = c_0 \hdots c_n$
 \end{enumerate}


 Questa sequenza di passi \textbf{termina sempre} $\implies$ $T$ \textbf{totale}.

 \item Sia $y$ la codifica della computazione \textbf{terminante} $M_i(x) \to ^k (h, \triangleright z \#)$, allora $U$ è tale che: \[U(y) = z\]

 I passi necessari a questo calcolo sono finiti (decodifica $y$, ottieni sequenza $c_0\hdots c_n$, accedi a $c_n$) e terminano tutti, quindi anche $U$ è \textbf{totale}.

\end{itemize}

L'intero procedimento è \emph{\textbf{effettivo}}, quindi $U$ e $T$ sono \textbf{calcolabili} per la tesi di \textbf{Church-Turing}

Inoltre le due funzioni sono \textbf{ricorsive primitive} perché

\begin{itemize}
 \item Le codifiche che usiamo lo sono
 \item I controlli effettuati lo sono
\end{itemize}


\end{proof}
\begin{cor}
Le funzioni $T$-calcolabili sono $\mu$-ricorsive (per la \textbf{nota} del teorema).
\end{cor}

\end{mdframed}

\newpage

\begin{mdframed}
\begin{teo}[Enumerazione, o della MdTU]
 \[ \exists z . \forall \,\,i, x\,\, . \,\,\varphi_i(x) = \varphi_z(i, x) \]

 Ossia esiste una \textbf{funzione calcolabile parziale} $\varphi_z$ ``a due posti'' che dati in input l'indice $i$ di una funzione calcolabile parziale $\varphi_i$ ``ad un posto'' ed il suo parametro $x$ restituisce $\varphi_i(x)$.
\end{teo}

\dotfill

\begin{proof}
Si usa $z : \varphi_z(i, x) = U(\mu y . T (i, x, y)) = \varphi_i(x)$, dove $T$ è il predicato di Kleene
\end{proof}

\dotfill\smallskip

\textbf{Intuitivamente:} $M_z$ \textbf{recupera la descrizione} di $M_i$ e la applica a $x$.\newpage


\end{mdframed}

\paragraph{Applicazione parziale:} data una funzione di due argomenti, posso \emph{fissarne uno} ed ottenere una funzione ad un argomento, e.g.

\[ f(x, y) = x+y \underset{\text{fisso }x=5}{\longrightarrow} f(y) = 5+y\]
\begin{mdframed}
\begin{teo}[Del parametro, $S_1^1$]
Esiste una funzione $s$ calcolabile totale ed \textbf{iniettiva} tale che:
\[\forall \,\,i, x \,\,.\,\, \varphi_{s(i, x)} = \lambda y.\varphi_i(x, y)\]
\end{teo}

\dotfill

\paragraph{Intuizione:} Basta fissare ``in memoria'' (o ambiente) il \textbf{parametro} $x$. L'indice della funzione con il parametro fissato dipende dalla scelta del parametro e dall'indice della funzione ``a due posti''.

\dotfill

\begin{proof} [Dimostrazione (sempre molto ``intuitiva'')] Si ``predispone lo stato iniziale'', fissando $x$ in anticipo (e.g. si sostituisce l'operazione che legge il parametro $x$ con $x := k$. Questa è una \textbf{procedura effettiva}); In questo modo abbiamo che una tale funzione esiste, ma è iniettiva?\smallskip

Se non lo fosse, si può costruire $s'$ che genera indici (che esistono e sono $\aleph_0$ per il padding lemma) in modo strettamente crescente. La funzione s' è strettamente crescente e totale, perciò è iniettiva.
\end{proof}

\dotfill

\paragraph{Generalizzazione:} Data una funzione di $m + n$ parametri ne posso fissare $m$ ed ottengo una funzione n-aria.

\begin{teo}[Del parametro, $S_n^m$ -- non dimostrato]
$\forall \,\,m, n \geq 0\,\, \exists\,\, s \text{ calcolabile totale \textbf{iniettiva} ad $m+1$ posti t.c. }\forall i, (x_1, \hdots, x_m)$:
\[ \varphi_s^{(n)}(i, \vec x) = \lambda \vec y . \varphi_i^{(m+n)} (\vec x, \vec y)\]
\end{teo}


Si noti come il teorema del parametro e quello di enumerazione siano ``l'uno l'inverso dell'altro'', in quanto uno ``aumenta'' il numero di argomenti e l'altro lo diminuisce.


\end{mdframed}


L'importanza dei teoremi di enum e parametro è sintetizzata nel seguente teorema, non dimostrato:
\begin{mdframed}
\begin{teo}[espressività]
Un formalismo è Turing-equivalente \textbf{\emph{se e solo se}}:
\begin{itemize}
 \item Vale il teorema di enumerazione (i.e. ha un algoritmo ``universale'')
 \item Vale il teorema del parametro
\end{itemize}

\end{teo}

\end{mdframed}

\newpage
\begin{mdframed}
\begin{teo}[Di Ricorsione, Kleene II]
\[ \forall f \text{ calc. totale } \exists(n) . \varphi_n = \varphi_{f(n)} \]

Per ogni funzione calcolabile totale esiste un indice che ne è \textbf{punto fisso}, ossia tale che $\varphi_n = \varphi_{f(n)}$
\end{teo}

\dotfill

\paragraph{Intuizione:} La funzione $f$ ``trasforma'' programmi in altri programmi; quando si considera il punto fisso, la trasformazione operata \textbf{non cambia la funzione calcolata}, ovvero trasforma un programma in un programma diverso \textbf{con la stessa semantica}.\smallskip

\dotfill

\begin{proof}
Definiamo la seguente funzione calcolabile:
\[ \psi(u, z) = \begin{cases}
                                  \varphi_{\varphi_u(u)} (z) &\text{se } \varphi_u(u) \downarrow\\
                                  \text{indefinita} & \text{altrimenti}
                                 \end{cases}
 \quad\quad (1) \]
\begin{itemize}
 \item $\psi(u, z) = \varphi_i(u, z)$ \hfill (Church-Turing, calcolabile $\implies \exists$ indice $i$)
 \item $\exists g : \varphi_i (u, z) = \varphi_{g(i, u)}(z)$ \hfill  (Teorema del parametro)
 \item Definisco: $\lambda x . g(i, x) = d(x)$\hfill  (\emph{Barbatrucco})
 \item $\implies \varphi_{g(i, u)}(z) = \varphi_{d(u)}(z)$ \hfill (Applico il \emph{Barbatrucco})
 \item $f(d(x)) = \varphi_v(x)$ \hfill (CT, calcolabile $\implies \exists$ indice $v$)
 \item \[f,d \text{ totali }\implies f(d(x)) = \varphi_v(x)\text{ totale }\implies \varphi_v(v) \downarrow \quad\quad (2)\]
 \item $\implies \varphi_{d(v)}(z) = \psi(v, z) = \varphi_{\varphi_u(u)}(z)$ \hfill (1)
 \item \[ n := d(v) \quad\quad\quad (3)\]
\end{itemize}

Adesso possiamo dimostrare che $n$ è punto fisso di $f$:
\[ \varphi_n \overset{(3)}{=} \varphi_{d(v)} \overset{(1)}{=} \varphi_{\varphi_v(v)} \overset{(2)}{=} \varphi_{f(d(v))} \overset{(3)}{=} \varphi_{f(n)} \]

\end{proof}

\end{mdframed}
\bigskip
\subsection{Problemi insolubili e riducibilità}
\begin{deff}[Insieme ricorsivo]
 Un insieme $I$ si dice \textbf{ricorsivo} se la sua funzione caratteristica è \textbf{calcolabile totale}.
\end{deff}

\paragraph{Intuizione} Esiste un algoritmo che, dato $x$ in input, determina se $x \in I$ in un numero finito di passi.

\paragraph{Nota:} talvolta con ``funzione ricorsiva'' si intende ``funzione ricorsiva generale totale'', mentre le funzioni ricorsive generali \emph{parziali} vengono dette semplicamente ``ricorsive generali''.

\begin{deff}[Insieme ricorsivamente enumerabile]

Un insieme $I$ si dice \emph{ricorsivamente enumerabile} se e solo se $\exists i.I = dom(\varphi_i)$, ossia sse è dominio di almeno una funzione calcolabile parziale. 
\paragraph{Intuizione} 
$S$ è r.e. se $\exists$ algo che, dato in input $x$, termina se $x \in S$; alternativamente, r.e. se esiste un algoritmo capace di enumerare gli elementi di $S$

\end{deff}
\newpage
\begin{mdframed}
\begin{teo} $I$ ricorsivo $\implies I$ ricorsivamente numerabile
\end{teo}
\begin{multicols}{2}
\begin{proof}
 La funzione $\varphi_i$ di cui $I$ è dominio converge su $x$ se e solo se $\chi_I(x) = 1$, e.g:

  \[ \varphi_i = \begin{cases}
                  1 & \chi_I(x) = 1\\
                  \text{indef.} & \text{altrim.}
                 \end{cases}
 \]

\end{proof}


\end{multicols}

\end{mdframed}
\begin{mdframed}


\begin{teo} $I$ (e $\bar I$) ricorsivamente enumerabili $\iff$ $I$ (e $\bar I$) ricorsivi.

\end{teo}

\dotfill

\begin{proof}
 ``Se'': vd punto precedente. ``Solo se'': Siano $\varphi_i(x)$ e $\varphi_{\bar i}(x)$ le funzioni i cui domini sono risp. $I$ e $\bar I$; Allora si esegue il seguente ciclo:
 \begin{itemize}
  \item Esegui un passo nel calcolo di $\varphi_i(x)$; se $\varphi_i(x)\downarrow$ allora $x \in I$, $\chi_I(x) = 1$
  \item Altrimenti esegui un passo nel calcolo di $\varphi_{\bar i}$; se $\varphi_{\bar i} \downarrow$ allora $x \notin I$, $\chi_{\bar I} = 0$.
 \end{itemize}

\end{proof}
\end{mdframed}

\begin{mdframed}

 \begin{teo}

 \begin{multicols}{2}
   $I \neq \emptyset$ è ricorsivamente enumerabile $\iff \exists f$ calcolabile tale che $I = \text{imm} (f)$
   
     \begin{center}
  \begin{tabular}{|c|c|c|c}

   9 & 13 & 18 & 24\\
   \hline
   5 & 8 & 12 & 17\\
   \hline
   2 & 4 & 7  & 11\\
   \hline
   0 & 1 & 3  & 6\\
   \hline
  \end{tabular}
  \end{center}
 \end{multicols}
  \end{teo}
 
 \paragraph{Intuizione} $(\Rightarrow)$
  Devo costruire una funzione la cui immagine è $I$. \smallskip

  Per ogni $m, n \in \naturals$, sia $\langle m, n \rangle$ la codifica di $(m, n)$, ossia l'elemento in posizione $(m, n)$ della tabella:
  \begin{itemize}
   \item Se $\varphi_i(n)$ termina in $m$ passi, $f(\langle m, n \rangle) = n$
   \item Se no $f(\langle m, n \rangle) = \bar n$ che si ottiene spostandosi ripetutamente all'intero successivo $\langle m, n \rangle + 1$ nella tabella, finché per qualche $\langle m, \bar n \rangle$ vale che $\varphi_i(\bar n)$ termina in $m$ passi.
  \end{itemize}


 \dotfill

 \begin{proof}
  \begin{itemize}
   \item []
   \item[$(\Rightarrow)$] $I = \text{dom}(\varphi_i) \neq \emptyset$, costruisco $f$ : $I = \text{imm}(f)$
   \begin{itemize}
    \item Si cerca un elemento di $I$ mediante un procedimento a coda di colomba
    \item Sia $n$ l'argomento di $\varphi_i$ ed $m$ il numero di passi nel calcolo di $\varphi_i(n)$.
    \item Chiamiamo $\langle n, m \rangle$ la codifica di $(m, n)$
    \item Si calcolano $m$ passi del calcolo di $\varphi_i(n)$:
    \begin{itemize}
     \item se il calcolo termina, si pone $f(\langle m, n \rangle) = n$;
     \item se il calcolo non termina, si prosegue (si considera $\langle m, n \rangle + 1$ muovendosi ``a coda di colomba'') finché per qualche $m, \bar n$ il calcolo non si arresta, e si pone $f(\langle m, n \rangle) = \bar n$.
    \end{itemize}

    $\bar n$ è sicuramente in $I$, quindi con questo procedimento si generano tutti gli elementi di $I$.
   \end{itemize}
    \item[$(\Leftarrow)$] (\emph{non riportata da Degano, ma credo sia questo:}) Se $I$ è immagine di una funzione calcolabile totale $f$, allora è dominio della funzione $g$ costruita considerando, per ogni $y \in I$, \textbf{uno} degli $x : f(x) = y$ e ponendo $g(y) = x$.
  \end{itemize}

 \end{proof}


\end{mdframed}

\newpage
\begin{deff}
  Chiamiamo $rec$ l'insieme delle funzioni ``ricorsive'', allora:
  \[I \text{ è }\begin{cases}
    \text{ricorsiva} &\iff \chi_I \in rec = \{ \varphi_x : \text{ dom } (\varphi_x) = \naturals \}\\
    \text{ric. enumerabile} &\iff I = \text{ dom } (\varphi_x)
    \end{cases}
\]
\end{deff}

Inoltre chiamiamo:
\begin{multicols}{2}

\begin{itemize}
 \item L'insieme degli insiemi ricorsivi $\R$
 \item L'insieme degli insiemi r.e. $\RE$
 \item L'insieme degli insiemi non r.e. $non\RE$
\end{itemize}
E vedremo che $\R \subsetneq \RE \subsetneq non\RE$
\end{multicols}


Se $I$ è ricorsivamente enumerabile, posso determinare l'appartenenza all'insieme $I$ in tempo finito, ma non posso determinare la non appartenenza. Ad esempio:
\begin{mdframed}

\begin{prop}
\label{K}
L'insieme:
\[ K = \{ i : \varphi_i(i) \downarrow\} \]
È ricorsivamente enumerabile.
\end{prop}

\dotfill

\begin{proof}
 $K$ è dominio di:
 \[\psi(n) = \begin{cases}
              1 & \text{se } \varphi_n(n)\downarrow\\
              indef & \text{o/w}
             \end{cases}
\]
\end{proof}

\end{mdframed}

Ma \textbf{non tutti gli insiemi r.e. sono ricorsivi!} Infatti:

\begin{mdframed}

\begin{prop} L'insieme:
\[ K = \{ i : \varphi_i(i) \downarrow\} \]
Non è ricorsivo.
\end{prop}

\dotfill

\paragraph{Significato intuitivo}

Non esiste un algoritmo per decidere se $x \in K$ o no. Il problema è \emph{insolubile}.

\dotfill

\begin{proof}
Per assurdo: sia $k$ ricorsivo $\implies \chi_K \in rec$, cioè è calcolabile totale. Sia:

\[ \psi(n) = \begin{cases}
              \varphi_n(n) + 1 & \text{se } n \in K (\iff \chi_K(n) = 1 \iff \varphi_n(n) \downarrow)\\
              0 & \text{o/w}
             \end{cases}
 \]

 Se $\chi_K$ è ricorsiva (i.e. calcolabile totale) allora anche $\psi(n)$ lo è (è definita a partire da $\chi_K$), ma se scegliamo un qualsiasi $\bar n$ come indice della funzione:
\[\psi(\bar n) \neq \varphi_{\bar n} (\bar n)\]

\end{proof}

\dotfill

\paragraph{Corollario} $\bar K$ non è ricorsivamente enumerabile, poiché per il teorema su $I$ e $\bar I$ abbiamo che $K$ non ric $\implies \lnot(K$ r. $\wedge\,\, \bar K$ r.$) \implies \lnot(K$ r. e. $\wedge\,\, \bar K$ r. e.$)$, ma dato che $K$ è r.e. allora $\bar K$ è non r.e.

Potremmo dire che $\bar K \in \text{co-}\mathcal{RE}$, la classe dei problemi i cui complementi sono r.e.
\end{mdframed}
\newpage

\begin{mdframed}
 \begin{teo}[Problema della fermata, in realtà altro \textbf{Corollario}]
  Sia $K_0 = \{(x, y) : \varphi_y(x) \downarrow\} = \{ (x, y) : \exists z . T(y, x, z) \}$, dove $T$ è il pred. di Kleene.\smallskip

  Questo non è ricorsivo.

  \dotfill

 \end{teo}
 \begin{proof}
  $x \in K \iff (x, x) \in K_0$, quindi se $K_0$ fosse ricorsivo lo sarebbe anche $K$.
 \end{proof}

\end{mdframed}

Questo è un esempio di \textbf{riduzione}.

\begin{deff}
 $f$ è riduzione da $A$ a $B$ se $x \in A \iff f(x) \in B$, e si indica: $a \leq_f B$
\end{deff}
\begin{deff}
Dato un insieme di funzioni $\mathcal{F}$, $A \leq_\mathcal{F}B \iff \exists f \in \mathcal F : A \leq_f B$
\end{deff}

\begin{center}
\begin{tikzpicture}
 \draw (-2.5, -2.5) rectangle (2.5,2.5);
 \draw (0,0) circle (2);
 \draw (.3,0) circle (1.2);
 \node () at (1.6,1.6) {$\mathcal E$};
 \node () at (2,2.8) {$\mathcal H$};
 \node () at (-1,.6) {$\mathcal D$};
 \node (b) at (-.3,.5){$B$};
 \node (d) at (.6,.7){$D$};
 \node (c) at (1,-1.5){$C$};

 \draw[-latex] (b) -- (d);
 \draw[-latex] (d) -- (c);
 \draw[-latex, dashed] (b) -- (c);
 \draw[{latex}{Rays}-, dashed] (-.4,-.4) -- (-.8, -.8);
\end{tikzpicture}
\end{center}

Una relazione di riducibilità è un \textbf{preordine}, i.e. una relazione transitiva e riflessiva.

Si dice che $\mathcal F$ classifica $\mathcal D$ ed $\mathcal E$ sse:
\begin{enumerate}
 \item $A \leq_\mathcal{F} A$ \hfill (Riflessività)
 \item $A \leq B$, $B \leq C \iff A \leq C$ \hfill(Transitività)
 \item $A \leq B$, $B \in \mathcal D \implies$ $A \in \mathcal D$ \hfill(no ``frecce entranti'' in $\mathcal D$)
 \item $A \leq B, B \in \mathcal E \implies A \in \mathcal E$\hfill(no ``frecce entranti'' in $\mathcal E$)
\end{enumerate}\medskip

\dotfill \textsc{meta-digressione}\dotfill

Le ultime due proprietà derivano dal fatto che $\mathcal D$ e $\mathcal E$ sono \textbf{ideali} del \emph{proset} $(\mathcal H, \leq)$, ossia

\begin{itemize}
 \item $\forall\,\, B \in \mathcal D ,\quad A \in \mathcal H \quad.\quad A \leq B \implies A \in \mathcal D$ \hfill (Si dice che $\mathcal D$ è \textbf{chiuso verso il basso})
\end{itemize}
 [Se il preordine del \emph{proset} è un ordine parziale, un ideale deve essere anche un \textbf{insieme diretto}, i.e. $a, b \in I \implies \exists c : a \leq c, b \leq c$]


\dotfill

Le proprietà 1-4 possono anche essere espresse in modo equivalente:
\begin{enumerate}
 \item $\text{id} \in \mathcal{F}$
 \item $f, g \in \mathcal F \implies f \circ g \in \mathcal F$
 \item $f \in \mathcal F$, $B \in \mathcal D \implies \{f : f(x) \in B\} \in \mathcal D$
 \item $f \in \mathcal F$, $B \in \mathcal E \implies \{f : f(x) \in B\} \in \mathcal E$
\end{enumerate}
\newpage

Supponiamo che $\leq_\mathcal{F}$ classifichi $\mathcal{D}$ ed $\mathcal{E}$.

\dotfill

\begin{deff}
 $H$ si dice $\leq_\mathcal{F}$-arduo per $\mathcal{E}$ se e solo se:
 
 \[ \forall A \in \mathcal{E} : A \leq_\mathcal{F} H \]
\end{deff}

\paragraph{Intuizione:} Ogni $A$ è al più \emph{difficile} quanto $H$. Parallelismo con i problemi $NP$-ardui, i.e. i problemi ``più difficili'' della classe $NP$.

\paragraph{Difficile?} Perché un problema $A$ si possa ridurre ad un problema $B$, $B$ deve essere difficile almeno quanto $A$. Possiamo quindi interpretare $A \leq B$ come $A$ difficile al più quanto $B$.

\dotfill

\begin{deff}
 $C$ si dice $\leq_\mathcal{F}$-completo per $\mathcal{E}$ se e solo se:
 
 \[ C \text{ è }\leq_\mathcal{F}\text{-arduo per } \mathcal{E}, \quad H \in \mathcal{E} \]
\end{deff}

\paragraph{Intuizione:} Se $C$ è difficile quanto o più di ogni altro elemento di $\mathcal{E}$ ed è in $\mathcal E$, vuol dire che \textbf{il più difficile} problema\footnote{si parla di problema in senso di problema di decisione di appartenenza all'insieme} che è interno ad $\mathcal E$.\smallskip

Da ciò si può inoltre intuire che due problemi completi rispetto ad $\mathcal E$ saranno equivalenti, i.e. l'uno si può ridurre nell'altro.

\dotfill

\begin{deff}
 Con \textbf{gradi} di una relazione di riduzione si intendono le classi di equivalenza rispetto alla relazione $\equiv$, dove:
 
 \[A \equiv B \iff A \leq B, B \leq A\]
\end{deff}

\dotfill

\begin{mdframed}
 \begin{prop}
  Se $\leq$ classifica $\mathcal D$ ed $\mathcal E$, $\mathcal D \subseteq \mathcal E$, $C$ completo per $\mathcal E$, allora \[C \in \mathcal D \iff \mathcal D = \mathcal E\]
 \end{prop}
 
\dotfill
\paragraph{Intuizione} S(s)e un problema $NP$-completo si trovasse nella classe $P$, si avrebbe che $P=NP$

\dotfill

\begin{proof}
 \begin{itemize}
  \item []
  \item Se: ovvia
  \item Solo se: $C \in \mathcal{D}$, $A \in \mathcal E$. Per completezza $A \leq C$, e per chiusura verso il basso $A \in D$; di conseguenza $\mathcal E \subseteq \mathcal D$, e dato che $\mathcal D \subseteq \mathcal E$ si ha la tesi.
 \end{itemize}

\end{proof}

\end{mdframed}

\newpage

\begin{mdframed}
\begin{prop}
 $C$ $\leq$-completo per $\mathcal E$, $C \leq E$ allora anche $E$ è $\leq$-completo per $\mathcal E$
 
 \dotfill
 
 \begin{proof}
 $\forall D \in \mathcal E, D \leq A$ per completezza, ma $\leq$ classifica $\mathcal{D}$ ed $\mathcal{E}$, quindi \[D \leq A,\quad A \leq B \implies D \leq B\]quindi $B$ è arduo. Ma $B \in \mathcal E$, quindi è anche completo.
 
 \end{proof}

\end{prop}

\end{mdframed}

\begin{mdframed}

\begin{multicols}{2}  
\begin{prop} L'insieme di funzioni $rec = \{\varphi_x : \dom (\varphi_x) = \naturals\}$ (delle funzioni calcolabili totali) \textbf{classifica} $\R$ ed $\RE$.
\end{prop}
\begin{proof} $\R \subseteq \RE$, ok

 \begin{enumerate}
  \item $\id \in \rec$, ok
  \item $f, g \in \rec \implies f \circ g \in rec$
  \item Siano $f \in rec$, $B \in \mathcal R$. 
  
  La funzione caratteristica di $\{x : f(x) \in B\}$ è $\chi_B \circ f$, calcolabile totale ($\in rec$) perché composizione di funzioni calc tot.
  \item Siano $f \in rec$, $B \in \mathcal {RE}$. 
  
  La funzione \emph{semicaratteristica} di  $A=\{x : f(x) \in B\}$ è  calcolabile, quindi $A$ è dominio di una funzione calcolabile $\implies \in \RE$.
 \end{enumerate}

\end{proof}

  
 \begin{center}
\begin{tikzpicture} 
 \draw (-2.5, -2.5) rectangle (2.5,2.5);
 \draw (0,0) circle (2);
 \draw (.3,0) circle (1.2);
 \node () at (1.6,1.6) {$\RE$};
 \node () at (2,2.8) {$\mathcal H$};
 \node () at (-1,.6) {$\R$};
 \node (b) at (-.3,.5){$B$};
 \node (d) at (.6,.7){$D$};
 \node (c) at (1,-1.5){$C$};

 \draw[-latex] (b) -- (d);
 \draw[-latex] (d) -- (c);
 \draw[-latex, dashed] (b) -- (c);
 \draw[{latex}{Rays}-, dashed] (-.4,-.4) -- (-.8, -.8);
\end{tikzpicture}
\end{center}
 
 \end{multicols}

\end{mdframed}

\begin{mdframed}
 \begin{teo}
  $K$ è $\mathcal{RE}$-completo
 \end{teo}
 \begin{proof}
  Sappiamo che $K \in \RE$. Dobbiamo mostrare che $A \in \RE \implies A \leq K$\smallskip
  
  $A \in \RE \implies A$ è il dominio di una qualche funzione calcolabile $\psi$. \smallskip
  
  
  Costruiamo adesso $\psi' = \lambda x, y . \psi(x)$; adesso usiamo \emph{il solito barbatrucco}:
  
  \[ \psi'(x, y) = \varphi_i(x, y) = \varphi_{s(i, x)}(y) \]
  
  Allora $A = \{ x \mid \varphi_{s(i, x)}(y) \downarrow \}$, e ponendo $y = s(i, x)$ si ottiene:
  
  \[A = \{ x \mid \varphi_{s(i, x)}(s(i, x)) \downarrow \} = \{x \mid s(i, x) \in K\}\]
  
 Quindi, sia $f = \lambda x.s(i, x)$:\smallskip
 
 
 $x \in A \iff f(x) \in K$, ed $f$ è calcolabile totale per il teorema del parametro (barbatrucco)
 \end{proof}

\end{mdframed}
\newpage
\begin{mdframed}
 \begin{teo}[Lemma]
  Sia $A$ un iirf tc $\emptyset \neq A \neq \naturals$, allora: \[K \leq A \text{ oppure }K \leq \bar A\]
 \end{teo}
 \begin{proof}
  Sia $i_0 : \varphi_{i_0} = \lambda y. \text{ind}$
  
  \begin{itemize}
   \item $i_0 \in \bar A$: Poiché $A \neq \emptyset$ esiste un $i_1 \in A$, e si ha $\varphi_{i_1} \neq \varphi_{i_0} $ poiché $A$ è iirf. Definiamo ora:
   
   \[ \psi_(x, y) = ... = \varphi_{f(x)}(y) = \begin{cases}
                                               \varphi_{i_1} (y) & x \in K\\
                                               \text{ind} = \varphi_{i_0}(y) & \text{altrimenti}
                                              \end{cases}
 \]
 
  \begin{itemize}
   \item $x \in K \implies \varphi_{f(x)} = \varphi_{i_1}$, e dato che $i_1 \in A$ ($A$ iirf) vale $f(x) \in A$
  
   \item $x \notin K \implies \varphi_{f(x)} = \varphi_{i_0}$, e dato che $i_0 \notin A$ ($A$ iirf) vale $f(x) \not\in A$  
  \end{itemize}
Quindi si ha che $K \leq A$.
  \item $i_0 \in A$: Poiché $A \neq \naturals$ esiste $i_1 \notin A$; stessa $\psi$: 
  \begin{itemize}
   \item $x \in K \implies \varphi_{f(x)} = \varphi_{i_1}$, e dato che $i_1 \notin A$ ($A$ iirf) vale $f(x) \notin A$
   \item $x \notin K \implies \varphi_{f(x)} = \varphi_{i_0}$, e dato che $i_0 \in A$ ($A$ iirf) vale $f(x) \in A$
   \end{itemize}
Quindi si ha che $K \leq \bar A$.
  \end{itemize}
 \end{proof}

\end{mdframed}

\begin{mdframed}
 \begin{teo}[Rice]
  Sia $\mathcal F$ una classe di funzioni calcolabili, e $I = \{ n | \varphi_n \in \mathcal F \}$ (iirf). Questo insieme è ricorsivo se e solo se $\mathcal{A} = \emptyset$ oppure è la classe di tutte le funzioni calcolabili.
 \end{teo}
 \begin{proof}
  $I$ è iirf, quindi se diverso da $\emptyset$ e $\naturals$ non può essere ricorsivo, poiché per il lemma valgono una tra:
  \begin{itemize}
   \item $K \leq A$, $K$ non ricorsivo $\implies$ $A$ non ricorsivo
   \item $K \leq \bar A \implies \bar K \leq A$, $\bar K$ non ricorsivo $\implies$ $A$ non ricorsivo.
  \end{itemize}
  
  $\emptyset$ e $\naturals$ sono banalmente ricorsivi.

 \end{proof}

\end{mdframed}

\end{document}

