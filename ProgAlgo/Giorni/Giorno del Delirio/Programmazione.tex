\documentclass[a4paper,10pt]{article}
\usepackage[italian]{babel} 
\usepackage[T1]{fontenc} 
\usepackage[utf8]{inputenc} 
\usepackage{amsfonts}
\usepackage{amsthm}
\usepackage[margin=25mm]{geometry}
\usepackage{mathtools}
\usepackage{listings}
\usepackage{centernot}
\usepackage{multicol}
\usepackage{pgfplots}
\usetikzlibrary{backgrounds}
%opening
\title{Giorno 3}
\author{Lorenzo Pace}

\theoremstyle{definition}
\newtheorem{deff}{Def.}[section]
\newtheorem{lemma}[deff]{Lemma}
\newtheorem{es}[deff]{Es.}
\newtheorem{ex}[deff]{Esercizio}
\newtheorem{teo}[deff]{Teorema}
\newtheorem{prop}[deff]{Proposizione}
\lstset
{ %Formatting for code in appendix
    language=Pascal,
    otherkeywords={print, ref}, 
    basicstyle=\footnotesize,
    numbers=left,
    stepnumber=1,
    showstringspaces=false,
    tabsize=1,
    breaklines=true,
    breakatwhitespace=false,
}
\renewcommand{\qedsymbol}{\rule{1ex}{1ex}}
\setlength{\parindent}{0em}
\usetikzlibrary{shapes,snakes}
\begin{document}

\begin{center}
    \LARGE Giorno del Delirio
    
    \large (Programmazione)
\end{center}
\section{Record di attivazione}
\begin{enumerate}
 \item \textbf{Chi sono?} (nome programma/funzione)
 \item \textbf{Da dove vengo?} (Catena dinamica)
 \item \textbf{Come torno a casa?} (Indirizzo di ritorno)
 \item \textbf{Cosa faccio?} (Risultato della chiamata)
 \item \textbf{A chi lascio quello che faccio?} (Indirizzo Risultato)
 \item \textbf{Cosa possiedo?}
 \begin{itemize}
  \item Catena statica
  \item Parametri
  \item Variabili locali
 \end{itemize}

\end{enumerate}
\begin{center}
\LARGE
\begin{tikzpicture}[scale=2, framed, every node/.style = {shape=rectangle, align=center, circle, draw=black!60, thick, minimum width=30mm}, level 1/.style={sibling distance=15em}]
    \node[rectangle, thin, black , fill=lightgray] at (0,0) {Statica};
    \node[rectangle, thin, white , fill=darkgray] at (0,1) {Ris.};
    \node[rectangle, thin, black , fill=lightgray] at (0,2) {Ris.\\ Chiamata};
    \node[rectangle, thin, black , fill=lightgray] at (2,1) {Dinamica};
    \node[rectangle, thin, black , fill=lightgray] at (2,2) {Nome};
    \node[rectangle, thin, black , fill=lightgray] at (4,0) {Var. locali};
    \node[rectangle, thin, black , fill=lightgray] at (4,1) {Parametri};
    \node[rectangle, thin, white , fill=darkgray] at (4,2) {Ritorno};
\end{tikzpicture}

\end{center}

\end{document}
