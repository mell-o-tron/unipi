\documentclass[a4paper,10pt]{article}
\usepackage[italian]{babel} 
\usepackage[T1]{fontenc} 
\usepackage[utf8]{inputenc} 
\usepackage{amsfonts}
\usepackage{amsthm}
\usepackage[margin=25mm]{geometry}
\usepackage{mathtools}
\usepackage{listings}
\usepackage{centernot}
\usepackage{multicol}
\usepackage{pgfplots}
%opening
\title{Giorno 2}
\author{Lorenzo Pace}

\theoremstyle{definition}
\newtheorem{deff}{Def.}[section]
\newtheorem{lemma}[deff]{Lemma}
\newtheorem{es}[deff]{Es.}
\newtheorem{ex}[deff]{Esercizio}
\newtheorem{teo}[deff]{Teorema}
\newtheorem{prop}[deff]{Proposizione}
\lstset
{ %Formatting for code in appendix
    language=Pascal,
    otherkeywords={print, ref}, 
    basicstyle=\footnotesize,
    numbers=left,
    stepnumber=1,
    showstringspaces=false,
    tabsize=1,
    breaklines=true,
    breakatwhitespace=false,
}
\renewcommand{\qedsymbol}{\rule{1ex}{1ex}}
\begin{document}

\begin{center}
    \LARGE Giorno 2
\end{center}
\section{Divide et Impera}
L'approccio Divide et Impera consiste nel dividere un problema in sottoproblemi che sono versioni ``più piccole'' dello stesso. Si applicano tre passi:
\begin{enumerate}
    \item \textbf{Divide}: Si divide il problema in sottoproblemi
    \item \textbf{Impera}: Si applica ricorsivamente l'algoritmo per risolvere ognuno dei sottoproblemi
    \item \textbf{Combina}: Si combinano le soluzioni dei sottoproblemi per generare la soluzione al problema iniziale
\end{enumerate}
\subsection{Merge Sort}
Un esempio di algoritmo divide et impera è Merge Sort, un algoritmo di sorting ricorsivo.
\begin{multicols}{2}
\begin{quote}
 
\begin{lstlisting}[escapeinside={(?}{?)}]
MS(A, p, r):
    if(r > p):
        q = (? $\bigg\lfloor \dfrac{p+r}{2} \bigg\rfloor$ ?);
        MS(A, p, q);
        MS(A, q+1, r);
        Merge(A, p, q, r);
\end{lstlisting}

\end{quote}

\bigskip
\bigskip
\bigskip
\bigskip
\bigskip
Costo di Merge: $\Theta(n)$

 \begin{lstlisting}[escapeinside={(?}{?)}]
Merge(A, p, q, r):
    n1 = q - p + 1;
    n2 = r - q;
    L = nuovo array di lunghezza n1+1;
    R = nuovo array di lunghezza n2+1;
    for(i = 1 to n1): L[i] = A[p+i-1];
    for(i = 1 to n2): R[i] = A[q+i];
    L[n1 + 1] = (?$+\infty$?);
    R[n2 + 1] = (?$+\infty$?);
    i = 1; j = 1;
    for(k = p to r):
        if(L[i] (?$\geq$?) R[j]):
            A[k] = R[j]; j++;
        else:
            A[k] = L[i]; i++;
\end{lstlisting}
\end{multicols}
\subsubsection{Invariante di ciclo di Merge}
All'inizio della $k$-esima iterazione, $A[p...k-1]$ contiene, ordinati i $k-p$ elementi più piccoli di $L$ e $R$.

\begin{enumerate}
    \item \textbf{Inizializzazione}: Prima della prima iterazione si ha $k = p$, e zero elementi sono banalmente ordinati.
    \item \textbf{Conservazione}: Ad ogni iterazione si inserisce in $A[k]$ o il primo elemento di $L[i, n_1]$ o il primo elemento di $R[j, n_2]$. Tali elementi sono rispettivamente il minimo di $L[i, n_1]$ e quello di $R[j, n_2]$ poiché i due array sono ordinati.
    \item \textbf{Conclusione}: L'array $A$, alla fine dell'ultima iterazione, è ordinato poiché tutti gli elementi sono stati copiati tranne le sentinelle ($+\infty$).
\end{enumerate}

\newpage
\subsection{Ricorrenze}
Una ricorrenza è un'equazione o disequazione che definisce una funzione in termini del suo valore con input più piccoli. Ad esempio, la ricorrenza che definisce la complessità in tempo di Merge Sort è:
\[T(n) = \begin{cases}
            \Theta(1) &\text{Se $n = 1$}\\
            2T\bigg(\dfrac{n}{2}\bigg) + \Theta(n) &\text{se $n > 1$}
         \end{cases}
\]
\subsection{Metodi di risoluzione delle ricorrenze}
\subsubsection{Metodo di sostituzione}
Si ipotizza una possibile soluzione e la si dimostra per induzione.

\subsubsection{Metodo dell'albero di ricorsione}
Si rappresentano i costi dei sottoproblemi in alberi: 
Si consideri ad esempio la ricorrenza di Merge Sort, che viene rappresentata come:\smallskip
\begin{multicols}{2}
All' $i$-esimo livello il costo è \[2^{i-1}\cdot\Theta\bigg(\dfrac{n}{2^{i-1}}\bigg) = \Theta(n)\]

Sia $h$ l'altezza dell'albero: si ha che \[\begin{aligned}
T\bigg(\dfrac{n}{2^{h-1}}\bigg) = T(1) &\iff \dfrac{n}{2^{h-1}} = 1\\
&\iff n = 2^{h-1} \\&\iff h-1 = \log_2n
\end{aligned}\] 
Quindi l'altezza dell'albero è $h = \log_2 n + 1$. Per trovare il costo totale si sommano i costi ad ogni livello:


\begin{tikzpicture}[every node/.style = {shape=rectangle, align=center}, level 1/.style={sibling distance=7em},
  level 2/.style={sibling distance=4em},
  level 3/.style={sibling distance=2em},
  level 4/.style={sibling distance=1em}]]
  \node (Root) {$cn$}
    child { node {$\frac{cn}{2}$} 
        child {node {$\hdots$} 
            child[dashed] {node {T(1)}}}
        child {node {$\hdots$} 
        child[dashed] {node {T(1)}}}}
     child { node {$\frac{cn}{2}$} 
        child {node {$\hdots$} child[dashed] {node {T(1)}}}
        child {node {$\hdots$} child[dashed] {node {T(1)}}}};
\path (Root)  ++(4cm,0) node{$\Theta(n)$};
\path (Root-2 -| Root)  ++(4cm,0) node{$2\cdot\Theta(\frac{n}{2})$};
\path (Root-2-2 -| Root)  ++(4cm,0) node{$4\cdot\Theta(\frac{n}{4})$};
\path (Root-2-2-1 -| Root)  ++(4cm,0) node{$2^{h-1}\cdot\Theta(\frac{n}{2^{h-1}})$};
\end{tikzpicture}

\end{multicols}
\[\sum_{i = 0}^{\log_2 n + 1}2^{i-1}\cdot\Theta\bigg(\dfrac{n}{2^{i-1}}\bigg) = \sum_{i = 0}^{\log_2 n + 1}\Theta(n) = \Theta(n) \sum_{i = 0}^{\log_2 n + 1} 1 = \Theta(n) \cdot \log_2 (n) + 1 = \Theta(n \log n)\] 

\subsubsection{Master Theorem}
\begin{teo}
    Data una ricorrenza della forma $T(n) = a \cdot T\bigg(\dfrac{n}{b}\bigg) + f(n)$, $b \neq 0$:
    
    \begin{enumerate}
     \item Se $f(n) = O(n^{\log_b a - \varepsilon})$ per qualche $\varepsilon > 0$, allora $T(n) = \Theta(n^{\log_b a})$
     \item Se $f(n) = \Theta(n^{\log_b a})$ allora $T(n) = \Theta(n^{\log_b a} \log n)$
     \item 
     \begin{itemize}
      \item $f(n) = \Omega(n^{\log_b a + \varepsilon})$ per qualche $\varepsilon > 0$
      \item $af\bigg(\dfrac{n}{b}\bigg) \leq cf(n)$ definitivamente per qualche $c<1$
     \end{itemize}
    $\implies T(n) = \Theta(f(n))$
    \end{enumerate}

\end{teo}


\newpage \section{Quicksort}
Quicksort suddivide ricorsivamente un array in due partizioni, una che contiene solo elementi maggiori di un certo elemento detto \emph{pivot}, ed un altro che contiene solo gli elementi ad esso maggiori. Il pivot viene inserito tra le due partizioni, nella sua posizione finale.

\begin{multicols}{2}
\begin{quote}
 
\begin{lstlisting}[escapeinside={(?}{?)}]
QS(A, p, r):
    if(r > p):
        q = Partizion(A, p, r)
        QS(A, p-1, q);
        QS(A, q+1, r);
\end{lstlisting}

\end{quote}

\bigskip
\bigskip
\bigskip
\bigskip
\bigskip
Costo di Partition: $\Theta(n)$

 \begin{lstlisting}[escapeinside={(?}{?)}]
Partition(A, p, r):
    x = A[r]
    i = p-1; j = p;
    while(j<r):
        if(A[j] (?$\leq$?) x):
            i++;
            scambia A[i], A[j];
        j++;
    scambia A[i+1], A[r]
    return i+1;

\end{lstlisting}
\end{multicols}
\subsubsection{Invariante di ciclo di Partition}
All'inizio della $j$ esima iterazione:
\begin{enumerate}
 \item Il sottoarray $A[p .. i]$ contiene solo elementi minori di $x$;
 \item Il sottoarray $A[i+1 .. j-1]$ contiene solo elementi maggiori di $x$;
 \item $A[r] = x$.
\end{enumerate}
\begin{enumerate}
    \item \textbf{Inizializzazione}: Prima della prima iterazione i due sottoarray sono vuoti, quindi le proprietà sono vacuamente soddisfatte;
    \item \textbf{Conservazione}: Ad ogni iterazione, se $A[j]$ è minore di $x$ si incrementa $i$ e si scambiano $A[i] ed A[j]$ (proprietà 1), mentre se $A[j]$ è maggiore di $x$, si incrementa semplicemente $j$ (proprietà 2). 
    \item \textbf{Conclusione}: Alla fine del ciclo, $A[p .. i]$ e $A[i+1 .. r-1]$ sono le due partizioni, e $A[r]$ vale ancora $x$.
\end{enumerate}

\subsection{Complessità di Quicksort}
La complessità di QS varia tra il caso ottimo, il caso pessimo ed il caso medio.
\subsubsection{Caso ottimo}
Al caso ottimo l'albero di ricorsione è bilanciato (il pivot è sempre il valor medio dell'array). In questo caso la ricorrenza che caratterizza $T(n)$ è:
\[T(n) \leq \begin{cases}
            \Theta(1) &\text{Se $n = 1$}\\
            2T\bigg(\dfrac{n}{2}\bigg) + \Theta(n) &\text{se $n > 1$}
         \end{cases}
\]
La cui soluzione sappiamo essere (per il teorema dell'esperto) $T(n) = n \log n$
\subsubsection{Caso pessimo}
Al caso pessimo l'albero di ricorsione è molto sbilanciato, poiché il pivot è sempre maggiore o minore di tutti gli elementi (quindi una delle partizioni è sempre vuota). Si ha che:
\[T(n) = \begin{cases}
            \Theta(1) &\text{Se $n = 1$}\\
            T(n-1) + T(0) + \Theta(n) &\text{se $n > 1$}
         \end{cases}
\]

L'albero di ricorsione ha profondità $n$, ed il costo ad ogni livello è $\Theta(n)$, perciò si ha un costo totale di $n \cdot \Theta(n) = \Theta(n^2)$

\subsubsection{Ripartizione bilanciata}
Nel caso di ripartizione bilanciata, dato $a<1$, si ha:
\[T(n) \leq \begin{cases}
            \Theta(1) &\text{Se $n = 1$}\\
            T(a\cdot n) + T((1-a)\cdot n) + \Theta(n) &\text{se $n > 1$}
         \end{cases}
\]
Quindi l'albero di ricorsione, assumendo ad esempio $a>\dfrac{1}{2}$, ha questa forma:

\begin{center}
 
\begin{tikzpicture}[every node/.style = {shape=rectangle, align=center}, level 1/.style={sibling distance=17em},
  level 2/.style={sibling distance=7em},
  level 3/.style={sibling distance=4em},
  level 4/.style={sibling distance=2em}]]
  \node (Root) {$cn$}
    child { node {$ca\cdot n$} 
        child {node {$ca^2\cdot n$} 
            child[dashed] {node {$T(1)$}}
            child[dashed] {node {}}}
        child {node {$c(a - a^2)n$} 
        child[dashed] {node {}}
        child[dashed] {node {}}}}
     child { node {$c(1-a)\cdot n$} 
        child {node {$c(a - a^2)n$} child[dashed] {node {}}child[dashed] {node {}}}
        child {node {$c(1-a)^2 \cdot n$} child[dashed] {node {}} 
        child {node {$c(1-a)^3$}child[dashed] {node {}} child[dashed] {node {}child[missing] {node {}}child[dashed] {node {$T(1)$}}} } }};
\path (Root)  ++(7cm,0) node{$cn$};
\path (Root-2 -| Root)  ++(7cm,0) node{$cn$};
\path (Root-2-2 -| Root)  ++(7cm,0) node{$cn$};
\path (Root-2-2-1 -| Root)  ++(7cm,0) node{$cn$};
\path (Root-2-2-2-1 -| Root)  ++(7cm,0) node{$\leq cn$};
\path (Root-2-2-2-2-2 -| Root)  ++(7cm,0) node{$\leq cn$};
\end{tikzpicture}
\end{center}

(e ok do go on)
\end{document}
