\documentclass[a4paper,10pt]{article}
\usepackage[italian]{babel} 
\usepackage[T1]{fontenc} 
\usepackage[utf8]{inputenc} 
\usepackage{amsfonts}
\usepackage{amsthm}
\usepackage[margin=25mm]{geometry}
\usepackage{mathtools}
\usepackage{listings}
\usepackage{centernot}
\usepackage{multicol}
\usepackage{pgfplots}
%opening
\title{Giorno 1}
\author{Lorenzo Pace}

\theoremstyle{definition}
\newtheorem{deff}{Def.}[section]
\newtheorem{lemma}[deff]{Lemma}
\newtheorem{es}[deff]{Es.}
\newtheorem{ex}[deff]{Esercizio}
\newtheorem{teo}[deff]{Teorema}
\newtheorem{prop}[deff]{Proposizione}
\lstset
{ %Formatting for code in appendix
    language=Pascal,
    otherkeywords={print, ref}, 
    basicstyle=\footnotesize,
    numbers=left,
    stepnumber=1,
    showstringspaces=false,
    tabsize=1,
    breaklines=true,
    breakatwhitespace=false,
}
\renewcommand{\qedsymbol}{\rule{1ex}{1ex}}
\begin{document}

\begin{center}
    \LARGE Giorno 1
\end{center}


\section{Notazioni asintotiche}

\begin{multicols}{2}

\begin{deff}
    Date due funzioni $f(n)$ e $g(n)$, si dice che $f(n) = O(g(n))$ se esistono $c \neq 0$ ed $n_0$ tali che:
    \[\forall n > n_0 : 0 \leq f(n) \leq c g(n)\]
\end{deff}
\smallskip

\begin{deff}
 Date due funzioni $f(n)$ e $g(n)$, si dice che $f(n) = \Omega(g(n))$ se esistono $c \neq 0$ ed $n_0$ tali che:
 \[\forall n > n_0 : 0 \leq cg(n) \leq f(n)\]
\end{deff}
\begin{center}

\begin{tikzpicture}[scale = .6]
\begin{axis}[
    axis lines = left,
    xlabel = {},
    ylabel = {},
]
%Below the red parabola is defined
\addplot [
    domain=2:40, 
    samples=100, 
    ]
    {sqrt(x)};
\addplot [
    domain=2:40, 
    samples=100, 
    ]
    {ln(x)};
\addlegendentry{$f(n) = \Omega (g(n))$}
\addlegendentry{$cg(n)$}

\end{axis}
\end{tikzpicture}
    \end{center}

\end{multicols}

\begin{multicols}{2}

\begin{deff}
    Date due funzioni $f(n)$ e $g(n)$, si dice che $f(n) = \Theta(g(n))$ se \[f(n) = O(g(n)), f(n) = \Omega(g(n))\]
    Ossia se: 
    \[\exists c_1, c_2 \neq 0, n_0 : \forall n > n_0 : c_1g(n) \leq f(n) \leq c_2g(n)\]
\end{deff}
\bigskip
\bigskip
\begin{center}

\begin{tikzpicture}[scale = .6]
\begin{axis}[
    axis lines = left,
    xlabel = {},
    ylabel = {},
]
%Below the red parabola is defined
\addplot [
    domain=2:40, 
    samples=100, 
    ]
    {2.5*ln(x)};
\addplot [
    domain=2:40, 
    samples=100, 
    ]
    {2*ln(x)};
\addplot [
    domain=2:40, 
    samples=100, 
    ]
    {1.5*ln(x)};
\addlegendentry{$c_1 g(n)$}
\addlegendentry{$f(n) = \Theta(g(n))$}
\addlegendentry{$c_2 g(n)$}
\end{axis}
\end{tikzpicture}
    \end{center}

\end{multicols}

\section{Limiti Superiori ed Inferiori}
\subsection{Definizioni}
\begin{deff}
    Dato un problema $\Pi$, se esiste un algoritmo $A$ che risolve $\Pi$ in tempo $t(n)$, allora $t(n)$ è un \emph{limite superiore} alla complessità in tempo di $\Pi$. 
\end{deff}
\begin{deff}
    Dato un problema $\Pi$, se ogni algoritmo $A$ che lo risolve deve impiegare necessariamente almeno tempo $t(n)$, allora $t(n)$ è un \emph{limite inferiore} alla complessità in tempo di $\Pi$. 
\end{deff}
\begin{deff}
    Dato un problema $\Pi$, se un algoritmo $A$ che lo risolve impiega tempo $t(n)$, e $t(n)$ è un limite inferiore alla complessità in tempo di $\Pi$, allora si dice che $A$ è un \emph{algoritmo ottimo}. 
\end{deff}
\subsection{Metodi per individuare limiti inferiori}
\begin{enumerate}
 \item \textbf{Dimensione dei dati}: Se per risolvere un problema si deve necessariamente accedere a tutti i dati in input, la loro dimensione è un limite inferiore alla complessità in tempo del problema. \smallskip
 
 \emph{Esempio}: Per cercare l'elemento massimo di un array $A$ di dimensione $n$ si deve scansionare l'intero $A$ (il minimo potrebbe essere in ognuna delle $n$ posizioni).
  \item \textbf{Eventi contabili}: Se al fine della risoluzione del problema è necessario che un particolare evento si verifichi un numero $f(n)$ di volte, allora $f(n)$ è un limite inferiore alla complessità in tempo del problema.\smallskip
  
  \emph{Esempio}: Se voglio generare le permutazioni di un insieme di $n$ numeri, devo, ovviamente, generare \emph{ogni} permutazione. La generazione di una delle $n!$ permutazioni è perciò un evento contabile, e $n!$ è un limite inferiore alla complessità del problema.
  \item \textbf{Albero di decisione}:
  Si rappresentano tutte le possibili ``decisioni'' nei nodi interni di un albero. I possibili esiti di ogni nodo-decisione sono i suoi figli ed i risultati sono rappresentati nelle foglie. L'altezza dell'albero è un limite inferiore alla complessità del problema.\smallskip
  
  \emph{Esempio}: Il limite inferiore per la ricerca binaria dell'elemento $k$ su un array di $n$ elementi è uguale all'altezza di un albero binario con $n$ foglie ($\log n$), poiché ogni scelta ha due possibili esiti (due figli), ed ognuno degli $n$ elementi può essere $k$.
\end{enumerate}

\section{Analisi di algoritmi}
\subsection{Ricerca}
\begin{multicols}{2}
\subsubsection{Ricerca sequenziale}
Sia $A$ un array di lunghezza $n$. Devo cercare la chiave $k$. 
\begin{quote}\begin{lstlisting}[escapeinside={(?}{?)}]
	SeqSearch(A, k):
        for(i = 1 to n): (?$O(n)$?)
            if(i == k) return i;
        return -1;              
\end{lstlisting}\end{quote}


Costo al caso ottimo $O(1)$;

Costo al caso pessimo $O(n)$, algoritmo ottimo. 
\bigskip
\bigskip
\bigskip


\subsubsection{Ricerca binaria}
Sia $A$ un array \textbf{ordinato} di lunghezza $n$. 

Devo cercare la chiave $k$. 

\begin{quote}\begin{lstlisting}[escapeinside={(?}{?)}]
	BinSearch(A, k):
        p = 1, r = n + 1;
        while(p (?$\leq$?) r):
            q = (?$\bigg\lfloor \dfrac{p+r}{2}\bigg\rfloor$?);
            if(A[q] == k) return q;
            if(k > q) p = q;
            else r = q;
        return -1;
\end{lstlisting}\end{quote}

Costo al caso ottimo $O(1)$;

Costo al caso pessimo $O(\log n)$, ottimo.

\end{multicols}

\subsection{Insertion, Selection Sort}
\begin{multicols}{2}
 
\subsubsection{Insertion Sort}
\begin{quote}\begin{lstlisting}[escapeinside={(?}{?)}]
	InsSort(A):
        for(j = 2 to n):
            k = A[j];
            i = j-1;
            while(i > 0 && A[i] > k):
                A[i+1] = A[i];
                i--;
            A[i+1] = k;
\end{lstlisting}\end{quote}


\textbf{Invariante di ciclo}: Ad ogni iterazione si ha che i primi $j-1$ elementi sono ordinati.
\begin{enumerate}
 \item Inizializzazione: Un solo elemento è banalmente ordinato
 \item Conservazione: Informalmente -- si inserisce il $j$-esimo elemento nel sottoarray $A[1,...,j-1]$.
 \item Alla fine il sottoarray corrisponde all'intero array, che è perciò ordinato.
\end{enumerate}

Costo al caso ottimo: $O(n)$;

Costo al caso pessimo: $O(n^2)$, non ottimo.

\end{multicols}
\begin{multicols}{2}
\subsubsection{Selection Sort}
\begin{quote}\begin{lstlisting}[escapeinside={(?}{?)}]
	SelSort(A):
        for(i = 1 to n):
            posmin = i;
            for(j = i+1 to n):
                if(A[j] < A[posmin])
                    posmin = j;
            if(A[i] > A[posmin]):
                tmp = A[i];
                A[i] = A[posmin];
                A[posmin] = tmp;
\end{lstlisting}\end{quote}
\begin{itemize}
 \item[]
 \item[]
 \item[]
 \item[]
\end{itemize}

Costo al caso ottimo: $O(n^2)$;

Costo al caso pessimo: $O(n^2)$, non ottimo.
\end{multicols}

\textbf{Invariante di ciclo}: Ad ogni iterazione si ha che i primi $i-1$ elementi sono ordinati e sono nella loro posizione finale
\begin{enumerate}
\item Il sottoarray vuoto è banalmente ordinato
 \item Conservazione: Ad ogni iterazione si prende il minimo del sottoarray $[i, ... , n]$ e lo si scambia con l'$i$-esimo elemento. 
 
 Il sottoarray $A[1, ..., i-1]$ è ordinato per ipotesi induttiva, e dato che ogni suo elemento è nella sua posizione finale si ha che tutti gli elementi di $A[i, ... , n]$ sono maggiori di $A[i-1]$. Se ad $A[1, ..., i-1]$ si \emph{append}e il minimo di $A[i, ... , n]$, allora il sottoarray $A[1, ..., i]$ è necessariamente ordinato, e l'elemento $A[i]$ è nella sua posizione finale (non esiste elemento $< A[i]$ in $A[i+1, n]$).
 \item Alla fine il sottoarray $[1, i]$ corrisponde all'intero array, che è perciò ordinato.
\end{enumerate}
\section{Limite inferiore per l'ordinamento (per confronti)}
\subsection{Limite inferiore $n \log n$}
$n\log n$ è limite inferiore per l'ordinamento.
\begin{proof}
Sia $A$ un array di $n$ elementi. Usiamo il metodo dell'albero di decisione per stabilire un limite inferiore per l'ordinamento, ed assumiamo w.l.o.g. che tutti gli elementi di A siano distinti.

\begin{enumerate}
 \item Usiamo come confronto la relazione $\leq$. Questa ha due possibili esiti, perciò l'albero sarà binario. 
 \item Ogni algoritmo di ordinamento deve essere in grado di generare ognuna delle $n!$ permutazioni degli elementi di $A$. 
\end{enumerate}
Perciò si ha che il numero di risultati $l$ (rappresentati dalle foglie dell'albero di decisione) possibili è compreso tra $n!$ e $2^h$ (massimo di foglie per un albero binario): 
\[\begin{aligned}
&n! \leq l \leq  2^h   \\
\implies & \log(n!) \leq log(l) \leq h\\
\text{(log monotona continua) }\implies& h > \log(n!)\\ 
\text{(Approx di Stirling)}\implies& h > \Omega(n\log(n))\\ 
\end{aligned}
\]

\end{proof}

\end{document}
